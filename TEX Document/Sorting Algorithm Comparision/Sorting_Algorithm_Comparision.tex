\documentclass[14pt]{extarticle}
\usepackage{amsmath}
\usepackage{listings}
\usepackage{xcolor}
\usepackage{tcolorbox}
\usepackage{geometry}
\geometry{margin=1in}

\lstset{
    backgroundcolor=\color{white},  % or just remove this line
    basicstyle=\ttfamily\footnotesize,
    keywordstyle=\color{blue},
    commentstyle=\color{gray},
    stringstyle=\color{red},
    breaklines=true,
    frame=single,
    showstringspaces=false
}


\title{Explanation of Sorting Algorithm Visualizer in Streamlit}
\author{Varun Kumar}
\date{July 5, 2025}

\begin{document}
\maketitle

\section*{1. Importing Libraries}
\begin{lstlisting}[language=Python]
import streamlit as st
import time
import plotly.express as px
import heapq
\end{lstlisting}

\textbf{Explanation:}
\begin{itemize}
  \item \texttt{streamlit} is used to create the web interface.
  \item \texttt{time} module helps measure performance.
  \item \texttt{plotly.express} is used for bar chart plotting.
  \item \texttt{heapq} is used for implementing heap sort.
\end{itemize}

\newpage
\section*{2. Sorting Algorithm Implementations}
Each sorting function returns a sorted copy of the input array.

\subsection*{Bubble Sort}
\begin{lstlisting}[language=Python]
def bubble_sort(arr):
    a = arr.copy()
    n = len(a)
    for i in range(n):
        for j in range(0, n - i - 1):
            if a[j] > a[j + 1]:
                a[j], a[j + 1] = a[j + 1], a[j]
    return a
\end{lstlisting}

\textbf{Explanation:} Compares adjacent elements and swaps if needed; slow $O(n^2)$.

\subsection*{Insertion Sort}
\begin{lstlisting}[language=Python]
def insertion_sort(arr):
    a = arr.copy()
    for i in range(1, len(a)):
        key = a[i]
        j = i - 1
        while j >= 0 and key < a[j]:
            a[j + 1] = a[j]
            j -= 1
        a[j + 1] = key
    return a
\end{lstlisting}

\textbf{Explanation:} Builds sorted array by inserting elements one at a time.

\subsection*{Selection Sort}
\begin{lstlisting}[language=Python]
def selection_sort(arr):
    a = arr.copy()
    for i in range(len(a)):
        min_idx = i
        for j in range(i + 1, len(a)):
            if a[j] < a[min_idx]:
                min_idx = j
        a[i], a[min_idx] = a[min_idx], a[i]
    return a
\end{lstlisting}

\textbf{Explanation:} Finds the minimum and places it at correct position iteratively.

\newpage
\subsection*{Merge Sort}
\begin{lstlisting}[language=Python]
def merge_sort(arr):
    if len(arr) <= 1:
        return arr
    mid = len(arr) // 2
    L = merge_sort(arr[:mid])
    R = merge_sort(arr[mid:])
    return merge(L, R)

def merge(left, right):
    result = []
    i = j = 0
    while i < len(left) and j < len(right):
        if left[i] < right[j]:
            result.append(left[i])
            i += 1
        else:
            result.append(right[j])
            j += 1
    result.extend(left[i:])
    result.extend(right[j:])
    return result
\end{lstlisting}

\textbf{Explanation:} A divide-and-conquer algorithm that splits and merges sorted halves. Time complexity: $O(n \log n)$.

\subsection*{Quick Sort}
\begin{lstlisting}[language=Python]
def quick_sort(arr):
    if len(arr) <= 1:
        return arr
    pivot = arr[0]
    left = [x for x in arr[1:] if x < pivot]
    mid = [x for x in arr if x == pivot]
    right = [x for x in arr[1:] if x >= pivot]
    return quick_sort(left) + mid + quick_sort(right)
\end{lstlisting}

\textbf{Explanation:} Uses pivot element and partitions data recursively.

\subsection*{Heap Sort}
\begin{lstlisting}[language=Python]
def heap_sort(arr):
    h = []
    for value in arr:
        heapq.heappush(h, value)
    return [heapq.heappop(h) for _ in range(len(h))]
\end{lstlisting}

\textbf{Explanation:} Uses a min-heap to extract elements in sorted order.

\section*{3. Mapping Sorting Names to Functions}
\begin{lstlisting}[language=Python]
SORT_FUNCTIONS = {
    "Bubble Sort": bubble_sort,
    "Insertion Sort": insertion_sort,
    "Selection Sort": selection_sort,
    "Merge Sort": merge_sort,
    "Quick Sort": quick_sort,
    "Heap Sort": heap_sort,
}
\end{lstlisting}

\textbf{Explanation:} Dictionary to link names (dropdown) to actual sorting functions.

\section*{4. Streamlit UI}
\begin{lstlisting}[language=Python]
st.title("Sorting Algorithm Visualizer")
input_str = st.text_input("Enter array (comma-separated)", "5, 3, 8, 4, 2")
algo = st.selectbox("Choose sorting algorithm", list(SORT_FUNCTIONS.keys()) + ["All (Test Performance)"])
\end{lstlisting}

\textbf{Explanation:} 
\begin{itemize}
  \item Sets page title.
  \item Text input for the user to enter a list.
  \item Dropdown (selectbox) to choose an algorithm or benchmark all.
\end{itemize}

\section*{5. When \texttt{Sort} Button is Clicked}
\begin{lstlisting}[language=Python]
if st.button("Sort"):
    try:
        input_array = list(map(int, input_str.split(',')))
\end{lstlisting}

\textbf{Explanation:} On button click, input string is split and converted to a list of integers.

\newpage
\section*{6. If "All (Test Performance)" is Selected}
\begin{lstlisting}[language=Python]
if algo == "All (Test Performance)":
    st.subheader("Performance Comparison")
    results = []
    for name, func in SORT_FUNCTIONS.items():
        start = time.perf_counter()
        _ = func(input_array)
        duration = time.perf_counter() - start
        results.append({"Algorithm": name, "Time (s)": duration})
\end{lstlisting}

\textbf{Explanation:} 
\begin{itemize}
  \item Loops over all sorting algorithms.
  \item Records execution time using \texttt{time.perf\_counter()}.
\end{itemize}

\subsection*{Plotting Performance}
\begin{lstlisting}[language=Python]
    fig = px.bar(
        results,
        x="Algorithm",
        y="Time (s)",
        hover_data={"Time (s)": ":.6f"},
        text="Time (s)",
        title="Sorting Algorithm Time Comparison",
    )
    fig.update_traces(texttemplate="%{text:.6f}", textposition="outside")
    st.plotly_chart(fig, use_container_width=True)
\end{lstlisting}

\textbf{Explanation:} Uses Plotly to render a bar chart showing performance of each algorithm.

\newpage
\section*{7. If Specific Algorithm is Selected}
\begin{lstlisting}[language=Python]
else:
    sort_func = SORT_FUNCTIONS[algo]
    start = time.perf_counter()
    sorted_array = sort_func(input_array)
    duration = time.perf_counter() - start

    st.success(f"Sorted Array: {sorted_array}")
    st.info(f"Time Taken: {duration:.6f} seconds")
\end{lstlisting}

\textbf{Explanation:}
\begin{itemize}
  \item Applies the selected sorting function.
  \item Displays sorted array and time taken.
\end{itemize}

\section*{8. Error Handling}
\begin{lstlisting}[language=Python]
except Exception as e:
    st.error(f"Error: {e}")
\end{lstlisting}

\textbf{Explanation:} Catches errors like invalid input and shows an error message to the user.

\end{document}
