\documentclass{article}
\usepackage{amsmath}
\usepackage{graphicx}
\usepackage{array}
\usepackage{booktabs}
\usepackage{circuitikz}
\usetikzlibrary{circuits.logic.IEC} % Correct library for logic gates

\title{SR Latch - Complete Summary with Circuits}
\author{Your Name}
\date{\today}

\begin{document}

\maketitle

\section{Introduction}
An \textbf{SR Latch (Set-Reset Latch)} is a basic \textbf{bistable} sequential circuit that stores \textbf{1 bit} of data.

\section{NOR-based SR Latch}
\subsection{Circuit Diagram}
\begin{center}
\begin{circuitikz}[circuit logic IEC]
    \node[nor gate] (nor1) at (0,0) {};
    \node[nor gate] (nor2) at (0,-2) {};
    
    \draw (nor1.input 1) -- ++(-0.5,0) node[left]{$S$};
    \draw (nor2.input 2) -- ++(-0.5,0) node[left]{$R$};
    
    \draw (nor1.output) -- ++(0.5,0) node[right]{$Q$};
    \draw (nor2.output) -- ++(0.5,0) node[right]{$\overline{Q}$};
    
    \draw (nor1.output) -- ++(0.3,0) |- (nor2.input 1);
    \draw (nor2.output) -- ++(0.3,0) |- (nor1.input 2);
\end{circuitikz}
\end{center}

\subsection{Truth Table}
\begin{center}
\begin{tabular}{ccccc}
\toprule
\textbf{S} & \textbf{R} & \textbf{Q} & \textbf{Q'} & \textbf{State} \\
\midrule
0 & 0 & Q & Q' & Hold \\
1 & 0 & 1 & 0 & Set \\
0 & 1 & 0 & 1 & Reset \\
1 & 1 & 0 & 0 & Invalid \\
\bottomrule
\end{tabular}
\end{center}

\section{NAND-based SR Latch}
\subsection{Circuit Diagram}
\begin{center}
\begin{circuitikz}[circuit logic IEC]
    \node[nand gate] (nand1) at (0,0) {};
    \node[nand gate] (nand2) at (0,-2) {};
    
    \draw (nand1.input 1) -- ++(-0.5,0) node[left]{$\overline{S}$};
    \draw (nand2.input 2) -- ++(-0.5,0) node[left]{$\overline{R}$};
    
    \draw (nand1.output) -- ++(0.5,0) node[right]{$Q$};
    \draw (nand2.output) -- ++(0.5,0) node[right]{$\overline{Q}$};
    
    \draw (nand1.output) -- ++(0.3,0) |- (nand2.input 1);
    \draw (nand2.output) -- ++(0.3,0) |- (nand1.input 2);
\end{circuitikz}
\end{center}

\subsection{Truth Table}
\begin{center}
\begin{tabular}{ccccc}
\toprule
$\overline{S}$ & $\overline{R}$ & Q & Q' & \textbf{State} \\
\midrule
1 & 1 & Q & Q' & Hold \\
0 & 1 & 1 & 0 & Set \\
1 & 0 & 0 & 1 & Reset \\
0 & 0 & 1 & 1 & Invalid \\
\bottomrule
\end{tabular}
\end{center}

\end{document}