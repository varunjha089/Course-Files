\documentclass{article}
\usepackage{amsmath}
\usepackage{graphicx}
\usepackage{array}
\usepackage{booktabs}

\title{SR Latch - Complete Summary}
\author{Your Name}
\date{\today}

\begin{document}

\maketitle

\section{Introduction}
An \textbf{SR Latch (Set-Reset Latch)} is a basic \textbf{bistable} (two stable states) sequential circuit that stores \textbf{1 bit} of data. It is constructed using \textbf{cross-coupled NOR gates} or \textbf{NAND gates}, forming a feedback loop to maintain its state.

\section{Types of SR Latches}
\begin{itemize}
    \item \textbf{NOR-based SR Latch}
    \item \textbf{NAND-based SR Latch} (also called \textbf{SR Latch with active-low inputs})
\end{itemize}

\section{NOR-based SR Latch}
\subsection{Structure}
Two NOR gates with feedback.

\subsection{Inputs}
\begin{itemize}
    \item \textbf{S (Set)} – Sets output \( Q = 1 \)
    \item \textbf{R (Reset)} – Resets output \( Q = 0 \)
\end{itemize}

\subsection{Truth Table}
\begin{center}
\begin{tabular}{ccccc}
\toprule
\textbf{S} & \textbf{R} & \textbf{Q} & \textbf{Q'} & \textbf{State} \\
\midrule
0 & 0 & Q & Q' & \textbf{Hold (Memory)} \\
1 & 0 & 1 & 0 & \textbf{Set (Q = 1)} \\
0 & 1 & 0 & 1 & \textbf{Reset (Q = 0)} \\
1 & 1 & 0 & 0 & \textbf{Invalid (Race)} \\
\bottomrule
\end{tabular}
\end{center}

\subsection{Invalid State}
When \( S = R = 1 \), both outputs \( Q \) and \( Q' \) become 0 (violates \( Q' = \overline{Q} \)).

\section{NAND-based SR Latch (Active-Low Inputs)}
\subsection{Structure}
Two NAND gates with feedback.

\subsection{Inputs}
\begin{itemize}
    \item \(\overline{S}\) (Set) – Active-low (0 sets \( Q = 1 \))
    \item \(\overline{R}\) (Reset) – Active-low (0 resets \( Q = 0 \))
\end{itemize}

\subsection{Truth Table}
\begin{center}
\begin{tabular}{ccccc}
\toprule
\(\overline{S}\) & \(\overline{R}\) & Q & Q' & \textbf{State} \\
\midrule
1 & 1 & Q & Q' & \textbf{Hold (Memory)} \\
0 & 1 & 1 & 0 & \textbf{Set (Q = 1)} \\
1 & 0 & 0 & 1 & \textbf{Reset (Q = 0)} \\
0 & 0 & 1 & 1 & \textbf{Invalid (Race)} \\
\bottomrule
\end{tabular}
\end{center}

\subsection{Invalid State}
When \( \overline{S} = \overline{R} = 0 \), both outputs \( Q \) and \( Q' \) become 1 (violates \( Q' = \overline{Q} \)).

\section{Characteristics}
\begin{itemize}
    \item \textbf{Level-Triggered}: Changes state based on input levels (not edges).
    \item \textbf{Asynchronous}: No clock signal required.
    \item \textbf{Race Condition}: Occurs when both inputs are active simultaneously.
\end{itemize}

\section{Applications}
\begin{itemize}
    \item Basic memory storage in registers.
    \item Debouncing switches.
    \item Temporary state storage in control circuits.
\end{itemize}

\section{Limitations}
\begin{itemize}
    \item \textbf{No Clock Control}: Cannot synchronize with a clock (unlike flip-flops).
    \item \textbf{Glitches}: Sensitive to input changes.
    \item \textbf{Invalid State}: Must avoid \( S = R = 1 \) (NOR) or \( \overline{S} = \overline{R} = 0 \) (NAND).
\end{itemize}

\section{Conclusion}
The \textbf{SR Latch} is the simplest sequential circuit used for \textbf{1-bit storage}. It has two stable states (\textbf{Set \& Reset}) but suffers from an \textbf{invalid state} when both inputs are active. It serves as the foundation for more complex storage elements like \textbf{Flip-Flops}.

\end{document}