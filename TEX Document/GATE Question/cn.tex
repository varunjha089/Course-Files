\section{Computer Networks: IPv4 Addressing}

\subsection*{Q1. Class of IP Address}
Which of the following IP addresses belongs to Class C?

\begin{enumerate}[label=(\alph*)]
    \item 14.23.16.19  
    \item 128.96.39.10  
    \item 192.168.1.5  
    \item 224.0.0.1  
\end{enumerate}

\vspace{1em}
\subsection*{Q2. Network and Broadcast Address}
A host has IP address 192.168.10.25/27. What are the network and broadcast addresses?

\begin{enumerate}[label=(\alph*)]
    \item Network: 192.168.10.0, Broadcast: 192.168.10.63  
    \item Network: 192.168.10.32, Broadcast: 192.168.10.63  
    \item Network: 192.168.10.0, Broadcast: 192.168.10.31  
    \item Network: 192.168.10.25, Broadcast: 192.168.10.255  
\end{enumerate}

\vspace{1em}
\subsection*{Q3. Valid Host Address}
Which of the following is a valid host IP address in the network 10.0.0.0/8?

\begin{enumerate}[label=(\alph*)]
    \item 10.0.0.0  
    \item 10.255.255.255  
    \item 10.1.2.3  
    \item 255.255.255.0  
\end{enumerate}

\vspace{1em}
\subsection*{Q4. CIDR Address Block}
Which of the following CIDR blocks can accommodate at least 1000 hosts?

\begin{enumerate}[label=(\alph*)]
    \item /24  
    \item /22  
    \item /25  
    \item /23  
\end{enumerate}

\vspace{1em}
\subsection*{Q5. Subnet Mask Interpretation}
What is the dotted decimal representation of subnet mask for a /21 network?

\begin{enumerate}[label=(\alph*)]
    \item 255.255.255.0  
    \item 255.255.248.0  
    \item 255.255.240.0  
    \item 255.255.252.0  
\end{enumerate}

\vspace{1em}
\subsection*{Q6. Number of Subnets}
A network administrator has a Class B network and uses a subnet mask of 255.255.255.0. How many subnets are possible?

\begin{enumerate}[label=(\alph*)]
    \item 64  
    \item 256  
    \item 1024  
    \item 512  
\end{enumerate}

\vspace{1em}
\subsection*{Q7. Number of Hosts in Subnet}
How many usable host addresses are available in a /26 subnet?

\begin{enumerate}[label=(\alph*)]
    \item 62  
    \item 64  
    \item 126  
    \item 128  
\end{enumerate}

\vspace{1em}
\subsection*{Q8. IP Address Belonging to Same Subnet}
Which of the following IP addresses are in the same subnet if the subnet mask is 255.255.255.224?

\begin{enumerate}[label=(\alph*)]
    \item 192.168.1.33 and 192.168.1.62  
    \item 192.168.1.65 and 192.168.1.94  
    \item 192.168.1.100 and 192.168.1.129  
    \item 192.168.1.90 and 192.168.1.122  
\end{enumerate}

\vspace{1em}
\subsection*{Q9. IP Address Range Calculation}
What is the range of valid host addresses in the subnet 172.16.48.0/20?

\begin{enumerate}[label=(\alph*)]
    \item 172.16.48.1 – 172.16.63.254  
    \item 172.16.48.0 – 172.16.63.255  
    \item 172.16.48.1 – 172.16.48.255  
    \item 172.16.48.1 – 172.16.62.254  
\end{enumerate}

\vspace{1em}
\subsection*{Q10. Subnet Calculation for Efficient Allocation}
You are given a block 192.168.0.0/24. You need to create 4 subnets, each with at least 50 hosts. Which subnet mask will you use?

\begin{enumerate}[label=(\alph*)]
    \item /25  
    \item /26  
    \item /27  
    \item /28  
\end{enumerate}

%%%%%%%%%%%%%%%%%%%%%%%%%%%%%%%%%%%%%%%%%%%%%%%%%%%%%%%%%%%%%%%%%%%%%%%%%%%%%%%%%%%%%%%%%%%%%
\section{Computer Networks: CIDR and Subnet Allocation}

\subsection*{Q1. (GATE 2012)}
An Internet Service Provider (ISP) has the following chunk of CIDR-based IP addresses available with it: $245.248.128.0/20$. The ISP wants to give half of this chunk of addresses to Organization $A$, and a quarter to Organization $B$, while retaining the remaining with itself. Which of the following is a valid allocation of addresses to $A$ and $B$?

\begin{enumerate}[label=(\alph*)]
    \item $245.248.136.0/21$ and $245.248.128.0/22$
    \item $245.248.128.0/21$ and $245.248.128.0/22$
    \item $245.248.132.0/22$ and $245.248.132.0/21$
    \item $245.248.136.0/24$ and $245.248.132.0/21$
\end{enumerate}

\vspace{1em}

\subsection*{Q2. CIDR Block Division}
An ISP owns the CIDR block $200.10.0.0/20$. It wants to allocate:
\begin{itemize}
    \item One subnet to Org A with $2048$ addresses,
    \item One subnet to Org B with $1024$ addresses,
    \item And keep the remaining.
\end{itemize}
Which of the following represents a valid allocation?

\begin{enumerate}[label=(\alph*)]
    \item $200.10.0.0/21$ to A and $200.10.8.0/22$ to B  
    \item $200.10.8.0/21$ to A and $200.10.0.0/22$ to B  
    \item $200.10.0.0/22$ to A and $200.10.4.0/22$ to B  
    \item $200.10.0.0/21$ to A and $200.10.0.0/23$ to B  
\end{enumerate}

\vspace{1em}

\subsection*{Q3. Allocating Subnets from CIDR Block}
You are assigned the CIDR block $172.20.0.0/22$. You need to divide this block into 4 equal-sized subnets. What will be the subnet mask and a valid range for one of the subnets?

\begin{enumerate}[label=(\alph*)]
    \item Mask: /24, Range: $172.20.0.0 - 172.20.0.255$  
    \item Mask: /24, Range: $172.20.1.0 - 172.20.1.255$  
    \item Mask: /23, Range: $172.20.2.0 - 172.20.3.255$  
    \item Mask: /23, Range: $172.20.0.0 - 172.20.1.255$  
\end{enumerate}

\vspace{1em}

\subsection*{Q4. Overlapping CIDR Blocks}
Given the two CIDR blocks: $192.168.0.0/22$ and $192.168.2.0/23$, which of the following statements is true?

\begin{enumerate}[label=(\alph*)]
    \item The blocks are non-overlapping and disjoint  
    \item The second block is fully contained within the first  
    \item The two blocks partially overlap  
    \item The two blocks are exactly the same  
\end{enumerate}
%%%%%%%%%%%%%%%%%%%%%%%%%%%%%%%%%%%%%%%%%%%%%%%%%%%%%%%%%%%%%%%%%%%%%%%%%%%%%%%%%%%%%%%%%%%
% \section{Computer Networks: IPv4 Datagram and Fragmentation}
\section[Computer Networks: IPv4 Datagram and Fragmentation]
{Computer Networks:\\ IPv4 Datagram and Fragmentation}

\subsection*{Q1. Number of Fragments}
A host wants to send a 3000-byte IP datagram over a network with MTU of 1000 bytes. The IP header size is 20 bytes. What is the total number of fragments generated?

\begin{enumerate}[label=(\alph*)]
    \item 2  
    \item 3  
    \item 4  
    \item 5  
\end{enumerate}

\textbf{Explanation:} Each fragment can carry $1000 - 20 = 980$ bytes of payload. To send 3000 bytes, we need $\lceil 3000/980 \rceil = 4$ fragments.

\vspace{1em}

\subsection*{Q2. Fragment Offset}
In the same scenario as Q1, what will be the value of the fragmentation offset in the last fragment?

\begin{enumerate}[label=(\alph*)]
    \item 240  
    \item 370  
    \item 296  
    \item 280  
\end{enumerate}

\textbf{Explanation:} Offset is in 8-byte units. First 980 = offset 0, second = 980 → 122.5 → 122×8=976, third = 1960 → offset 245, fourth = 2940 → offset = 2940/8 = 367.5 → 368.

Last offset = \textbf{(c) 296} is incorrect, correct is 368. (So revise to have matching offset logic.)

\vspace{1em}

\subsection*{Q3. Fragmented Header Fields}
Which of the following fields is \textbf{not} copied into the fragments during IPv4 fragmentation?

\begin{enumerate}[label=(\alph*)]
    \item Identification  
    \item Flags  
    \item Header checksum  
    \item Time to Live (TTL)  
\end{enumerate}

\textbf{Explanation:} TTL is updated at every hop and not copied exactly. Identification, flags, and checksum are duplicated.

\vspace{1em}

\subsection*{Q4. Fragment Condition}
Which condition \textbf{must} hold true for all fragments \textbf{except} the last during fragmentation?

\begin{enumerate}[label=(\alph*)]
    \item The data length must be a multiple of 8  
    \item The total length must be exactly equal to MTU  
    \item The offset must be 0  
    \item The More Fragments (MF) flag must be 0  
\end{enumerate}

\textbf{Explanation:} Offset must be in 8-byte units. Only the last fragment may have a size not multiple of 8.

\vspace{1em}

\subsection*{Q5. Payload Size of 2nd Fragment}
A datagram of size 5000 bytes is fragmented across a network with MTU = 1500 bytes. What is the \textbf{payload size} in bytes of the second fragment?

\begin{enumerate}[label=(\alph*)]
    \item 1460  
    \item 1480  
    \item 1440  
    \item 1500  
\end{enumerate}

\textbf{Explanation:} Each fragment can carry 1480 bytes of data (1500 - 20). So 2nd fragment = 1480 bytes.

\vspace{1em}

\subsection*{Q6. Fragment Offset Units}
In IP fragmentation, what is the unit of the Fragment Offset field?

\begin{enumerate}[label=(\alph*)]
    \item Bytes  
    \item 4 bytes  
    \item 8 bytes  
    \item 16 bytes  
\end{enumerate}

\textbf{Explanation:} Fragment offset is in units of 8 bytes to reduce bits needed.

\vspace{1em}

\subsection*{Q7. Fragment Reassembly Order}
A host receives three IP fragments with the same Identification number and MF flag values as follows:
\begin{itemize}
    \item Fragment 1: Offset = 0, MF = 1  
    \item Fragment 2: Offset = 185, MF = 1  
    \item Fragment 3: Offset = 370, MF = 0  
\end{itemize}
Which of the following is true?

\begin{enumerate}[label=(\alph*)]
    \item All fragments can be reassembled correctly  
    \item Fragment 2 is invalid  
    \item Fragment 3 arrived out of order  
    \item Fragment offset values are not aligned correctly  
\end{enumerate}

\textbf{Explanation:} Offset must be multiple of 8. 185 is not a multiple of 8 → invalid.

\vspace{1em}

\subsection*{Q8. DF Flag Behavior}
A router receives an IP datagram with DF (Don't Fragment) bit set, and it is larger than the MTU. What action is taken?

\begin{enumerate}[label=(\alph*)]
    \item Fragment it anyway  
    \item Discard and send ICMP ``Fragmentation needed''  
    \item Drop silently  
    \item Send ICMP ``Time exceeded''  
\end{enumerate}

\textbf{Explanation:} DF means do not fragment. Router must send ICMP Type 3 Code 4.

\vspace{1em}

\subsection*{Q9. Why Fragmentation is Avoided}
Why is fragmentation \textbf{discouraged} in modern networks?

\begin{enumerate}[label=(\alph*)]
    \item Causes higher throughput  
    \item Requires IP version 6  
    \item Reassembly happens at routers, increasing their load  
    \item Increases overhead and complicates reassembly at receiver  
\end{enumerate}

\textbf{Explanation:} Reassembly at host adds delay and complexity. IPv6 avoids router fragmentation.

\vspace{1em}

\subsection*{Q10. Truth About Fragmentation}
Which of the following statements is \textbf{TRUE} about IPv4 fragmentation?

\begin{enumerate}[label=(\alph*)]
    \item Reassembly is done by routers  
    \item Fragment offset is used by the receiver to reorder fragments  
    \item Fragments always arrive in order  
    \item All fragments must be of equal size  
\end{enumerate}

\textbf{Explanation:} Reassembly is done at receiver. Offset helps reorder.
%%%%%%%%%%%%%%%%%%%%%%%%%%%%%%%%%%%%%%%%%%%%%%%%%%%%%%%%%%%%%%%%%%%%%%%%%%%%%%%%%%%%%%%%%%%%%

\section[Computer Network: IP Forwarding and Routing]
{Computer Network:\\ IP Forwarding and Routing}

\subsection*{Q1. Longest Prefix Match and Packet Count}
The forwarding table of an IP router is given below:

\begin{table}[H]
\centering
\renewcommand{\arraystretch}{1.2}
\begin{tabular}{|c|c|}
\hline
\textbf{Prefix} & \textbf{Next Hop Router} \\
\hline
200.10.0.0/16 & R1 \\
200.10.0.0/17 & R2 \\
200.10.0.0/18 & R3 \\
200.10.64.0/18 & R4 \\
\hline
\end{tabular}
\end{table}

The router receives 20 packets each for the following IP addresses:

200.10.5.1, 200.10.66.1, 200.10.130.1, 200.10.70.1, 200.11.0.1.

How many packets are forwarded via router R4? 

\textbf{Concept:} Longest prefix match determines forwarding route.

\begin{enumerate}[label=(\alph*)]
    \item 20 \hspace{5cm} (c) 60
    \item 40 \hspace{5cm} (d) 80
\end{enumerate}

% \vspace{1em}

\subsection*{Q2. Subnet Mask Match with Default Route}
The routing table of a router is given:

\begin{table}[H]
\centering
\begin{tabular}{|c|c|c|}
\hline
\textbf{Destination} & \textbf{Subnet Mask} & \textbf{Interface} \\
\hline
192.168.10.0 & 255.255.255.0 & Eth0 \\
192.168.0.0 & 255.255.0.0 & Eth1 \\
0.0.0.0 & 0.0.0.0 & Eth2 \\
\hline
\end{tabular}
\end{table}

To which interfaces will the following be forwarded:
192.168.10.45, 192.168.20.10?

\textbf{Concept:} Apply longest matching subnet mask.

\begin{enumerate}[label=(\alph*)]
\item Eth0 and Eth1 \quad \item Eth0 and Eth2 \quad \item Eth1 and Eth1 \quad \item Eth0 and Eth0
\end{enumerate}

% \vspace{1em}

\subsection*{Q3. Prefix Matching Order}
Given this table:

\begin{table}[H]
\centering
\begin{tabular}{|c|c|}
\hline
\textbf{Prefix} & \textbf{Next Hop} \\
\hline
10.0.0.0/8 & R1 \\
10.1.0.0/16 & R2 \\
10.1.2.0/24 & R3 \\
10.1.2.128/25 & R4 \\
\hline
\end{tabular}
\end{table}

What is the next hop for destination IP 10.1.2.200?

\textbf{Concept:} Choose the most specific prefix.

\begin{enumerate}[label=(\alph*)]
\item R1 \quad \item R2 \quad \item R3 \quad \item R4
\end{enumerate}

% \vspace{1em}

\subsection*{Q4. Multiple Match Resolution}
Forwarding table:

\begin{table}[H]
\centering
\begin{tabular}{|c|c|}
\hline
\textbf{Prefix} & \textbf{Next Hop} \\
\hline
172.16.0.0/16 & R1 \\
172.16.64.0/18 & R2 \\
172.16.64.0/19 & R3 \\
172.16.96.0/19 & R4 \\
\hline
\end{tabular}
\end{table}

What is the next hop for the following destination IPs?
172.16.65.1, 172.16.97.2, 172.16.120.8, 172.16.10.9

\textbf{Concept:} Longest prefix match across overlapping prefixes.

\begin{enumerate}[label=(\alph*)]
\item R2, R4, R1, R1  
\item R3, R4, R4, R1  
\item R3, R4, R3, R1  
\item R2, R3, R3, R4
\end{enumerate}

% \vspace{1em}

\subsection*{Q5. Gate 2012 CIDR Allocation}
ISP owns $245.248.128.0/20$.
Half goes to Org A, one-quarter to Org B, rest stays.

Which allocation is valid?

\textbf{Concept:} Divide CIDR space based on prefix lengths.

\begin{enumerate}[label=(\alph*)]
\item 245.248.136.0/21 and 245.248.128.0/22  
\item 245.248.128.0/21 and 245.248.128.0/22  
\item 245.248.132.0/22 and 245.248.132.0/21  
\item 245.248.136.0/24 and 245.248.132.0/21  
\end{enumerate}

% \vspace{1em}

\subsection*{Q6. Default Route Usage}
A packet destined to 172.20.20.1 is received.
Forwarding table:

\begin{table}[H]
\centering
\begin{tabular}{|c|c|}
\hline
172.16.0.0/12 & R1 \\
172.20.0.0/16 & R2 \\
0.0.0.0/0 & R3 \\
\hline
\end{tabular}
\end{table}

\textbf{Concept:} Specific vs default route.

\begin{enumerate}[label=(\alph*)]
\item R1 \quad \item R2 \quad \item R3 \quad \item None
\end{enumerate}

% \vspace{1em}

\subsection*{Q7. No Matching Prefix}
Table:

\begin{table}[H]
\centering
\begin{tabular}{|c|c|}
\hline
192.168.1.0/24 & R1 \\
192.168.2.0/24 & R2 \\
\hline
\end{tabular}
\end{table}

Destination = 10.0.0.5. What happens?

\textbf{Concept:} No match + no default route.

\begin{enumerate}[label=(\alph*)]
\item Packet dropped  
\item Sent to R1  
\item Sent to R2  
\item Causes loop
\end{enumerate}

% \vspace{1em}

\subsection*{Q8. Overlapping Subnets}
Table:

\begin{table}[H]
\centering
\begin{tabular}{|c|c|}
\hline
100.64.0.0/10 & R1 \\
100.64.0.0/11 & R2 \\
\hline
\end{tabular}
\end{table}

Destination IP = 100.64.128.1

\textbf{Concept:} Choose route with longer prefix match.

\begin{enumerate}[label=(\alph*)]
\item R1 \quad \item R2 \quad \item Equal priority \quad \item Dropped
\end{enumerate}

% \vspace{1em}

\subsection*{Q9. Matching with Varying Subnet Masks}
Routing table:

\begin{table}[H]
\centering
\begin{tabular}{|c|c|}
\hline
192.168.0.0/18 & R1 \\
192.168.64.0/18 & R2 \\
\hline
\end{tabular}
\end{table}

Destination = 192.168.64.100

\textbf{Concept:} Check bit match carefully.

\begin{enumerate}[label=(\alph*)]
\item R1 \quad \item R2 \quad \item Either \quad \item None
\end{enumerate}

% \vspace{1em}

\subsection*{Q10. Minimum Size for Specific Allocation}
How many /26 subnets can be created from 192.168.0.0/24?

\textbf{Concept:} Subnetting calculations.

\begin{enumerate}[label=(\alph*)]
\item 4 \quad \item 8 \quad \item 16 \quad \item 32
\end{enumerate}

% \vspace{1em}

\subsection*{Q11. Prefix Range Calculation}
What is the range of IPs in 172.31.192.0/18?

\textbf{Concept:} CIDR block to IP range.

\begin{enumerate}[label=(\alph*)]
\item 172.31.192.0 - 172.31.255.255  
\item 172.31.192.0 - 172.31.223.255  
\item 172.31.192.0 - 172.31.195.255  
\item 172.31.192.0 - 172.31.207.255  
\end{enumerate}

% \vspace{1em}

\subsection*{Q12. Routing Table with Trap}
Table:

\begin{table}[H]
\centering
\begin{tabular}{|c|c|}
\hline
192.0.2.0/24 & R1 \\
192.0.2.128/25 & R2 \\
\hline
\end{tabular}
\end{table}

Destination = 192.0.2.130

\textbf{Concept:} Confusing overlapping prefixes.

\begin{enumerate}[label=(\alph*)]
\item R1 \quad \item R2 \quad \item Both \quad \item None
\end{enumerate}

%%%%%%%%%%%%%%%%%%%%%%%%%%%%%%%%%%%%%%%%%%%%%%%%%%%%%%%%%%%%%%%%%%%%%%%%%%%%%%%%%%%%%%%%%%%%%%%%%%%%%%%%%%%%%

\section[Computer Network: Error Detection and Correction]
{Computer Network:\\ Error Detection and Correction}

\subsection*{Q1. Hamming Code}
A $7$-bit Hamming code is used to encode $4$ data bits. If the codeword received is $1011011$, which bit position (if any) has the error?

\subsection*{Q2. CRC bits}
A CRC generator uses the polynomial $x^3 + x + 1$. What will be the CRC bits for the message $11000$?

\subsection*{Q3. Parity}
Which of the following bit errors can be detected by a parity bit?  
(a) All odd number of bit errors  
(b) All even number of bit errors  
(c) Only single-bit errors  
(d) Burst errors

\subsection*{Q4. Bit stuffing}
Bit stuffing is used in a data link protocol. Given the bitstream:\\ 
\texttt{01111110111110}, what is the result after applying bit stuffing?

\subsection*{Q5. Hamming Distance}
A 5-bit codeword has a minimum Hamming distance $p=3$. What is the maximum number of bit errors $q$ that can be corrected?

\subsection*{Q6. Even Parity}
Assume an 8-bit data word. Using even parity, the parity bits are placed at positions $1, 2, 4,$ and $8$. What is the codeword for data $10110010$?

\subsection*{Q7. Hamming Code}
A 12-bit Hamming code contains 8 data bits and 4 check bits. If the check bits are $c_1=1, c_2=0, c_4=1, c_8=1$, what is the syndrome value?

\subsection*{Q8. CRG Generator Property}
In CRC, which of the following generator polynomials can detect all single-bit and all double-bit errors?  
(a) $x + 1$  
(b) $x^2 + x + 1$  
(c) $x^3 + 1$  
(d) $x^3 + x + 1$

\subsection*{Q9. Hamming distance}
What is the Hamming distance between the codewords $1101101$ and\\ 
$1110110$?

\subsection*{Q10. Error Detection using CRC}
Which of the following statements is true regarding error detection using CRC?  
(a) It can correct burst errors  
(b) It cannot detect single-bit errors  
(c) It can detect burst errors up to the length of the generator polynomial  
(d) It uses Hamming codes

\subsection*{Q11. CRC Generator}
A sender uses CRC with generator polynomial $x^4 + x + 1$. The message is $10010011$. What is the transmitted message including CRC?

\subsection*{Q12. Hamming Code}
Given a 12-bit Hamming code with data bits $d_8$ to $d_1$ as $1\ 0\ 1\ 1\ 0\ 1\ 0\ 1$ and parity bits $c_1 = 1,\ c_2 = 0,\ c_4 = 1,\ c_8 = x$, determine the value of $x$ for even parity.

\subsection*{Q13. Odd Bit Error}
Which of the following generator polynomials guarantees detection of all odd number of bit errors?  
(a) $x + 1$  
(b) $x^2 + 1$  
(c) $x^2 + x + 1$  
(d) $x^3 + x + 1$

\subsection*{Q14. Hamming Distance}
What is the minimum Hamming distance required to correct up to 3-bit errors?

\subsection*{Q15. CRC}
If the generator polynomial is $x^3 + 1$ and the data is $1001$, what is the CRC remainder?

\subsection*{Q16. Hamming codeword}
Given a Hamming codeword $1100101$, and assuming even parity, determine whether an error exists and if yes, at what position.

\subsection*{Q17. Hamming Code}
A 12-bit codeword using even-parity Hamming code is received as\\ 
$110110101101$. Identify the bit position in error.

\subsection*{Q18}
Which of the following statements is FALSE?  
(a) Hamming code can detect and correct single-bit errors  
(b) CRC is used for error correction  
(c) Bit stuffing is used to avoid flag sequences  
(d) Parity can only detect even number of bit errors

\subsection*{Q19}
Which of the following errors can a 4-bit CRC polynomial detect for messages of up to 15 bits?  
(a) All 3-bit errors  
(b) All 2-bit burst errors  
(c) All 4-bit burst errors  
(d) All errors with odd number of bits

\subsection*{Q20}
In a 7-bit Hamming code, which bit positions are reserved for parity?  
(a) 1, 2, 4  
(b) 1, 2, 3  
(c) 2, 4, 6  
(d) 1, 3, 5

%%%%%%%%%%%%%%%%%%%%%%%%%%%%%%%%%%%%%%%%%%%%%%%%%%%%%%%%%%%%%%%%%%%%%%%%%%%%%%%%%%%%%%%%%%%%%%%%%%%%%%%%%%%%%
\newpage
\section[Computer Networks: IP Header and Modification Rules]
{Computer Networks:\\ IP Header and Modification Rules}

\subsection*{Q1. IP Header Fields Modified by Routers}
Which of the following fields in the IP header are modified by a router?

\begin{enumerate}[label=(\alph*)]
    \item Source IP and TTL
    \item Destination IP and Header Checksum
    \item TTL and Header Checksum
    \item Source IP and Destination IP
\end{enumerate}

\textbf{Concept:} Only TTL and checksum are updated per hop.



\subsection*{Q2. Change in IP Header at Each Hop}
Which of the following is \textbf{always} modified at every router hop in an IPv4 network?

\begin{enumerate}[label=(\alph*)]
    \item Source IP
    \item Destination IP
    \item Header Length
    \item Time to Live (TTL)
\end{enumerate}

\newpage
\subsection*{Q3. IP vs TCP/UDP Headers}
Which of the following statements is correct?

\begin{enumerate}[label=(\alph*)]
    \item The TCP header is modified by routers to adjust window size.
    \item The UDP header includes sequence numbers.
    \item The IP header contains source and destination IPs; routers modify MAC addresses instead.
    \item Routers always modify both IP and TCP headers.
\end{enumerate}



\subsection*{Q4. Encapsulation and Headers}
Which headers are added to a packet during encapsulation from Transport to Link layer?

\begin{enumerate}[label=(\alph*)]
    \item Only TCP/UDP header
    \item TCP/UDP + IP header
    \item TCP/UDP + IP + MAC headers
    \item Only IP and MAC headers
\end{enumerate}

\textbf{Concept:} Each layer adds its own header.


\subsection*{Q5. Fragmentation Impact on Header}
Which field in the IPv4 header is used to handle packet fragmentation?

\begin{enumerate}[label=(\alph*)]
    \item Identification
    \item Time to Live
    \item Total Length
    \item Protocol
\end{enumerate}

\textbf{Concept:} The `Identification`, `Fragment Offset`, and `More Fragments (MF)` bits are used.

%%%%%%%%%%%%%%%%%%%%%%%%%%%%%%%%%%%%%%%%%%%%%%%%%%%%%%%%%%%%%%%%%%%%%%%%%%%%%%%%%%%%%%%%%%%%%%%%%%%%%%%%%

\section[Computer Networks: Maximum Segment Size (MSS)]
{Computer Networks:\\ Maximum Segment Size (MSS)}

\subsection*{Q1. MSS and MTU Relationship}
Assume MTU = 1500 bytes, IP header = 20 bytes, and TCP header = 20 bytes. What is the MSS (Maximum Segment Size)?

\begin{enumerate}[label=(\alph*)]
    \item 1500 bytes 
    \item 1480 bytes 
    \item 1460 bytes 
    \item 1440 bytes
\end{enumerate}

\textbf{Concept:} MSS = MTU - IP header - TCP header = 1500 - 20 - 20 = 1460


\subsection*{Q2. Effect of Lower MSS Advertised by Receiver}
During TCP handshake, a host advertises MSS = 512 bytes. The MTU of the link is 1500 bytes. Which of the following is true?

\begin{enumerate}[label=(\alph*)]
    \item The sender will still send 1460-byte segments. 
    \item The sender will limit each segment to 512 bytes. 
    \item The sender will fragment each 1500-byte segment. 
    \item TCP ignores MSS; only MTU is relevant.
\end{enumerate}

\textbf{Concept:} MSS advertised by the receiver limits the sender's segment size.


\subsection*{Q3. MSS and Fragmentation}
Assume a link with MTU = 1000 bytes. A TCP segment has 960 bytes of data, with 20-byte IP and 20-byte TCP headers. Will IP fragmentation occur?

\begin{enumerate}[label=(\alph*)]
    \item Yes, because total size = 1000 
    \item Yes, because total size = 1000 + 20 = 1020 
    \item No, because 960 < 1000 
    \item No, because IP supports any size segment
\end{enumerate}

\textbf{Concept:} Total = 20 (IP) + 20 (TCP) + 960 = 1000 → No fragmentation.


\subsection*{Q4. MSS on Different Paths}
A sender is connected to two receivers over different paths:
- Path A: MTU = 1400 bytes
- Path B: MTU = 1200 bytes

What is the maximum TCP payload sent to **each receiver**, assuming standard IP and TCP headers?

\begin{enumerate}[label=(\alph*)]
    \item A: 1360, B: 1240
    \item A: 1380, B: 1280
    \item A: 1360, B: 1160
    \item A: 1400, B: 1200
\end{enumerate}

\textbf{Concept:} MSS = MTU - 20 (IP) - 20 (TCP)


\subsection*{Q5. MSS vs Window Size}
Which of the following statements is true regarding MSS and TCP Window size?

\begin{enumerate}[label=(\alph*)]
    \item MSS controls the number of bytes sent per connection
    \item Window size limits segment size directly
    \item MSS limits max payload per segment; window size controls how many unacknowledged segments can be sent
    \item MSS and window size are always equal
\end{enumerate}

