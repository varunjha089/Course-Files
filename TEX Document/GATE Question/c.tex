\section{C Programming: Tricky Questions}

\subsection*{Q1. Pointer Overlap Copy}
\noindent
Consider the C code:
\begin{lstlisting}[language=C]
#include <stdio.h>
void mycopy(char *dest, char *src) {
    while (*src)
        *dest++ = *src++;
}
int main() {
    char str[30] = "ABCDE";
    mycopy(str + 1, str);
    printf("%s\n", str);
    return 0;
}
\end{lstlisting}
\textbf{What is the output?}
\begin{enumerate}[label=(\alph*)]
    \item AABCDE
    \item ABCDE
    \item Infinite loop
    \item BCDE
\end{enumerate}

\vspace{1em}
\subsection*{Q2. Post-Increment and Array}
\noindent
\begin{lstlisting}[language=C]
#include <stdio.h>
int main() {
    int a[] = {10, 20, 30, 40, 50};
    int i = 0;
    printf("%d %d %d\n", a[i], i++, a[i]);
    return 0;
}
\end{lstlisting}
\textbf{What is the output?}
\begin{enumerate}[label=(\alph*)]
    \item 10 0 10
    \item 10 0 20
    \item 10 1 20
    \item Undefined behavior
\end{enumerate}

\vspace{1em}
\subsection*{Q3. Pointer to String Literal}
\noindent
\begin{lstlisting}[language=C]
#include <stdio.h>
int main() {
    char *p = "GATE2025";
    printf("%c\n", *(p + 4));
    return 0;
}
\end{lstlisting}
\textbf{What is the output?}
\begin{enumerate}[label=(\alph*)]
    \item 2
    \item T
    \item E
    \item 0
\end{enumerate}

\vspace{1em}
\subsection*{Q4. Evaluation Order in Function Args}
\noindent
\begin{lstlisting}[language=C]
#include <stdio.h>
void func(int x, int y) {
    printf("%d %d\n", x, y);
}
int main() {
    int i = 5;
    func(i++, ++i);
    return 0;
}
\end{lstlisting}
\textbf{What is the output?}
\begin{enumerate}[label=(\alph*)]
    \item 5 7
    \item 6 6
    \item Compiler error
    \item Undefined behavior
\end{enumerate}

\vspace{1em}
\subsection*{Q5. Array Indexing Trick}
\noindent
\begin{lstlisting}[language=C]
#include <stdio.h>
int main() {
    char str[] = "abcde";
    printf("%c\n", 2[str]);
    return 0;
}
\end{lstlisting}
\textbf{What is the output?}
\begin{enumerate}[label=(\alph*)]
    \item b
    \item c
    \item Compilation error
    \item e
\end{enumerate}
%%%%%%%%%%%%%%%%%%%%%%%%%%%%%%%%%%%%%%%%%%%%%%%%%%%%%%%%%%%%%%%%%%%%
\section{C Programming: Stack \& Queue Tricky Questions}

\subsection*{Q1. Circular Queue Edge Case}
Which condition correctly checks if a circular queue is full?

\begin{enumerate}[label=(\alph*)]
    \item \texttt{rear == SIZE - 1}
    \item \texttt{front == -1 \&\& rear == -1}
    \item \texttt{rear + 1 == front}
    \item \texttt{(rear + 1) \% SIZE == front}
\end{enumerate}

\subsection*{Q2. Stack Memory Growth}
Consider the following C code:

\begin{lstlisting}
#include <stdio.h>
void foo(int n) {
    if (n == 0) return;
    int x = n;
    foo(n - 1);
}
int main() {
    foo(10000);
    return 0;
}
\end{lstlisting}

What is most likely to happen when this program is run?

\begin{enumerate}[label=(\alph*)]
    \item It prints numbers from 10000 to 1.
    \item It runs successfully and terminates normally.
    \item It causes a stack overflow error.
    \item It goes into an infinite loop.
\end{enumerate}

\newpage
\subsection*{Q3. Queue Implementation Bug}
What will be the output of the following program?

\begin{lstlisting}
#include <stdio.h>
#define SIZE 5
int queue[SIZE], front = -1, rear = -1;

void enqueue(int x) {
    if (rear == SIZE - 1) return;
    queue[++rear] = x;
    if (front == -1) front = 0;
}

int dequeue() {
    if (front == -1 || front > rear) return -1;
    return queue[front++];
}

int main() {
    enqueue(10); enqueue(20); dequeue();
    dequeue(); dequeue();
    printf("%d\n", front);
    return 0;
}
\end{lstlisting}

\begin{enumerate}[label=(\alph*)]
    \item -1
    \item 1
    \item 0
    \item 3
\end{enumerate}

\subsection*{Q4. Stack Address Behavior}
Which of the following is true about the stack in C?

\begin{enumerate}[label=(\alph*)]
    \item Stack grows upwards in memory
    \item Stack grows downwards in memory
    \item Stack is allocated from heap
    \item Stack size is infinite
\end{enumerate}


\subsection*{Q5. Stack Push/Pop Behavior}
What will be the final output?

\begin{lstlisting}
#include <stdio.h>
#define SIZE 3
int stack[SIZE], top = -1;

void push(int val) {
    if (top < SIZE - 1)
        stack[++top] = val;
}
int pop() {
    if (top >= 0)
        return stack[top--];
    return -1;
}
int main() {
    push(1); push(2); push(3); push(4);
    printf("%d ", pop());
    printf("%d ", pop());
    return 0;
}
\end{lstlisting}

\begin{enumerate}[label=(\alph*)]
    \item 3 2
    \item 4 3
    \item 2 1
    \item -1 -1
\end{enumerate}

%%%%%%%%%%%%%%%%%%%%%%%%%%%%%%%%%%%%%%%%%%%%%%%%%%%%%%%%%%%%%%%%%%%%%%%%%%%%%%%%%%%%%%%%%%%

\newpage
\section{C Programming: Recursion Questions}

\subsection*{Q1. Simple Recursion and Return}
What is the output of the following code?

\begin{lstlisting}
#include <stdio.h>
int fun(int n) {
    if (n == 0) return 0;
    return n + fun(n - 1);
}
int main() {
    printf("%d", fun(3));
    return 0;
}
\end{lstlisting}

\begin{enumerate}[label=(\alph*)]
    \item 3
    \item 6
    \item 0
    \item 1
\end{enumerate}

\vspace{1em}
\subsection*{Q2. Post-Order Recursion Print}
What will be printed?

\begin{lstlisting}
#include <stdio.h>
void print(int n) {
    if (n == 0) return;
    print(n - 1);
    printf("%d ", n);
}
int main() {
    print(3);
    return 0;
}
\end{lstlisting}

\begin{enumerate}[label=(\alph*)]
    \item 3 2 1
    \item 1 2 3
    \item 0 1 2
    \item 3 2 1 0
\end{enumerate}

\vspace{1em}
\subsection*{Q3. Stack Frame Behavior}
Which of the following is true about recursive function calls in C?

\begin{enumerate}[label=(\alph*)]
    \item All calls share a single stack frame
    \item Each call gets its own stack frame
    \item Only leaf calls get stack frames
    \item Recursion doesn't use the stack
\end{enumerate}

\vspace{1em}
\subsection*{Q4. Recursion with Local Static}
What is the output of the program?

\begin{lstlisting}
#include <stdio.h>
int fun(int n) {
    static int x = 0;
    if (n > 0) {
        x++;
        fun(n - 1);
    }
    return x;
}
int main() {
    printf("%d", fun(5));
    return 0;
}
\end{lstlisting}

\begin{enumerate}[label=(\alph*)]
    \item 5
    \item 1
    \item 0
    \item 25
\end{enumerate}

\vspace{1em}
\subsection*{Q5. Mutual Recursion}
Consider the following code:

\begin{lstlisting}
#include <stdio.h>
void odd(int);
void even(int);

void odd(int x) {
    if (x == 0) return;
    printf("Odd: %d\n", x);
    even(x - 1);
}

void even(int x) {
    if (x == 0) return;
    printf("Even: %d\n", x);
    odd(x - 1);
}

int main() {
    odd(3);
    return 0;
}
\end{lstlisting}

What is the last line printed?

\begin{enumerate}[label=(\alph*)]
    \item Odd: 3
    \item Even: 2
    \item Odd: 1
    \item Even: 1
\end{enumerate}
%%%%%%%%%%%%%%%%%%%%%%%%%%%%%%%%%%%%%%%%%%%%%%%%%%%%%%%%%%%%%%%%%%%%%%%%%%%%%%%%%%%%%%%%%%
\newpage
\section{C Expression Evaluation with Custom Operator Precedence}

\subsection*{Q1. Custom Operator Precedence}

Consider the following custom operator precedence and associativity table:

\begin{table}[H]
\centering
\begin{tabular}{|c|c|c|}
\hline
\textbf{Operator} & \textbf{Precedence (High to Low)} & \textbf{Associativity} \\
\hline
\texttt{+} & 1 (Highest) & Left-to-Right \\
\texttt{/} & 2 & Left-to-Right \\
\texttt{-} & 3 & Right-to-Left \\
\texttt{*} & 4 (Lowest) & Left-to-Right \\
\hline
\end{tabular}
\end{table}

Evaluate the expression below according to the table above (not standard C rules):

\[
\texttt{6 + 3 * 2 - 4 / 2 + 1 * 2 - 3}
\]

What is the final result?

\begin{enumerate}[label=(\alph*)]
    \item 9 \hspace{5cm} (c) 7
    \item 8 \hspace{5cm} (d) 5
\end{enumerate}

\vspace{1em}
\subsection*{Q2. Operator Table and Mixed Associativity}

You are given the following precedence table:

\begin{table}[H]
\centering
\begin{tabular}{|c|c|c|}
\hline
\textbf{Operator} & \textbf{Precedence} & \textbf{Associativity} \\
\hline
\texttt{-} & 1 (Highest) & Right-to-Left \\
\texttt{*} & 2 & Left-to-Right \\
\texttt{+} & 3 & Left-to-Right \\
\texttt{/} & 4 (Lowest) & Left-to-Right \\
\hline
\end{tabular}
\end{table}

Evaluate the expression:

\[
\texttt{10 - 2 * 3 + 4 / 2 - 1}
\]

Assume only integer arithmetic and no parenthesis.

\begin{enumerate}[label=(\alph*)]
    \item 9 \hspace{5cm} (c) 7
    \item 8 \hspace{5cm} (d) 6
\end{enumerate}

%%%%%%%%%%%%%%%%%%%%%%%%%%%%%%%%%%%%%%%%%%%%%%%%%%%%%%%%%%%%%%%%%%%%%%%%%%%%%%%%%%%%%%%%%%%%%%%%%%%%%%%%%

\section[C Programming: Pointer and String Manipulation (Hard MCQs)]
{C Programming: Pointer and\\ String Manipulation (Hard MCQs)}

\subsection*{Q1}
What will be the output of the following C code?
\begin{lstlisting}[language=C]
#include <stdio.h>
int main() {
    char *s = "GATE";
    s++;
    printf("%c", *s++);
    printf("%c", *s);
    return 0;
}
\end{lstlisting}

\begin{enumerate}[label=(\alph*)]
    \item GA  
    \item AT  
    \item AE  
    \item AT
\end{enumerate}

\subsection*{Q2}
What is the output of the following code?
\begin{lstlisting}[language=C]
#include <stdio.h>
void fun(char *s) {
    while(*s) {
        printf("%c", *s++);
        s++;
    }
}
int main() {
    char str[] = "GATECSE";
    fun(str);
    return 0;
}
\end{lstlisting}

\begin{enumerate}[label=(\alph*)]
    \item GTEC  
    \item GAE  
    \item GAESE  
    \item GTC
\end{enumerate}

\subsection*{Q3}
What will be the output of the following program?
\begin{lstlisting}[language=C]
#include <stdio.h>
int main() {
    char str[] = "abc";
    char *p = str;
    printf("%c", ++*p);
    printf("%c", *p++);
    printf("%c", *p);
    return 0;
}
\end{lstlisting}

\begin{enumerate}[label=(\alph*)]
    \item abc  
    \item bbc  
    \item bac  
    \item bcc
\end{enumerate}

\subsection*{Q4}
Predict the output:
\begin{lstlisting}[language=C]
#include <stdio.h>
void change(char *p) {
    p++;
    *p = 'Z';
}
int main() {
    char str[] = "GATE";
    change(str);
    printf("%s", str);
    return 0;
}
\end{lstlisting}

\begin{enumerate}[label=(\alph*)]
    \item GATE  
    \item GZTE  
    \item GZTE  
    \item GAZE
\end{enumerate}

\subsection*{Q5}
What will the following code print?
\begin{lstlisting}[language=C]
#include <stdio.h>
int main() {
    char *s = "ABCDE";
    printf("%c", *s++);
    printf("%c", *(s+1));
    return 0;
}
\end{lstlisting}

\begin{enumerate}[label=(\alph*)]
    \item AB  
    \item AC  
    \item AD  
    \item AE
\end{enumerate}

\subsection*{Q6}
Output of the following program?
\begin{lstlisting}[language=C]
#include <stdio.h>
int main() {
    char *s = "GATECSE";
    printf("%c", *(s + *(s+3) - 'A'));
    return 0;
}
\end{lstlisting}

\begin{enumerate}[label=(\alph*)]
    \item G  
    \item A  
    \item E  
    \item C
\end{enumerate}

\newpage
\subsection*{Q7}
What will be the output?
\begin{lstlisting}[language=C]
#include <stdio.h>
int main() {
    char a[] = "Hello";
    char *p = a;
    printf("%s", p + 3);
    return 0;
}
\end{lstlisting}

\begin{enumerate}[label=(\alph*)]
    \item Hello  
    \item llo  
    \item lo  
    \item o
\end{enumerate}

\subsection*{Q8}
What is the output of this code?
\begin{lstlisting}[language=C]
#include <stdio.h>
int main() {
    char *p = "GATE";
    printf("%c", *p);
    p += 4;
    printf("%c", *(p-1));
    return 0;
}
\end{lstlisting}

\begin{enumerate}[label=(\alph*)]
    \item GG  
    \item GE  
    \item EE  
    \item EG
\end{enumerate}

\newpage
\subsection*{Q9}
What is the output of the program?
\begin{lstlisting}[language=C]
#include <stdio.h>
void rev(char *s) {
    if(*s) {
        rev(s+1);
        putchar(*s);
    }
}
int main() {
    rev("CSE");
    return 0;
}
\end{lstlisting}

\begin{enumerate}[label=(\alph*)]
    \item CSE  
    \item ESC  
    \item SEC  
    \item ESC
\end{enumerate}

\subsection*{Q10}
What is the output?
\begin{lstlisting}[language=C]
#include <stdio.h>
int main() {
    char s1[] = "abcde";
    char *s2 = "abcde";
    s1[0] = 'z';
    // s2[0] = 'z'; // Uncommenting causes?
    printf("%s %s", s1, s2);
    return 0;
}
\end{lstlisting}

\begin{enumerate}[label=(\alph*)]
    \item zbcde zbcde  
    \item Compilation error  
    \item zbcde abcde  
    \item abcde abcde
\end{enumerate}

%%%%%%%%%%%%%%%%%%%%%%%%%%%%%%%%%%%%%%%%%%%%%%%%%%%%%%%%%%%%%%%%%%%%%%%%%%%%%%%%%%%%%%%%%%%%%%%%%%
\newpage
\section[Tricky C Programming MCQs: Edge Cases and Pitfalls]
{Tricky C Programming MCQs:\\ Edge Cases and Pitfalls}

\subsection*{Q1. Assignment inside condition}
\begin{verbatim}
#include <stdio.h>
int main() {
    int a = 10;
    if (a = 5)
        printf("True");
    else
        printf("False");
    return 0;
}
\end{verbatim}
\textbf{What is the output of the program?}

\begin{enumerate}[label=(\alph*)]
    \item True
    \item False
    \item Compilation Error
    \item Runtime Error
\end{enumerate}

\subsection*{Q2. Modifying string literal (UB)}
\begin{verbatim}
#include <stdio.h>
int main() {
    char s[] = "hello";
    s[5] = '!';
    printf("%s", s);
    return 0;
}
\end{verbatim}
\textbf{What is the output of the program?}

\begin{enumerate}[label=(\alph*)]
    \item hello!
    \item hello
    \item Undefined Behavior
    \item Compilation Error
\end{enumerate}

\subsection*{Q3. Semicolon after for loop}
\begin{verbatim}
#include <stdio.h>
int main() {
    int i = 0;
    for(; i < 3; i++);
        printf("%d ", i);
    return 0;
}
\end{verbatim}
\textbf{What is the output of the program?}

\begin{enumerate}[label=(\alph*)]
    \item 0 1 2 
    \item 3
    \item 0 1 2 3
    \item Nothing
\end{enumerate}

\subsection*{Q4. undefined behavior on write}
\begin{verbatim}
#include <stdio.h>
void foo(char *str) {
    str[0] = 'A';
}
int main() {
    char *s = "hello";
    foo(s);
    printf("%s", s);
    return 0;
}
\end{verbatim}
\textbf{What is the output of the program?}

\begin{enumerate}[label=(\alph*)]
    \item Aello
    \item hello
    \item Compilation Error
    \item Segmentation Fault / Undefined Behavior
\end{enumerate}

\subsection*{Q5. $a[i] == i[a]$ trick}
\begin{verbatim}
#include <stdio.h>
int main() {
    int arr[3] = {1, 2, 3};
    printf("%d", 1[arr]);
    return 0;
}
\end{verbatim}
\textbf{What is the output of the program?}

\begin{enumerate}[label=(\alph*)]
    \item 1
    \item 2
    \item Compilation Error
    \item Garbage Value
\end{enumerate}


