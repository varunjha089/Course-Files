\section{Operating Systems: Synchronization}

\subsection*{Q1. [MCQ]}
Which of the following is not a necessary condition for a deadlock to occur?

\begin{enumerate}[label=(\alph*)]
    \item Mutual Exclusion
    \item Hold and Wait
    \item No Preemption
    \item Context Switching
\end{enumerate}

\subsection*{Q2. [MCQ]}
Consider two processes accessing a shared variable using busy-wait locks. Which scenario can lead to starvation?

\begin{enumerate}[label=(\alph*)]
    \item Mutual exclusion is violated
    \item No process yields the CPU
    \item Both use priority inheritance
    \item Critical section is empty
\end{enumerate}

\subsection*{Q3. [MSQ]}
Which of the following techniques can help avoid deadlock?

\begin{enumerate}[label=(\alph*)]
    \item Resource allocation graph with a single instance per resource
    \item Banker's algorithm
    \item Wait-for graph cycle detection
    \item Allowing all requests at once
\end{enumerate}


\subsection*{Q4. [MCQ]}
The Peterson's algorithm ensures:

\begin{enumerate}[label=(\alph*)]
    \item Deadlock freedom
    \item Starvation freedom
    \item Mutual exclusion
    \item All of the above
\end{enumerate}

\subsection*{Q5. [MCQ]}
Which of the following is true for the Bakery Algorithm?

\begin{enumerate}[label=(\alph*)]
    \item It requires hardware support
    \item It prevents mutual exclusion
    \item It works for more than two processes
    \item It allows priority scheduling
\end{enumerate}

\subsection*{Q6. [MCQ]}
In a system using semaphores, a process executes \textbf{wait(S)} and is put to sleep. When \textbf{signal(S)} is called by another process:

\begin{enumerate}[label=(\alph*)]
    \item S is incremented, but the sleeping process continues to sleep
    \item Sleeping process is immediately awakened
    \item \textbf{signal()} fails silently
    \item S is reset to 0
\end{enumerate}

\subsection*{Q7. [MCQ]}
In which scenario does the use of spinlocks become inefficient?

\begin{enumerate}[label=(\alph*)]
    \item Multiprocessor systems
    \item Critical section is very short
    \item Process holds CPU for long durations in CS
    \item Atomic instructions are expensive
\end{enumerate}

\subsection*{Q8. [MSQ]}
Consider a system with preemptive scheduling and semaphores. Which conditions must be met for bounded waiting?

\begin{enumerate}[label=(\alph*)]
    \item FIFO ordering of blocked processes
    \item Starvation prevention
    \item Mutual exclusion
    \item Strict priority scheduling
\end{enumerate}

\subsection*{Q9. [MCQ]}
Which of the following mechanisms is non-blocking and busy-waiting?
 
\begin{enumerate}[label=(\alph*)]
    \item Counting semaphore
    \item Binary semaphore
    \item Spinlock
    \item Mutex
\end{enumerate}

\subsection*{Q10. [MCQ]}
Which of the following issues is most likely to occur if semaphores are misused?

\begin{enumerate}[label=(\alph*)]
    \item Deadlock
    \item Thrashing
    \item Paging
    \item Fragmentation
\end{enumerate}

\newpage
\subsection*{Q11. [MCQ]}
Which of the following statements is true regarding priority inversion?

\begin{enumerate}[label=(\alph*)]
    \item It is solved using Banker's algorithm
    \item It occurs when a low-priority process holds a lock needed by a high-priority one
    \item It is always prevented in spinlocks
    \item Round-robin scheduling prevents it
\end{enumerate}

\subsection*{Q12. [MCQ]}
The use of monitors provides:

\begin{enumerate}[label=(\alph*)]
    \item Mutual exclusion only
    \item Synchronization without requiring condition variables
    \item Encapsulation of shared data and synchronization
    \item Only inter-process communication
\end{enumerate}

\subsection*{Q13. [MCQ]}
In a classic producer-consumer problem, using bounded buffer and\\
semaphores, which semaphore combination is incorrect?

\begin{enumerate}[label=(\alph*)]
    \item Empty initialized to N
    \item Full initialized to 0
    \item Mutex initialized to 0
    \item Mutex initialized to 1
\end{enumerate}

\subsection*{Q14. [MSQ]}
A correct solution to the Dining Philosophers problem must ensure:
  
\begin{enumerate}[label=(\alph*)]
    \item No deadlock
    \item No starvation
    \item Maximum concurrency
    \item No preemption
\end{enumerate}

\subsection*{Q15. [MCQ]}
A process enters its critical section. Suddenly, it is preempted. What is the main concern in such a system?

\begin{enumerate}[label=(\alph*)]
    \item Starvation
    \item Race condition
    \item Deadlock
    \item Context switching overhead
\end{enumerate}

\subsection*{Q16. [MCQ]}
Busy waiting in critical section management is:

\begin{enumerate}[label=(\alph*)]
    \item Always preferable on single-core systems
    \item Efficient when CS time is small and context-switching is expensive
    \item Prevented by semaphores
    \item Never used in modern OS
\end{enumerate}

\subsection*{Q17. [MCQ]}
Which of the following statements is false?

\begin{enumerate}[label=(\alph*)]
    \item Semaphore \textbf{wait()} operation is also called \textbf{P()}
    \item A monitor can have condition variables
    \item Spinlock wastes CPU cycles
    \item Peterson's algorithm works with preemption
\end{enumerate}

\subsection*{Q18. [MSQ]}
Which of the following mechanisms support process synchronization?
 
\begin{enumerate}[label=(\alph*)]
    \item Monitors
    \item Semaphores
    \item Message passing
    \item File I/O operations
\end{enumerate}

\subsection*{Q19. [MCQ]}
A deadlock can occur if:

\begin{enumerate}[label=(\alph*)]
    \item All four Coffman conditions hold
    \item At least three Coffman conditions hold
    \item Only circular wait holds
    \item Mutual exclusion is violated
\end{enumerate}

\subsection*{Q20. [MCQ]}
Two threads simultaneously execute \textbf{wait(mutex)} and get stuck. What likely happened?

\begin{enumerate}[label=(\alph*)]
    \item Mutual exclusion was violated
    \item \textbf{signal(mutex)} was never called
    \item Deadlock occurred due to wrong semaphore initialization
    \item Mutex had a race condition
\end{enumerate}

%%%%%%%%%%%%%%%%%%%%%%%%%%%%%%%%%%%%%%%%%%%%%%%%%%%%%%%%%%%%%%%%%%%%%%%%%%%%%%%%%%%%%%%%%%%%%%%%%%
\section[Operating Systems: Synchronization - Numerical Problems]
{Operating Systems:\\ Synchronization - Numerical Problems}

\subsection*{Q1. [NAT]}
A system has 4 resources of the same type. P1, P2, P3 processes are competing for them. 
- P1 holds 2 resources
- P2 holds 1 resource
- P3 holds 1 resource

No process is releasing a resource. What is the minimum number of additional resources that must be added to avoid deadlock?

\vspace{1em}

\subsection*{Q2. [NAT]}
A system has 5 processes $\{P_0, P_1, P_2, P_3, P_4\}$ and 3 resource types A, B, C.
- Total instances: A = 10, B = 5, C = 7  
- Allocation matrix:  
\[
\begin{bmatrix}
0 & 1 & 0 \\
2 & 0 & 0 \\
3 & 0 & 2 \\
2 & 1 & 1 \\
0 & 0 & 2 \\
\end{bmatrix}
\]  
- Maximum matrix:  
\[
\begin{bmatrix}
7 & 5 & 3 \\
3 & 2 & 2 \\
9 & 0 & 2 \\
2 & 2 & 2 \\
4 & 3 & 3 \\
\end{bmatrix}
\]  

What is the number of safe sequences possible?

\vspace{1em}

\subsection*{Q3. [NAT]}
A counting semaphore is initialized to 3. Ten processes simultaneously perform `wait()` on it. How many processes get blocked?

\vspace{1em}

\subsection*{Q4. [MCQ]}
In a bounded buffer producer-consumer problem:
- Buffer size = 7
- `empty` semaphore initialized to 7
- `full` semaphore initialized to 0
- 10 producers perform `wait(empty)`
- 4 consumers perform `wait(full)`

What is the value of the `empty` semaphore after these operations?

(a) 3 \quad (b) 7 \quad (c) 10 \quad (d) 1

\vspace{1em}

\subsection*{Q5. [NAT]}
Suppose 5 resources of a single type are available. There are 3 processes. Each process may request up to 3 resources. What is the maximum number of resources that can be allocated to prevent deadlock?

\vspace{1em}

\subsection*{Q6. [MCQ]}
In a Peterson's algorithm-based system, two processes are executing concurrently and take 3 cycles in the critical section and 2 cycles in the remainder section. What is the maximum number of times a process can enter its critical section in 20 cycles?

(a) 3 \quad (b) 4 \quad (c) 5 \quad (d) 6

\vspace{1em}

\subsection*{Q7. [NAT]}
In a round-robin scheduling system with semaphore synchronization:
- Semaphore S is initialized to 1
- Each process needs 3 critical section entries
- Each entry takes 4 cycles
- There are 4 processes

Assuming context switching is zero-cost, how many cycles are needed in total?

\vspace{1em}

\subsection*{Q8. [MCQ]}
In the Dining Philosopher problem with 5 philosophers using semaphores:
- Each philosopher takes 1 unit time to eat and 2 units to think
- Two philosophers are allowed to eat concurrently
What is the minimum total time required for all 5 to eat once?

(a) 3 \quad (b) 5 \quad (c) 7 \quad (d) 9

\vspace{1em}

\subsection*{Q9. [NAT]}
In a critical section problem with `n` processes using a binary semaphore, each critical section takes 5 ms, and non-critical section takes 15 ms. What is the maximum throughput (in processes/second)?

\vspace{1em}

\subsection*{Q10. [MCQ]}
A system using monitors has 6 threads accessing a shared resource. The monitor allows at most 2 threads to be inside at a time. If all threads request access simultaneously and one entry takes 10ms, what is the total time for all threads to execute?

(a) 20ms \quad (b) 30ms \quad (c) 40ms \quad (d) 60ms

%%%%%%%%%%%%%%%%%%%%%%%%%%%%%%%%%%%%%%%%%%%%%%%%%%%%%%%%%%%%%%%%%%%%%%%%%%%%%%%%%%%%%%%%%%%%%%%%%%
\section[Operating Systems: Hard Synchronization Problems]
{Operating Systems: Hard\\ Synchronization Problems}

\subsection*{Q1. [GATE 2022 - Rephrased]}
Consider the following threads \(T_1, T_2, T_3\) with 3 semaphores \(S_1, S_2, S_3\). The code for each thread is:
\begin{center}
\begin{tabular}{|c|c|c|}
\hline
\textbf{T\textsubscript{1}} & \textbf{T\textsubscript{2}} & \textbf{T\textsubscript{3}} \\
\hline
\texttt{$while(true)\{$} & \texttt{$while(true)\{$} & \texttt{$while(true)\{$} \\
\quad \texttt{wait(S3);} & \quad \texttt{wait(S1);} & \quad \texttt{wait(S2);} \\
\quad \texttt{print("C");} & \quad \texttt{print("B");} & \quad \texttt{print("A");} \\
\quad \texttt{signal(S2);} & \quad \texttt{signal(S3);} & \quad \texttt{signal(S1);} \\
\texttt{$\}$} & \texttt{$\}$} & \texttt{$\}$} \\
\hline
\end{tabular}
\end{center}

Which semaphore initialization gives the sequence \texttt{BCABCABCA...}?

(a) \(S_1 = 1, S_2 = 1, S_3 = 1\) \quad
(b) \(S_1 = 1, S_2 = 1, S_3 = 0\) \\
(c) \(S_1 = 1, S_2 = 0, S_3 = 0\) \quad
(d) \(S_1 = 0, S_2 = 1, S_3 = 1\)

\vspace{1em}

\subsection*{Q2. [GATE 2018 - Rephrased]}
In a classical synchronization producer-consumer problem has following variables:
\begin{enumerate}
    \item Buffer size = \(N\)
    \item Semaphores: \(empty = 0\), \(full = N\), \(mutex = 1\)
\end{enumerate}

Determine the correct values of \(P, Q, R, S\) in the following code:

\[
\begin{array}{|l|l|}
\hline
\textbf{Producer} & \textbf{Consumer} \\
\hline
\texttt{$do \{$} & \texttt{$do \{$} \\
\quad \texttt{wait(P);} & \quad \texttt{wait(R);} \\
\quad \texttt{wait(mutex);} & \quad \texttt{wait(mutex);} \\
\quad \texttt{/* Add/Consume item */} & \quad \texttt{/* Add/Consume item */} \\
\quad \texttt{signal(mutex);} & \quad \texttt{signal(mutex);} \\
\quad \texttt{signal(Q);} & \quad \texttt{signal(S);} \\
\texttt{$\}$ while(1);} & \texttt{$\}$ while(1);} \\
\hline
\end{array}
\]

(a) \(P = full, Q = full, R = empty, S = empty\) \\
(b) \(P = empty, Q = empty, R = full, S = full\) \\
(c) \(P = full, Q = empty, R = empty, S = full\) \\
(d) \(P = empty, Q = full, R = full, S = empty\)

\vspace{1em}

\subsection*{Q3. [GATE 2010 - Rephrased]}
Three processes \(P_0, P_1, P_2\) and semaphores initialized as: \(S_0 = 1, S_1 = 0, S_2 = 0\)
\[
\begin{array}{|l|l|l|}
\hline
P_0 & P_1 & P_2 \\
\hline
\texttt{while(true)} & \texttt{wait(S1);} & \texttt{wait(S2);} \\
\quad \texttt{wait(S0);} & \quad \texttt{signal(S0);} & \quad \texttt{signal(S0);} \\
\quad \texttt{print("0");} & & \\
\quad \texttt{signal(S1);} & & \\
\quad \texttt{signal(S2);} & & \\
\hline
\end{array}
\]

How many times can \(P_0\) print \texttt{"0"}?

(a) Exactly once \quad
(b) Exactly twice \quad
(c) At least twice \quad
(d) Exactly thrice

\vspace{1em}

\subsection*{Q4. [Semaphore Sequence]}
Three semaphores \(S1 = 0\), \(S2 = 1\), \(S3 = 0\). Processes:

\begin{enumerate}[label=-]
    \item P1: \texttt{wait(S1); print("X"); signal(S2);}
    \item P2: \texttt{wait(S2); print("Y"); signal(S3);}
    \item P3: \texttt{wait(S3); print("Z"); signal(S1);}
\end{enumerate}


What is the sequence printed by the processes?

(a) XYZXYZ... \quad
(b) ZYXZYX... \quad
(c) YZX... \quad
(d) Deadlock

\vspace{1em}

\subsection*{Q5. [Critical Section Throughput]}
A semaphore is used for mutual exclusion between $n=3$ threads. Each thread executes:
\begin{verbatim}
while(true){
    wait(S);
    critical_section();
    signal(S);
}
\end{verbatim}
If the critical section takes $10$ ms and the remainder section takes $5$ ms, what is the maximum possible throughput?

\begin{itemize}
    \item[(a)] 1 thread/sec
    \item[(b)] 100 threads/sec
    \item[(c)] 66.67 threads/sec
    \item[(d)] 200 threads/sec
\end{itemize}

\vspace{1em}

\subsection*{Q6. [Deadlock Detection]}
There are $3$ processes and $3$ identical resources. Each process holds one resource and requests one more. Are we guaranteed to have deadlock?

\begin{itemize}
    \item[(a)] Yes
    \item[(b)] No
    \item[(c)] Only if request order is circular
    \item[(d)] Depends on scheduling
\end{itemize}

\vspace{1em}

\subsection*{Q7. [Producer-Consumer Semaphore Values]}
In a bounded-buffer system:
\begin{itemize}
    \item Buffer size = $5$
    \item Initial values: \texttt{empty = 5, full = 0, mutex = 1}
    \item Producer uses: \texttt{wait(empty); wait(mutex);\\ signal(mutex); signal(full);}
    \item Consumer uses: \texttt{wait(full); wait(mutex);\\ signal(mutex); signal(empty);}
\end{itemize}
If $5$ producer and $5$ consumer operations are performed, what is the value of \texttt{empty}?

\begin{itemize}
    \item[(a)] 5
    \item[(b)] 0
    \item[(c)] 1
    \item[(d)] 2
\end{itemize}

\vspace{1em}

\subsection*{Q8. [Thread Interleaving with Semaphore]}
Two threads $T_1$ and $T_2$ run concurrently:
\begin{verbatim}
T1: wait(S); print("A"); signal(S);
T2: wait(S); print("B"); signal(S);
\end{verbatim}
If $S = 1$ initially, which sequences are possible?

\begin{itemize}
    \item[(a)] AB
    \item[(b)] BA
    \item[(c)] AABB
    \item[(d)] ABAB
\end{itemize}

\vspace{1em}

\subsection*{Q9. [Semaphore Starvation]}
A semaphore is initialized to $1$ and accessed by multiple threads. If a few threads wait indefinitely due to scheduler behavior, what phenomenon is observed?

\begin{itemize}
    \item[(a)] Deadlock
    \item[(b)] Starvation
    \item[(c)] Race Condition
    \item[(d)] Livelock
\end{itemize}

\vspace{1em}

\subsection*{Q10. [Gate-style Synchronization Loop]}
Consider $3$ processes synchronized as follows:
\begin{itemize}
    \item P1: \texttt{wait(S1); signal(S2);}
    \item P2: \texttt{wait(S2); signal(S3);}
    \item P3: \texttt{wait(S3); signal(S1);}
\end{itemize}
With initial values: $S1=1$, $S2=0$, $S3=0$

What type of execution order is enforced?

\begin{itemize}
    \item[(a)] Arbitrary
    \item[(b)] P1 $\rightarrow$ P2 $\rightarrow$ P3 cyclically
    \item[(c)] P3 always runs first
    \item[(d)] Deadlock
\end{itemize}



