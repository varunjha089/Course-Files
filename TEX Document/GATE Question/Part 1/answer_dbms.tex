\subsection*{Answers: Functional Dependencies}
\noindent\textbf{Answers:} Q1: (b), Q2: (d), Q3: (b), Q4: (a), Q5: (c)

\subsection*{Answers: Normalization}
\noindent\textbf{Answers:} Q1: (a), Q2: (b), Q3: (a), Q4: (b), Q5: (c), Q6: (a), Q7: (b), Q8: (b), Q9: (c), Q10: (c)

\subsection*{Answers: Normalization + Lossless Join + Dependency Preservation}
\noindent\textbf{Answers:} Q1: (b), Q2: (b), Q3: (a), Q4: (c), Q5: (c)

\subsection*{Answers: SQL Conceptual Questions}
\noindent\textbf{Answers:} Q1: (b), Q2: (c), Q3: (a), Q4: (a), Q5: (a), Q6: (a), Q7: (b), Q8: (a), Q9: (b), Q10: (b), Q11: (a), Q12: (b)

% ================================
\subsection*{Answers: Tuple Relational Calculus}
\noindent\textbf{Answers:} Q1: (b), Q2: (a), Q3: (d), Q4: (b), Q5: (b), Q6: (a)

% ================================
\subsection*{Answers: Transaction Scheduling}
\noindent\textbf{Answers:} Q1: (b), Q2: (c), Q3: (a), Q4: (a)

\subsection*{Answers: B and B+ Trees}
\noindent\textbf{Answers:} Q1: (depends on $n$, but generally at least $\lceil \frac{n+1}{d} \rceil - 1$), \\ 
Q2: (2), Q3: (c), Q4: (4), Q5: (approx. 30000), Q6: (3),  \\
Q7: B+ Trees maintain all data pointers in leaf level with linked leaves enabling fast range traversal,  \\
Q8: (h block accesses), Q9: (c), Q10: (d)

\subsection*{Answers: B+ Tree: Moderate to Hard}
\noindent\textbf{Answers:}  
Q1: (c), Q2: (a), (b), (d), Q3: 3, Q4: (d), Q5: (a), (b), Q6: (c), Q7: 10000, Q8: (b), Q9: (a), (b),  Q10: 3

\subsection*{Answers: B and B$^+$ Tree Balance}
\noindent\textbf{Answers:}  
Q1: (c), Q2: (a), (b), (d), Q3: (b), Q4: (b), Q5: (a), (b)