%%%%%%%%%%%%%%%%%%%%%%%%%%%%%%%% Introduction to Continuour Random Variable %%%%%%%%%%%%%%%%%%%%%%%%%%%%%%%%
\section[Introduction to Continuous Random Variables]
{Introduction to\\ Continuous Random Variables}

\subsection*{Q1}
Let $X$ be a continuous random variable with probability density function (PDF) given by
\[
f(x) = \begin{cases}
2x & 0 \leq x \leq 1 \\
0 & \text{otherwise}
\end{cases}
\]
Compute $\mathbb{P}(0.25 \leq X \leq 0.75)$.

\subsection*{Q2}
If $X$ is a continuous random variable with PDF $f(x)$, which of the following is \textbf{always true}?

\textbf{(a)} $f(x) \geq 0$ for all $x$ \\
\textbf{(b)} $f(x)$ is always continuous \\
\textbf{(c)} $f(x) \leq 1$ for all $x$ \\
\textbf{(d)} $f(x)$ has a finite number of local maxima

\subsection*{Q3}
Let $X$ be a continuous random variable with PDF:
\[
f(x) = \begin{cases}
k(1 - x^2), & -1 \leq x \leq 1 \\
0, & \text{otherwise}
\end{cases}
\]
Find the value of $k$ such that $f(x)$ is a valid PDF.

\subsection*{Q4}
Let $X$ be a continuous random variable uniformly distributed over $[a, b]$. Then the cumulative distribution function (CDF) $F_X(x)$ is:

\textbf{(a)} $\dfrac{x - a}{b - a}$ for $x \in [a, b]$ \\
\textbf{(b)} $1 - \dfrac{x - a}{b - a}$ for $x \in [a, b]$ \\
\textbf{(c)} Constant for all $x$ \\
\textbf{(d)} $\dfrac{b - x}{b - a}$ for $x \in [a, b]$

\subsection*{Q5}
Let $X$ be a continuous random variable with PDF $f(x)$ defined as
\[
f(x) = \begin{cases}
3x^2, & 0 \leq x \leq 1 \\
0, & \text{otherwise}
\end{cases}
\]
Compute $\mathbb{E}[X]$.

\subsection*{Q6}
Which of the following statements is \textbf{false} for a continuous random variable $X$?

\textbf{(a)} $\mathbb{P}(a < X \leq b) = \int_a^b f(x)\,dx$ \\
\textbf{(b)} $\mathbb{P}(X = c) = 0$ for any real $c$ \\
\textbf{(c)} $\mathbb{P}(a \leq X \leq b) = \mathbb{P}(a < X < b)$ \\
\textbf{(d)} The area under the PDF curve can be greater than $1$

\subsection*{Q7}
A continuous random variable $X$ has the PDF:
\[
f(x) = \begin{cases}
\lambda e^{-\lambda x}, & x \geq 0 \\
0, & x < 0
\end{cases}
\]
This distribution is known as:

\textbf{(a)} Normal distribution \\
\textbf{(b)} Poisson distribution \\
\textbf{(c)} Exponential distribution \\
\textbf{(d)} Uniform distribution

%%%%%%%%%%%%%%%%%%%%% Expectation and Variance of Continuous Random Variable %%%%%%%%%%%%%%%%%%%%%%%%
\section[Expectation and Variance of Continuous Random Variables]
{Expectation and Variance of\\ Continuous Random Variables}

\subsection*{Q1}
Let $X$ be a continuous random variable with PDF:
\[
f(x) = \begin{cases}
3x^2 & 0 \leq x \leq 1 \\
0 & \text{otherwise}
\end{cases}
\]
Compute $\mathbb{E}[X]$ and $\text{Var}(X)$.

\subsection*{Q2}
Let $X$ be a continuous random variable uniformly distributed over $[2, 4]$. Compute $\mathbb{E}[X^2]$ and $\text{Var}(X)$.

\subsection*{Q3}
The exponential distribution with parameter $\lambda > 0$ has the PDF
\[
f(x) = \begin{cases}
\lambda e^{-\lambda x} & x \geq 0 \\
0 & \text{otherwise}
\end{cases}
\]
Compute $\mathbb{E}[X]$ and $\text{Var}(X)$.

\subsection*{Q4}
Let $X$ be a continuous random variable with PDF:
\[
f(x) = \begin{cases}
\frac{1}{2}x, & 0 \leq x \leq 2 \\
0, & \text{otherwise}
\end{cases}
\]
Find $\mathbb{E}[X]$ and $\text{Var}(X)$.

\subsection*{Q5}
If $X$ is a continuous random variable with mean $\mu$ and variance $\sigma^2$, what is the variance of $Y = 3X + 5$?

\subsection*{Q6 (Conceptual)}
Which of the following is \textbf{true} for a continuous random variable $X$?

\textbf{(a)} $\mathbb{E}[a] = a$ for any constant $a$ \\
\textbf{(b)} $\mathbb{E}[aX + b] = a\mathbb{E}[X] + b$ \\
\textbf{(c)} $\text{Var}(aX + b) = a^2\text{Var}(X)$ \\
\textbf{(d)} All of the above

\subsection*{Q7 (Conceptual)}
Let $X$ and $Y$ be independent continuous random variables with variances $\sigma_X^2$ and $\sigma_Y^2$. What is the variance of $X + Y$?

\textbf{(a)} $\sigma_X^2 + \sigma_Y^2$ \\
\textbf{(b)} $\sigma_X^2 - \sigma_Y^2$ \\
\textbf{(c)} $|\sigma_X^2 - \sigma_Y^2|$ \\
\textbf{(d)} Depends on the distribution

%%%%%%%%%%%%%%%%%%%%%%%%%%%%%%%%%%% Uniform Random Variable %%%%%%%%%%%%%%%%%%%%%%%%%%%%%%%%%%%%%%%%%
\section{Uniform Random Variable: GATE-Style Questions}

\subsection*{Q1 (MCQ)}
Let $X \sim \mathcal{U}(2, 6)$. What is the probability that $X \leq 3$?

\textbf{(a)} $\frac{1}{4}$  
\textbf{(b)} $\frac{1}{2}$  
\textbf{(c)} $\frac{1}{6}$  
\textbf{(d)} $\frac{2}{3}$

\subsection*{Q2 (NAT)}
Let $X$ be uniformly distributed over $[1, 5]$. Find the expected value $\mathbb{E}[X]$.

\subsection*{Q3 (MCQ)}
Let $X$ be a discrete uniform random variable over the set $\{1,2,3,4,5,6\}$. What is $\mathbb{E}[X]$?

\textbf{(a)} $3.5$  
\textbf{(b)} $3$  
\textbf{(c)} $4$  
\textbf{(d)} $2.5$

\subsection*{Q4 (MSQ)}
Let $X \sim \mathcal{U}(0, 10)$. Which of the following are true?

\textbf{(a)} $\mathbb{E}[X] = 5$  
\textbf{(b)} $\text{Var}(X) = \frac{25}{3}$  
\textbf{(c)} $\mathbb{P}(X > 8) = 0.2$  
\textbf{(d)} PDF is constant for $x \in [0, 10]$

\subsection*{Q5 (MCQ)}
Let $X$ be a continuous uniform random variable on $[a,b]$. Which of the following is the correct formula for $\text{Var}(X)$?

\textbf{(a)} $\frac{(b-a)^2}{4}$  
\textbf{(b)} $\frac{(b-a)^2}{12}$  
\textbf{(c)} $\frac{(b+a)^2}{12}$  
\textbf{(d)} $\frac{(b+a)^2}{4}$

\subsection*{Q6 (NAT)}
Let $X$ be a continuous random variable with uniform distribution over $[3, 9]$. Compute $\text{Var}(X)$.

\subsection*{Q7 (Conceptual MCQ)}
Which of the following statements is always true for a continuous uniform distribution $\mathcal{U}(a, b)$?

\textbf{(a)} The PDF is symmetric about $\frac{a+b}{2}$  \\
\textbf{(b)} The mode equals the mean  \\
\textbf{(c)} The CDF is linear in $[a, b]$  \\
\textbf{(d)} $\text{Var}(X)$ depends only on the range $(b-a)$

\subsection*{Q8 (Conceptual MSQ)}
Which of the following statements are true for a discrete uniform distribution on the set $\{1, 2, ..., n\}$?

\textbf{(a)} All values have the same probability  \\
\textbf{(b)} $\text{Var}(X) = \frac{n^2 - 1}{12}$  \\
\textbf{(c)} $\mathbb{E}[X] = \frac{n+1}{2}$  \\
\textbf{(d)} $\text{PDF}(x) = \frac{1}{n}$ for all $x$

\subsection*{Q9 (NAT)}
Let $X$ be a discrete uniform random variable on $\{1, 2, 3, 4, 5\}$.\\ 
Find $\text{Var}(X)$.

\subsection*{Q10 (MCQ)}
Let $X \sim \mathcal{U}(-1, 1)$. What is the probability that $|X| < \frac{1}{2}$?

\textbf{(a)} $0.25$  
\textbf{(b)} $0.5$  
\textbf{(c)} $1$  
\textbf{(d)} $0.75$

%%%%%%%%%%%%%%%%%%%%%%%%%%%%%%% Normal Random Variable %%%%%%%%%%%%%%%%%%%%%%%%%%%%%%%%%%%%%%%%%%%%
\section{Normal Random Variable}

\subsection*{Q1 (MCQ)}
Let $X \sim \mathcal{N}(5, 4)$ (i.e., mean $= 5$, variance $= 4$). What is the probability that $X$ lies within one standard deviation of the mean?

\textbf{(a)} $0.68$  
\textbf{(b)} $0.95$  
\textbf{(c)} $0.997$  
\textbf{(d)} $0.34$

\subsection*{Q2 (NAT)}
If $X \sim \mathcal{N}(0, 1)$ and $P(X \leq z) = 0.8413$, then the value of $z$ is \underline{\hspace{2cm}} (round to 2 decimal places).

\subsection*{Q3 (MSQ)}
Which of the following are properties of the standard normal distribution?

\textbf{(a)} Mean = 0  
\textbf{(b)} Variance = 1  
\textbf{(c)} Symmetric about $x = 1$  
\textbf{(d)} Total area under PDF = 1

\subsection*{Q4 (MCQ)}
Let $X \sim \mathcal{N}(10, 1)$. What is the value of $P(9 \leq X \leq 11)$ approximately?

\textbf{(a)} $0.68$  
\textbf{(b)} $0.95$  
\textbf{(c)} $0.997$  
\textbf{(d)} $0.50$

\subsection*{Q5 (Conceptual MCQ)}
Which of the following best describes the shape of a normal distribution?

\textbf{(a)} Bell-shaped and symmetric  
\textbf{(b)} Skewed left  
\textbf{(c)} Skewed right  
\textbf{(d)} Uniform

\subsection*{Q6 (NAT)}
Let $X \sim \mathcal{N}(\mu, \sigma^2)$ with $\mu = 100$ and $\sigma = 15$. What value of $X$ corresponds to $Z = 2$?

\subsection*{Q7 (MSQ)}
Let $X \sim \mathcal{N}(10, 25)$. Which of the following transformations result in a standard normal variable $Z$?

\textbf{(a)} $Z = \frac{X - 10}{5}$  
\textbf{(b)} $Z = \frac{X - 10}{\sqrt{25}}$  
\textbf{(c)} $Z = \frac{10 - X}{5}$  
\textbf{(d)} $Z = \frac{X}{10}$

\subsection*{Q8 (Conceptual MCQ)}
If $X \sim \mathcal{N}(\mu, \sigma^2)$, then which of the following statements is true?

\textbf{(a)} $P(X = \mu) = 1$  
\textbf{(b)} $P(X = \mu) = 0$  
\textbf{(c)} $P(X < \mu) = 0$  
\textbf{(d)} $P(X < \mu) = 0.5$

\subsection*{Q9 (MCQ)}
Let $X \sim \mathcal{N}(0, 1)$. What is the value of $P(-2 \leq X \leq 2)$?

\textbf{(a)} $0.95$  
\textbf{(b)} $0.997$  
\textbf{(c)} $0.68$  
\textbf{(d)} $0.84$

\subsection*{Q10 (NAT)}
The height of adult males in a population is normally distributed with a mean of $170$ cm and a standard deviation of $10$ cm. What is the height below which approximately $97.5\%$ of the males lie?

%%%%%%%%%%%%%%%%%%%%%%%%%%%%%%%%%%%%%% Exponential Random Variable %%%%%%%%%%%%%%%%%%%%%%%%%%%%%%%%%%%%%%%%%
\section{Exponential Random Variable}

\subsection*{Q1 [MCQ]}
Let $X$ be an exponential random variable with mean $5$. What is $P(X > 10)$?

\begin{enumerate}[label=(\alph*)]
\item $e^{-2}$
\item $e^{-1}$
\item $1 - e^{-2}$
\item $0.5$
\end{enumerate}

\subsection*{Q2 [MSQ]}
Let $X$ be an exponential random variable with rate parameter $\lambda = 3$. Which of the following are true?

\begin{enumerate}[label=(\alph*)]
\item $E[X] = \dfrac{1}{3}$
\item $\text{Var}(X) = \dfrac{1}{9}$
\item $P(X > 1) = e^{-3}$
\item $P(X < 1) = 1 - e^{-3}$
\end{enumerate}

\subsection*{Q3 [NAT]}
Let $X$$~\sim$Exponential with mean $4$. Compute $\text{Var}(X)$.

\noindent\textbf{Answer:} \underline{\hspace{3cm}}

\subsection*{Q4 [MCQ]}
Which of the following represents the memoryless property of the exponential distribution?

\begin{enumerate}[label=(\alph*)]
\item $P(X > s + t) = P(X > s) \cdot P(X > t)$
\item $P(X > s + t \mid X > s) = P(X > t)$
\item $P(X < s + t \mid X > s) = P(X < t)$
\item $P(X > t \mid X > s + t) = P(X > s)$
\end{enumerate}

\subsection*{Q5 [MCQ]}
Let $X$ be exponentially distributed with $\lambda = 2$. What is the value of the PDF $f_X(x)$ at $x = 1$?

\begin{enumerate}[label=(\alph*)]
\item $2e^{-2}$
\item $e^{-2}$
\item $4e^{-2}$
\item $2e^{-1}$
\end{enumerate}

\subsection*{Q6 [MSQ]}
Let $X$ and $Y$ be independent exponential random variables with rate $\lambda = 1$. Which of the following are true?

\begin{enumerate}[label=(\alph*)]
\item $P(X < Y) = \dfrac{1}{2}$
\item $P(X = Y) = 0$
\item $P(\min(X, Y) > t) = e^{-2t}$
\item $\min(X, Y)$ is exponential with rate $2$
\end{enumerate}

\subsection*{Q7 [NAT]}
Let $X \sim \text{Exp}(\lambda)$ and $Y = aX$, where $a > 0$. Then $Y$ is exponentially distributed with rate $\lambda' = $ \underline{\hspace{3cm}}

\subsection*{Q8 [MCQ]}
Which of the following distributions has the memoryless property like exponential?

\begin{enumerate}[label=(\alph*)]
\item Normal
\item Poisson
\item Geometric
\item Uniform
\end{enumerate}

\subsection*{Q9 [MCQ]}
Let the lifetime (in hours) of a component be exponentially distributed with a mean of $50$ hours. What is the probability that it lasts more than $150$ hours?

\begin{enumerate}[label=(\alph*)]
\item $e^{-3}$
\item $e^{-1.5}$
\item $1 - e^{-3}$
\item $1 - e^{-1.5}$
\end{enumerate}

\subsection*{Q10 [MSQ - Conceptual]}
Which of the following are correct properties of exponential distribution?

\begin{enumerate}[label=(\alph*)]
\item Defined for $x > 0$
\item CDF is $F(x) = 1 - e^{-\lambda x}$
\item Always symmetric about the mean
\item Has memoryless property
\end{enumerate}

%%%%%%%%%%%%%%%%%%%%%%%%%%%%%%%%%%%%%% Hazard Rate Function %%%%%%%%%%%%%%%%%%%%%%%%%%%%%%%%%%%%%%%%%%%%%%%%%

\section{Hazard Rate Function}

\subsection*{Q1 [MCQ]}
Let $X$ be a continuous random variable with PDF:
\[ f_X(x) = \begin{cases}
\lambda e^{-\lambda x}, & x \geq 0 \\
0, & \text{otherwise}
\end{cases} \]
The hazard rate function $h(x)$ is:

\begin{enumerate}[label=(\alph*)]
\item $\lambda x$
\item $\dfrac{1}{\lambda}$
\item $\lambda$
\item $e^{-\lambda x}$
\end{enumerate}

\subsection*{Q2 [MSQ]}
Which of the following statements are true regarding the hazard function $h(x)$?

\begin{enumerate}[label=(\alph*)]
\item For an exponential distribution, the hazard rate is constant.
\item Increasing hazard rate implies increasing failure rate.
\item Hazard function is the ratio of the PDF and survival function.
\item The hazard function is always greater than 1 for exponential distributions.
\end{enumerate}

\subsection*{Q3 [MCQ]}
The survival function is $S(t) = e^{-\lambda t^2}$ for $t \geq 0$. The hazard rate function $h(t)$ is:

\begin{enumerate}[label=(\alph*)]
\item $2\lambda t$
\item $\lambda t$
\item $\lambda t^2$
\item $e^{-\lambda t^2}$
\end{enumerate}

\subsection*{Q4 [Conceptual MCQ]}
The hazard rate function is best described as:

\begin{enumerate}[label=(\alph*)]
\item The probability of failure at time $t$.
\item The expected number of failures up to time $t$.
\item The instantaneous failure rate at time $t$, given survival up to $t$.
\item The cumulative distribution function of the failure time.
\end{enumerate}

\subsection*{Q5 [NAT]}
Let the lifetime of a component be exponentially distributed with mean $5$ units. What is the value of the hazard rate?

\subsection*{Q6 [NAT]}
The survival function of a system is given by $S(t) = e^{-0.01t^2}$. Compute the hazard rate at $t = 10$.

\subsection*{Q7 [NAT]}
The PDF of a component's lifetime is given as:
\[
f(t) = \frac{3t^2}{1000}, \quad 0 \leq t \leq 10
\]
Calculate the hazard rate at $t = 5$. Use:
\[
h(t) = \frac{f(t)}{S(t)}, \quad S(t) = 1 - \int_0^t f(x) dx
\]

\subsection*{Q8 [Conceptual MCQ]}
Which of the following distributions have increasing hazard rate functions?

\begin{enumerate}[label=(\alph*)]
    \item Exponential distribution
    \item Weibull distribution with shape parameter $k > 1$
    \item Uniform distribution
    \item Normal distribution
\end{enumerate}

%%%%%%%%%%%%%%%%%%%%%%%%%%%% Properties of Random Variable %%%%%%%%%%%%%%%%%%%%%%%%%%%%%%%%%%%%%
\section{Properties of Expectation: Conceptual Questions}

\subsection*{Q1.}
Let $X$ and $Y$ be random variables such that $E[X] = 5$ and $E[Y] = 7$. What is $E[X + Y]$?

\subsection*{Q2.}
Let $X$ be a random variable and $a$, $b$ be constants. Which of the following is true?
\begin{enumerate}[label=(\alph*)]
    \item $E[aX + b] = aE[X] + b$
    \item $E[aX + b] = E[X] + ab$
    \item $E[aX + b] = a + bE[X]$
    \item $E[aX + b] = aE[X - b]$
\end{enumerate}

\subsection*{Q3.}
If $E[X] = \mu$ and $Y = X - \mu$, then $E[Y]$ is:
\begin{enumerate}[label=(\alph*)]
    \item $\mu$
    \item $0$
    \item $- \mu$
    \item $1$
\end{enumerate}

\subsection*{Q4.}
Let $X$ be a discrete random variable taking values $\{1, 2, 3\}$ with equal probability. What is $E[X^2]$?

\subsection*{Q5.}
Let $X$ and $Y$ be independent. Which of the following is true?
\begin{enumerate}[label=(\alph*)]
    \item $E[XY] = E[X] + E[Y]$
    \item $E[XY] = E[X]E[Y]$
    \item $E[XY] = \max(E[X], E[Y])$
    \item $E[XY] = 0$
\end{enumerate}

\subsection*{Q6.}
Which of the following is \textbf{not always true}?
\begin{enumerate}[label=(\alph*)]
    \item $E[a] = a$
    \item $E[aX] = aE[X]$
    \item $E[X + Y] = E[X] + E[Y]$
    \item $E[XY] = E[X] + E[Y]$ (without independence)
\end{enumerate}

\subsection*{Q7.}
If $X$ is a constant random variable with value $c$, then $E[X^2] = $ ?

\subsection*{Q8.}
If $X$ is a fair die roll, what is $E[3X + 4]$?

\subsection*{Q9.}
If $X$ and $Y$ are random variables and $E[X] = E[Y]$, is it necessary that $X = Y$?

\subsection*{Q10.}
If $X$ is a discrete r.v. and $f$ is a function, then $E[f(X)] = f(E[X])$ — is this always true?

\subsection*{Q11.}
Let $X$ be a non-negative random variable. Which of the following is necessarily true?
\begin{enumerate}[label=(\alph*)]
    \item $E[X] \geq 0$
    \item $E[X] > 0$
    \item $E[X] < 0$
    \item $E[X] = 0$
\end{enumerate}

\subsection*{Q12.}
If $X$ and $Y$ are uncorrelated, is $E[XY] = E[X]E[Y]$ always true?

\subsection*{Q13.}
Let $X$ be such that $P(X = 2) = 0.5$, $P(X = 4) = 0.5$. What is $E[(X - 3)^2]$?

\subsection*{Q14.}
If $X$ is a random variable such that $E[X^2] = 25$ and $E[X] = 3$, what is $\text{Var}(X)$?

\subsection*{Q15.}
Let $X$ be a continuous r.v. uniformly distributed on $[0, 2]$. Find $E[X^2]$.

\subsection*{Q16.}
If $X$ is a Bernoulli($p$) variable, then $E[X] = \rule{3cm}{0.15mm}$

\subsection*{Q17.}
True or False:\\ If $X$ and $Y$ are independent, then $E[f(X)g(Y)] = E[f(X)]E[g(Y)]$

\subsection*{Q18.}
Let $X$ be a random variable with $E[X] = \mu$. Which of the following is true?

\begin{enumerate}[label=(\alph*)]
    \item $E[(X - \mu)^2] = 0$
    \item $E[(X - \mu)^2] = \mu$
    \item $E[(X - \mu)^2] = \text{Var}(X)$
    \item None of the above
\end{enumerate}

\subsection*{Q19.}
Let $X$ and $Y$ be independent and $Z = X + Y$. Is it always true that $E[Z] = E[X] + E[Y]$?

\subsection*{Q20.}
A r.v. $X$ is symmetric about $0$. What is $E[X^3]$?

