\begin{center}
    \Large\textbf{UPSC Mains – GS Paper 1}\\
    \normalsize Chapter 7: \textit{Many Voices of a Nation}\\
    \vspace{0.5em}
    \textit{Based on \textbf{From Plassey to Partition and After} by Sekhar Bandyopadhyay}
\end{center}

\vspace{1em}

\subsection*{Q.1 [Muslim Alienation]}
What were the major reasons behind Muslim alienation from the Congress-led national movement in the early 20th century?

\subsection*{Q.2 [Communal Identity and Politics]}
How did socio-religious identity shape political mobilization among Indian Muslims?

\subsection*{Q.3 [Non-Brahman Movement]}
Evaluate the ideological foundations and political strategies of the Non-Brahman movement in South India.

\subsection*{Q.4 [Ambedkar and Dalit Protest]}
How did Ambedkar's vision for Dalit empowerment differ from the Congress approach to upliftment?

\subsection*{Q.5 [Depressed Classes and Nationalism]}
Analyze the participation and representation of the ‘Depressed Classes’ in the national movement.

\subsection*{Q.6 [Business and Politics]}
What role did Indian business elites play in the nationalist struggle? Were their interests always aligned with the freedom movement?

\subsection*{Q.7 [Class Collaboration or Conflict]}
Did the nationalist leadership effectively address the class concerns of capitalist and working classes?

\subsection*{Q.8 [Working Class Mobilization]}
Discuss the emergence and limitations of working-class movements in the context of Indian nationalism.

\subsection*{Q.9 [Leftist Influence]}
How did socialist and communist ideologies shape the working-class movements during the colonial period?

\subsection*{Q.10 [Women and Nationalism]}
How did the Indian nationalist movement transform the political and social role of women?

\subsection*{Q.11 [Women’s Organizations]}
Examine the contributions of early women’s organizations to both gender reform and anti-colonial resistance.

\subsection*{Q.12 [Gendered Politics of Inclusion]}
Was women’s participation in the nationalist movement merely symbolic? Evaluate critically.

\subsection*{Q.13 [Intersectionality]}
Analyze the interplay of caste, class, and gender in shaping participation in the nationalist movement.

\subsection*{Q.14 [Subaltern Voices]}
How did subaltern groups assert their agency within the broader framework of elite-led nationalism?

\subsection*{Q.15 [Limits of Congress Inclusion]}
To what extent did the Indian National Congress accommodate the voices of diverse social groups?
