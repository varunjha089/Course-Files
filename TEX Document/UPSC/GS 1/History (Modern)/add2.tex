\begin{center}
    \Large\textbf{UPSC Mains – GS Paper 1}\\
    \normalsize Chapter 8: \textit{The Colonial Legacy and the Post-Colonial State}\\
    \vspace{0.5em}
    \textit{Based on \textbf{From Plassey to Partition and After} by Sekhar Bandyopadhyay}
\end{center}

\vspace{1em}

\subsection*{Q.1 [Continuities in Administration]}
To what extent did independent India inherit the colonial administrative structure? Examine the continuities and changes.

\subsection*{Q.2 [Colonial Legacy in Economy]}
How did the colonial economic structure shape India's post-independence planning and development strategies?

\subsection*{Q.3 [Social Hierarchies and the State]}
Discuss how caste, class, and community relations—reshaped under colonialism—continued to influence the post-colonial Indian state.

\subsection*{Q.4 [Language and Nation-Building]}
Analyze the challenges of language reorganization in post-colonial India. How were these linked to colonial legacies?

\subsection*{Q.5 [Colonial Legal Framework]}
Evaluate the extent to which the Indian legal system retained colonial laws and structures after independence.

\subsection*{Q.6 [Partition's Long Shadow]}
Examine the impact of Partition on state formation, refugee rehabilitation, and communal relations in post-colonial India.

\subsection*{Q.7 [Nationalism after 1947]}
Was Indian nationalism transformed after independence, or did it continue the legacies of the colonial era?

\subsection*{Q.8 [Democracy and Authoritarianism]}
Did the colonial experience contribute to the authoritarian tendencies in post-independence India?

\subsection*{Q.9 [Land and Agrarian Policies]}
How did colonial land revenue policies affect post-independence agrarian reforms?

\subsection*{Q.10 [Federalism and Centre-State Relations]}
Trace the evolution of federalism in India and its links to colonial administrative geography.

\subsection*{Q.11 [Military and Security Structure]}
To what extent did India’s military and policing institutions post-1947 evolve out of colonial frameworks?

\subsection*{Q.12 [Colonial Mindset and Bureaucracy]}
Analyze the continuance of the colonial bureaucratic ethos in independent India's civil services.

\subsection*{Q.13 [Secularism and the State]}
Was the idea of secularism in India a product of colonial experience or an indigenous political development?

\subsection*{Q.14 [Educational Continuities]}
Examine how the colonial model of education shaped post-independence knowledge production and nation-building.

\subsection*{Q.15 [Postcolonial Historiography]}
How has postcolonial historiography critiqued the narrative of India's transition from colony to nation?
