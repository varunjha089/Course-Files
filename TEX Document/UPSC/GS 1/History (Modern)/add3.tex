\begin{center}
    \Large\textbf{UPSC Mains – GS Paper 1}\\
    \normalsize Chapter 9: \textit{Nationalism and Social Groups: Mobilization and Inclusion}\\
    \vspace{0.5em}
    \textit{Based on \textbf{From Plassey to Partition and After} by Sekhar Bandyopadhyay}
\end{center}

\vspace{1em}

\subsection*{Q.1 [Caste and Nationalism]}
How did caste influence the nature and scope of the nationalist movement in colonial India?

\subsection*{Q.2 [Dalits and the Freedom Struggle]}
Discuss the participation of Dalits in the freedom movement and the challenges they faced in aligning with Congress-led nationalism.

\subsection*{Q.3 [Ambedkar’s Vision vs Congress Nationalism]}
Compare and contrast the nationalist vision of Dr. B.R. Ambedkar with that of the Indian National Congress.

\subsection*{Q.4 [Peasant Mobilization]}
How did the Congress and other political groups mobilize the Indian peasantry during the national movement?

\subsection*{Q.5 [Women’s Role]}
Assess the nature and extent of women’s participation in the nationalist movement. Were their concerns adequately represented?

\subsection*{Q.6 [Tribal Communities]}
To what extent were tribal communities integrated into or alienated from the mainstream nationalist movement?

\subsection*{Q.7 [Muslim Participation]}
Analyze the trajectory of Muslim participation in the freedom struggle from the Khilafat movement to Partition.

\subsection*{Q.8 [Class and Urban Workers]}
Discuss how class identity shaped the political participation of the working class in India’s freedom movement.

\subsection*{Q.9 [Social Reform vs Political Mobilization]}
Was social reform a prerequisite or a byproduct of nationalist mobilization?

\subsection*{Q.10 [Congress and Marginalized Groups]}
Evaluate the Indian National Congress’s strategies and limitations in representing marginalized communities.

\subsection*{Q.11 [Construct of ‘Nation’]}
How inclusive was the construct of “nation” in the discourse of Indian nationalism?

\subsection*{Q.12 [Communists and Social Movements]}
Assess the role of communist and socialist movements in mobilizing peasants, workers, and Dalits in colonial India.

\subsection*{Q.13 [Gandhian Inclusion Strategy]}
How did Gandhi attempt to integrate women, Harijans, and peasants into the national mainstream?

\subsection*{Q.14 [Language and Identity]}
Examine how language became a site of both inclusion and exclusion in the nationalist discourse.

\subsection*{Q.15 [Limits of Inclusion]}
Despite its mass appeal, Indian nationalism had several blind spots. Discuss with reference to at least two marginalized groups.
