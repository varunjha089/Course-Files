
\begin{center}
    \Large\textbf{UPSC Mains – GS Paper 1}\\
    \normalsize Chapter 1: \textit{Transition of the Eighteenth Century}\\
    \vspace{0.5em}
    \textit{Based on \textbf{From Plassey to Partition} by Sekhar Bandyopadhyay}
\end{center}

\vspace{1em}

\subsection*{Q.1 [Mughal Decline and Regional Powers]}
"The decline of the Mughal Empire did not result in a political vacuum, but in the rise of regional powers with their own legitimacy." Discuss with examples.

\subsection*{Q.2 [Historiography of the 18th Century]}
Critically analyze the nature of the eighteenth century in India: was it a period of decay or transformation?

\subsection*{Q.3 [Economic Resilience]}
How did the economic structure of India sustain itself despite political fragmentation after Aurangzeb?

\subsection*{Q.4 [Role of the Marathas]}
Evaluate the role of the Marathas in shaping the political landscape of eighteenth-century India.

\subsection*{Q.5 [Agrarian Structure and Revenue]}
How did the weakening of Mughal central authority impact agrarian society and revenue collection?

\subsection*{Q.6 [Battle of Plassey]}
"The Battle of Plassey was less a battle and more a betrayal." Examine in the context of the East India Company’s rise.

\subsection*{Q.7 [European Trading Companies]}
Discuss the impact of European trading companies on Indian politics during the first half of the 18th century.

\subsection*{Q.8 [Successor States]}
What were the key features of successor states like Awadh, Bengal, and Hyderabad? How did they maintain autonomy?

\subsection*{Q.9 [Dark Age Debate]}
Was the eighteenth century truly a "dark age" in Indian history? Discuss with reference to recent historiography.

\subsection*{Q.10 [Urban Economy]}
Describe the socio-economic profile of urban centres like Murshidabad, Surat, and Delhi in the 18th century.

\subsection*{Q.11 [Company's Political Rise]}
Examine the transformation of the East India Company from a trading body to a political power in the 18th century.

\subsection*{Q.12 [Regionalism and Colonial Resistance]}
How did regional politics in the 18th century affect India's resistance to colonialism?

\subsection*{Q.13 [Succession Policies]}
To what extent did the failure of succession policies under the Mughals contribute to the decline of imperial control?

\subsection*{Q.14 [Carnatic Wars]}
Assess the nature and significance of the Carnatic Wars in the context of Anglo-French rivalry in India.

\subsection*{Q.15 [Forces of Political Transformation]}
Examine how internal and external forces together contributed to the transformation of Indian polity in the 18th century.

