\begin{center}
    \Large\textbf{UPSC Mains – GS Paper 1}\\
    \normalsize Chapter 9: \textit{After Independence and Partition}\\
    \vspace{0.5em}
    \textit{Based on \textbf{From Plassey to Partition and After} by Sekhar Bandyopadhyay}
\end{center}

\vspace{1em}

\subsection*{Q.1 [The Transition]}
Discuss the key political and administrative challenges faced by the Indian state immediately after independence.

\subsection*{Q.2 [Partition and Refugees]}
Analyze the demographic, economic, and social consequences of the refugee crisis following Partition.

\subsection*{Q.3 [Princely States – Kashmir and Hyderabad]}
Evaluate the process of integration of princely states, with special focus on Kashmir and Hyderabad.

\subsection*{Q.4 [The Communist Challenge]}
How did the post-independence Indian state respond to Communist-led movements and opposition?

\subsection*{Q.5 [Constitutional Framing]}
What were the foundational principles of the Indian Constitution? How did it address diversity and inequality?

\subsection*{Q.6 [Democratic Institutions]}
Assess the strengths and limitations of India's early democratic institutions during the Nehruvian era.

\subsection*{Q.7 [Nehruvian Economic Model]}
Examine the ideology and structure of the Nehruvian model of state-led development.

\subsection*{Q.8 [Planning Commission and Five-Year Plans]}
To what extent did the Planning Commission achieve its intended goals in the first two decades?

\subsection*{Q.9 [Social Reform and Law]}
Discuss how the post-colonial state used legal reforms to address caste and gender inequalities.

\subsection*{Q.10 [Foreign Policy and Non-Alignment]}
Analyze the evolution of India’s foreign policy under Nehru, with a focus on non-alignment and internationalism.

\subsection*{Q.11 [Centre-State Relations]}
Critically evaluate the balance of power between the Centre and States in independent India’s early decades.

\subsection*{Q.12 [Emergency and Authoritarianism]}
Was the Emergency (1975–77) an aberration or an outcome of deeper institutional weaknesses?

\subsection*{Q.13 [Decline of Congress System]}
What were the major factors behind the decline of the Congress system by the late 1960s and 1970s?

\subsection*{Q.14 [Popular Movements and Political Realignment]}
Discuss the emergence of regional and caste-based parties in the context of the decline of the Congress.

\subsection*{Q.15 [Continuity and Change]}
To what extent did the post-independence Indian state break away from colonial structures in its political, economic, and social vision?
