\begin{center}
    \Large\textbf{UPSC Mains – GS Paper 1}\\
    \normalsize Chapter 4: \textit{Emergence of Indian Nationalism}\\
    \vspace{0.5em}
    \textit{Based on \textbf{From Plassey to Partition} by Sekhar Bandyopadhyay}
\end{center}

\vspace{1em}

\subsection*{Q.1 [Historiography of Nationalism]}
Discuss the different historiographical interpretations of Indian nationalism.

\subsection*{Q.2 [Economic Critique of Colonialism]}
How did early nationalists use economic critique as a tool against colonial rule?

\subsection*{Q.3 [Rise of Middle Class]}
Analyze the role of the newly emerging Indian middle class in shaping early nationalist consciousness.

\subsection*{Q.4 [Formation of Indian National Congress]}
Examine the circumstances leading to the formation of the Indian National Congress in 1885.

\subsection*{Q.5 [Role of Moderates]}
Assess the achievements and limitations of the Moderate phase of the Indian National Congress (1885–1905).

\subsection*{Q.6 [Early Nationalist Methods]}
Why did the early nationalists prefer constitutional agitation over mass movements?

\subsection*{Q.7 [Economic Nationalism]}
Explain the concept of economic nationalism and its role in building a national identity in colonial India.

\subsection*{Q.8 [Class and Caste in Nationalism]}
To what extent did caste and class affect the reach and appeal of early nationalism?

\subsection*{Q.9 [Swadeshi Movement: Roots]}
How did the Partition of Bengal in 1905 transform Indian nationalism into a mass-based movement?

\subsection*{Q.10 [Pre-Gandhian Nationalism]}
Discuss the strengths and weaknesses of Indian nationalism before the advent of Mahatma Gandhi.

\subsection*{Q.11 [Vernacular Press and Political Awakening]}
Examine the role of vernacular press in shaping nationalist sentiment during the late 19th century.

\subsection*{Q.12 [Education and Political Consciousness]}
How did Western education contribute to the emergence of political consciousness in colonial India?

\subsection*{Q.13 [Communalism and Nationalism]}
How did early nationalist politics deal with the challenge of communalism?

\subsection*{Q.14 [Role of Provincial Associations]}
Evaluate the significance of regional political associations before the formation of the Indian National Congress.

\subsection*{Q.15 [Leadership and Ideology]}
Assess the role of leaders like Dadabhai Naoroji, Gopal Krishna Gokhale, and Surendranath Banerjea in shaping early nationalism.
