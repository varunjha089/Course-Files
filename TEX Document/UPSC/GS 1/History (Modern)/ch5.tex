\begin{center}
    \Large\textbf{UPSC Mains – GS Paper 1}\\
    \normalsize Chapter 5: \textit{Early Nationalism – Discontent and Dissension}\\
    \vspace{0.5em}
    \textit{Based on \textbf{From Plassey to Partition} by Sekhar Bandyopadhyay}
\end{center}

\vspace{1em}

\subsection*{Q.1 [Moderate-Extremist Split]}
What were the major ideological and strategic differences between the Moderates and Extremists in the Indian National Congress?

\subsection*{Q.2 [Swadeshi Movement and Boycott]}
Evaluate the impact of the Swadeshi and Boycott movements on Indian nationalism after the Partition of Bengal (1905).

\subsection*{Q.3 [Extremist Ideology]}
What were the key features of the Extremist approach to nationalism? How did it differ from earlier nationalist methods?

\subsection*{Q.4 [Hindu Revivalism]}
Analyze the role of Hindu revivalist movements in shaping nationalist politics in the early 20th century.

\subsection*{Q.5 [Muslim Politics]}
Trace the evolution of Muslim political consciousness leading to the formation of the All India Muslim League in 1906.

\subsection*{Q.6 [Partition of Bengal]}
How did the Partition of Bengal (1905) become a turning point in Indian politics?

\subsection*{Q.7 [Dadabhai Naoroji’s Contributions]}
Discuss the economic ideas and political role of Dadabhai Naoroji in the national movement.

\subsection*{Q.8 [Role of Tilak]}
Critically examine Bal Gangadhar Tilak’s role in transforming Indian nationalism during the Extremist phase.

\subsection*{Q.9 [Surat Split]}
What led to the Surat Split of 1907? What were its implications for the national movement?

\subsection*{Q.10 [Revivalist vs Reformist Trends]}
Compare the reformist and revivalist strands of nationalism in the early 20th century.

\subsection*{Q.11 [Cultural Symbols and Nationalism]}
How did early nationalists use religious and cultural symbols to build a collective political identity?

\subsection*{Q.12 [Press and Nationalism]}
Assess the role of Indian language newspapers and journals in mobilizing nationalist sentiments during the Swadeshi era.

\subsection*{Q.13 [Women's Participation]}
How did women engage in the Swadeshi movement? What was the impact on gender roles?

\subsection*{Q.14 [Impact on Rural India]}
To what extent did the early nationalist and Swadeshi movements succeed in reaching beyond the urban intelligentsia?

\subsection*{Q.15 [Legacy of Early 20th Century Movements]}
Evaluate the long-term contributions of early 20th-century movements to the later phases of the freedom struggle.
