\begin{center}
    \Large\textbf{UPSC Mains – GS Paper 1}\\
    \normalsize Chapter 6: \textit{The Age of Gandhian Politics}\\
    \vspace{0.5em}
    \textit{Based on \textbf{From Plassey to Partition} by Sekhar Bandyopadhyay}
\end{center}

\vspace{1em}

\subsection*{Q.1 [Gandhi’s Ideology]}
What were the core principles of Mahatma Gandhi's political philosophy, and how did they shape the Indian freedom struggle?

\subsection*{Q.2 [Non-Cooperation Movement]}
Discuss the significance of the Non-Cooperation Movement (1920–22) in transforming Indian nationalism into a mass movement.

\subsection*{Q.3 [Role of Khilafat Movement]}
Evaluate the role of the Khilafat Movement in fostering Hindu-Muslim unity during the early Gandhian phase.

\subsection*{Q.4 [Constructive Programme]}
What was Gandhi’s Constructive Programme, and how did it relate to his vision of Swaraj?

\subsection*{Q.5 [Civil Disobedience Movement]}
How was the Civil Disobedience Movement (1930–34) different from the Non-Cooperation Movement in terms of strategy and mass participation?

\subsection*{Q.6 [Salt March and Symbolism]}
Why was the Salt March chosen as the launchpad for Civil Disobedience? Comment on its symbolic and practical significance.

\subsection*{Q.7 [Gandhi-Irwin Pact]}
What were the terms and political implications of the Gandhi-Irwin Pact (1931)?

\subsection*{Q.8 [Limitations of Mass Movements]}
To what extent were Gandhi's mass movements inclusive of peasants, workers, women, and marginalized castes?

\subsection*{Q.9 [Round Table Conferences]}
Assess the objectives and outcomes of the Round Table Conferences (1930–32) with regard to Indian self-governance.

\subsection*{Q.10 [Role of Congress under Gandhi]}
How did Mahatma Gandhi change the character and organisation of the Indian National Congress?

\subsection*{Q.11 [Quit India Movement]}
What made the Quit India Movement (1942) more radical and widespread compared to earlier Gandhian movements?

\subsection*{Q.12 [Limits of Gandhian Strategy]}
Critically assess the limitations of Gandhi’s methods in achieving complete independence from British rule.

\subsection*{Q.13 [Role of Women in Gandhian Politics]}
Examine how Gandhian politics influenced the participation and visibility of women in the freedom struggle.

\subsection*{Q.14 [Gandhi’s Relationship with Ambedkar]}
Discuss the ideological differences between Gandhi and Ambedkar with reference to the Poona Pact.

\subsection*{Q.15 [Legacy of Gandhian Politics]}
What is the long-term legacy of Gandhian politics in post-independence India?

