
\begin{center}
    \Large\textbf{UPSC Mains – GS Paper 1}\\
    \normalsize Chapter 2: \textit{British Empire in India}\\
    \vspace{0.5em}
    \textit{Based on \textbf{From Plassey to Partition} by Sekhar Bandyopadhyay}
\end{center}

\vspace{1em}

\subsection*{Q.1 [Imperial Ideology]}
How did the British justify their rule in India? Critically examine the ideological foundations of the British Empire.

\subsection*{Q.2 [Role of Parliament and the Company]}
Discuss the evolution of British parliamentary control over the East India Company’s administration in India.

\subsection*{Q.3 [Permanent Settlement]}
Evaluate the economic and social impact of the Permanent Settlement on Indian society.

\subsection*{Q.4 [Revenue Systems]}
Compare and contrast the major land revenue systems introduced by the British in India.

\subsection*{Q.5 [Administrative Structure]}
How did the British administrative apparatus evolve between 1765 and 1857?

\subsection*{Q.6 [British Policy and Indian Peasantry]}
Discuss the impact of British land revenue policies on the Indian peasantry in the 18th and early 19th century.

\subsection*{Q.7 [Doctrine of Lapse and Subsidiary Alliances]}
Analyze the role of Doctrine of Lapse and Subsidiary Alliances in the expansion of British territory.

\subsection*{Q.8 [Colonial Economy]}
To what extent did British economic policies transform India into a colonial economy?

\subsection*{Q.9 [Impact on Artisans and Trade]}
How did the British rule affect India’s traditional industries and artisan classes?

\subsection*{Q.10 [Judicial and Legal Reforms]}
Discuss the nature and consequences of the British introduction of Western legal and judicial institutions in India.

\subsection*{Q.11 [Education Policy]}
How did the early British education policy serve colonial interests? Was it merely a tool of cultural imperialism?

\subsection*{Q.12 [Ideological Hegemony]}
Explain how the British established ideological hegemony in India using law, education, and institutions.

\subsection*{Q.13 [Colonial Knowledge Systems]}
Discuss how the British production of knowledge (gazetteers, surveys, censuses) contributed to imperial control.

\subsection*{Q.14 [Company to Crown]}
Trace the administrative and institutional changes between 1757 and 1858 that enabled the transition from Company rule to Crown rule.

\subsection*{Q.15 [Fiscal Drain]}
Critically evaluate the theory of “economic drain” under British rule and its implications for Indian development.
