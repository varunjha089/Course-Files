\begin{center}
    \Large\textbf{UPSC Mains – GS Paper 1}\\
    \normalsize Chapter 8: \textit{Freedom with Partition}\\
    \vspace{0.5em}
    \textit{Based on \textbf{From Plassey to Partition and After} by Sekhar Bandyopadhyay}
\end{center}

\vspace{1em}

\subsection*{Q.1 [Quit India Movement]}
Critically analyze the Quit India Movement in terms of mass participation and its limitations.

\subsection*{Q.2 [Leftist Role in the 1940s]}
Evaluate the role of the Communist Party and other Leftist groups in India’s anti-colonial struggle during the 1940s.

\subsection*{Q.3 [INA and Subhas Chandra Bose]}
Assess the significance of Subhas Chandra Bose and the Indian National Army in the final phase of India’s independence struggle.

\subsection*{Q.4 [Naval Mutiny and Revolts]}
To what extent did the 1946 Royal Indian Navy Mutiny reflect popular discontent with British rule?

\subsection*{Q.5 [Communal Politics in the 1940s]}
Discuss the rise of communalism in the 1940s and its effect on the unity of the nationalist movement.

\subsection*{Q.6 [Wavell Plan and Simla Conference]}
What were the objectives and outcomes of the Wavell Plan and the Simla Conference of 1945?

\subsection*{Q.7 [Cabinet Mission Plan]}
Examine the provisions of the Cabinet Mission Plan. Why did it ultimately fail to prevent Partition?

\subsection*{Q.8 [Direct Action Day]}
Analyze the events and consequences of Direct Action Day (August 16, 1946). How did it change the trajectory toward Partition?

\subsection*{Q.9 [Mountbatten Plan]}
Explain the key features of the Mountbatten Plan. Why was it acceptable to both the Congress and the Muslim League?

\subsection*{Q.10 [British Policy and Withdrawal]}
Was the British withdrawal from India a planned transition or a hasty exit? Substantiate your answer.

\subsection*{Q.11 [Partition and Communal Violence]}
Discuss the patterns, scale, and causes of communal violence during Partition. How did it affect post-independence nation-building?

\subsection*{Q.12 [Gender and Partition]}
Critically examine the gendered nature of Partition violence. How did women experience the trauma differently?

\subsection*{Q.13 [Refugees and Rehabilitation]}
How did the Indian state respond to the refugee crisis generated by Partition? Evaluate the effectiveness of rehabilitation policies.

\subsection*{Q.14 [Popular Memory and Narratives]}
How has Partition been remembered in Indian popular memory, oral histories, and literature?

\subsection*{Q.15 [Was Partition Inevitable?]}
Was the Partition of India inevitable by 1947? Discuss with reference to political, communal, and administrative developments.
