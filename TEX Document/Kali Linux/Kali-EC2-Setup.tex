\documentclass[12pt]{article}
\usepackage[a4paper,margin=1in]{geometry}
\usepackage{tcolorbox}
\usepackage{titlesec}
\usepackage{listings}
\usepackage{color}
\usepackage{fancyvrb}
\usepackage{enumitem}
\usepackage{hyperref}
\usepackage{graphicx}

\definecolor{lightgray}{gray}{0.95}
\definecolor{darkblue}{rgb}{0,0.1,0.6}

\titleformat{\section}{\large\bfseries}{Q\thesection}{1em}{}
\titleformat{\subsection}[runin]{\bfseries}{A:}{0.5em}{}[.]

\lstset{
    backgroundcolor=\color{lightgray},
    basicstyle=\ttfamily\small,
    frame=single,
    breaklines=true,
    postbreak=\mbox{\textcolor{red}{$\hookrightarrow$}\space}
}

\begin{document}

\title{Kali Linux GPU and Hashcat Setup: Questions and Answers}
\author{}
\date{}
\maketitle

\section{How to verify if my Tesla T4 GPU is properly installed and usable with NVIDIA drivers?}

\subsection
Run the command:
\begin{lstlisting}
nvidia-smi
\end{lstlisting}

You should see the Tesla T4 listed along with the driver and CUDA version:

\begin{tcolorbox}[colback=lightgray]
GPU: Tesla T4\\
Driver Version: 550.163.01\\
CUDA Version: 12.4\\
Memory Usage: 1 MiB / 15360 MiB\\
Processes: No running processes
\end{tcolorbox}

This confirms your GPU is active and ready for workloads like CUDA, Hashcat, or PyTorch.

\section{How do I install Hashcat on Kali and check if it sees the GPU?}

\subsection
Run the following:
\begin{lstlisting}
sudo apt update
sudo apt install -y hashcat
hashcat -I
\end{lstlisting}

The output of `hashcat -I` should list your Tesla T4 as an available CUDA device.

\section{How do I crack a known password like \texttt{Varun@2025} using Hashcat?}

\subsection
Hashcat works on hashes, not plaintext. So first generate the MD5 hash:
\begin{lstlisting}
echo -n "Varun@2025" | md5sum
# Output: f271ea3b226cddf3d83e5f256763cbd5
\end{lstlisting}

Then create:
\begin{itemize}[noitemsep]
  \item \texttt{hash.txt} — contains the hash
  \item \texttt{mywordlist.txt} — contains the guessed password
\end{itemize}

\begin{lstlisting}
echo "f271ea3b226cddf3d83e5f256763cbd5" > hash.txt
echo "Varun@2025" > mywordlist.txt
\end{lstlisting}

Now run Hashcat:
\begin{lstlisting}
hashcat -m 0 -a 0 hash.txt mywordlist.txt
hashcat -m 0 -a 0 hash.txt mywordlist.txt --show
\end{lstlisting}

Output will be:
\begin{tcolorbox}[colback=lightgray]
f271ea3b226cddf3d83e5f256763cbd5:Varun@2025
\end{tcolorbox}

\section{Image Question: What does the following output mean? (screenshot of NVIDIA-SMI)}

\subsection
\textbf{Question:} How do I interpret the output of \texttt{nvidia-smi} after successful GPU driver installation?

\begin{tcolorbox}[colback=lightgray]
This confirms that the NVIDIA driver is properly installed and detects the Tesla T4 GPU. The card is idle with 1 MiB of memory in use and zero active processes. The CUDA version (12.4) is compatible with most modern AI/ML tools.
\end{tcolorbox}

\section{What happens if the \texttt{kali} user password is locked, and how to fix it?}

\subsection
Running:
\begin{lstlisting}
sudo passwd -S kali
\end{lstlisting}
If output is:
\begin{tcolorbox}[colback=lightgray]
kali L ...
\end{tcolorbox}
It means the account is locked (`L`).

To fix:
\begin{lstlisting}
echo "kali:Varun@2025" | sudo chpasswd
sudo passwd -u kali
\end{lstlisting}

Check again:
\begin{lstlisting}
sudo passwd -S kali
# Expected: kali P ...
\end{lstlisting}

\section{What are these prompts during installation? (from screenshots)}

\subsection
\textbf{Q: Should I allow Wireshark to let non-root users capture packets?}

\begin{tcolorbox}[colback=lightgray]
No — on EC2 or server environments, select \texttt{<No>} for security. Packet capture is unnecessary unless explicitly needed.
\end{tcolorbox}

\subsection
\textbf{Q: What is the `macchanger` prompt asking during installation?}

\begin{tcolorbox}[colback=lightgray]
It asks whether to randomize MAC addresses on every connection. On cloud instances like EC2, select \texttt{<No>} to prevent networking issues.
\end{tcolorbox}

\subsection
\textbf{Q: What does the `kismet` prompt mean?}

\begin{tcolorbox}[colback=lightgray]
Kismet requests elevated privileges via setuid. Since EC2 has no Wi-Fi interface, select \texttt{<No>}.
\end{tcolorbox}

\subsection
\textbf{Q: What is the `sslh` mode selection?}

\begin{tcolorbox}[colback=lightgray]
Choose \texttt{standalone} mode for better performance. It runs persistently and is recommended for most Kali setups.
\end{tcolorbox}

\section{How do I benchmark GPU cracking speed?}

\subsection
Run:
\begin{lstlisting}
hashcat -b
\end{lstlisting}

This benchmarks your GPU against all supported hashing algorithms, showing speeds like hashes/sec for MD5, SHA256, bcrypt, etc.

\end{document}
