\documentclass[14pt]{extarticle}
\usepackage[margin=1in]{geometry}
\usepackage{amsmath}
\usepackage{listings}
\usepackage{xcolor}
\usepackage{graphicx}
\usepackage{tcolorbox}
\usepackage{multicol}
\usepackage{titlesec}
\usepackage{enumitem}
\usepackage{makecell}

\definecolor{codegray}{gray}{0.95}
\lstset{
    basicstyle=\ttfamily\small,
    frame=single,
    breaklines=true,
    columns=fullflexible,
}

\title{\textbf{Insertion Sort }}
\author{Varun Kumar}
\date{July 4, 2025}

\begin{document}

\maketitle

\section*{1. Logic}

Insertion Sort builds the sorted array one element at a time by repeatedly picking the next element and inserting it into the correct position among the already sorted elements.

\begin{tcolorbox}[
  colback=white,
  colframe=black,
  title=Key Idea
]
At the $i^{th}$ iteration, the first $i$ elements are sorted, and the $(i+1)^{th}$ element is inserted into its correct position.
\end{tcolorbox}

\section*{2. Number of Comparisons and Shifts}

Let $n$ be the number of elements.

\subsection*{Worst Case (Reverse Sorted)}

\begin{itemize}
    \item Comparisons: $\frac{n(n-1)}{2}$
    \item Shifts: $\frac{n(n-1)}{2}$
\end{itemize}

\subsection*{Best Case (Already Sorted)}

\begin{itemize}
    \item Comparisons: $n - 1$
    \item Shifts: $0$
\end{itemize}

\section*{3. Optimal Behavior}

Insertion sort is adaptive by default. It reduces unnecessary comparisons and shifts when the array is already or partially sorted.

\section*{4. Pseudocode}

\begin{lstlisting}[language=Python]
function insertionSort(arr):
    n = length(arr)
    for i = 1 to n-1:
        key = arr[i]
        j = i - 1
        while j >= 0 and arr[j] > key:
            arr[j+1] = arr[j]
            j = j - 1
        arr[j+1] = key
\end{lstlisting}

\begin{tcolorbox}[
  colback=white,
  colframe=black,
  title=Stability Note
]
Insertion Sort is stable because it inserts elements in a way that preserves the relative order of equal elements.
\end{tcolorbox}

\section*{5. Example Walkthrough}

Given: \texttt{[5, 1, 4, 2, 8]}

\subsection*{Pass 1}
\texttt{[5, 1, 4, 2, 8]} $\Rightarrow$ \texttt{[1, 5, 4, 2, 8]}

\subsection*{Pass 2}
\texttt{[1, 5, 4, 2, 8]} $\Rightarrow$ \texttt{[1, 4, 5, 2, 8]}

\subsection*{Pass 3}
\texttt{[1, 4, 5, 2, 8]} $\Rightarrow$ \texttt{[1, 2, 4, 5, 8]}

\subsection*{Pass 4}
\texttt{[1, 2, 4, 5, 8]} → Already in position → Done

\section*{6. Time \& Space Complexity and Its Properties}

\begin{center}
\begin{tabular}{|l|l||l|l|}
\hline
\textbf{Case}     & \textbf{Complexity}   & \textbf{Property}     & \textbf{Value} \\
\hline
Best Case         & $O(n)$                & Stable                & Yes            \\
Average Case      & $O(n^2)$              & In-place              & Yes            \\
Worst Case        & $O(n^2)$              & Adaptive              & Yes            \\
\hline
\noalign{\vskip 2pt}
Space Complexity  & $O(1)$ (in-place)     & Recursive             & No             \\
\hline
\end{tabular}
\end{center}

\section*{7. Python Code with Explanation}

\begin{lstlisting}[language=Python]
def insertion_sort(arr):
    # Traverse from 1 to len(arr)
    for i in range(1, len(arr)):
        key = arr[i]
        j = i - 1

        # Move elements greater than key to one position ahead
        while j >= 0 and arr[j] > key:
            arr[j + 1] = arr[j]
            j -= 1

        # Insert the key at the correct position
        arr[j + 1] = key

    return arr
\end{lstlisting}


\newpage
\section*{8. Important Notes on Conditions}

\subsection*{A. Difference Between \texttt{j >= 0} and \texttt{j > 0}}

\begin{itemize}
    \item \texttt{j >= 0} ensures that even the element at index \texttt{0} is compared and shifted if needed. It is the correct condition.
    \item \texttt{j > 0} skips the element at index \texttt{0}, which may result in incorrect sorting.
\end{itemize}

\begin{tcolorbox}[colback=white, colframe=black, title=Example]
Given \texttt{[3, 2]}:

\begin{itemize}
    \item With \texttt{j >= 0}: Result → \texttt{[2, 3]} (Correct)
    \item With \texttt{j > 0}: Result → \texttt{[3, 2]} (Incorrect)
\end{itemize}
\end{tcolorbox}

\vspace{1em}

\subsection*{B. How to Make Insertion Sort Unstable}

\begin{itemize}
    \item Insertion Sort is stable because it does not move equal elements.
    \item If you change the condition in the while loop from \texttt{arr[j] > key} to \texttt{arr[j] >= key}, equal elements can be shifted, reversing their original order.
\end{itemize}

\begin{lstlisting}[language=Python, caption=Unstable Version]
while j >= 0 and arr[j] >= key:  # Unstable: may move equal elements
    arr[j + 1] = arr[j]
    j -= 1
\end{lstlisting}

\begin{tcolorbox}[colback=white, colframe=black, title=Stability Impact]
\texttt{arr[j] > key} → Stable  
\newline
\texttt{arr[j] >= key} → Unstable
\end{tcolorbox}

\vspace{1em}

\subsection*{C. Summary of Conditions}

\begin{center}
\begin{tabular}{|c|p{8cm}|}
\hline
\textbf{Condition} & \textbf{Effect} \\
\hline
\texttt{j >= 0} & Correct — includes index 0 in \newline comparisons \\
\texttt{j > 0} & Incorrect — skips index 0, may miss minimum value \\
\texttt{arr[j] > key} & Stable — preserves order of equal \newline elements \\
\texttt{arr[j] >= key} & Unstable — may shift equal elements, reversing order \\
\hline
\end{tabular}
\end{center}


\section*{9. Making \texttt{arr[j] >= key} Stable}

\subsection*{Why it Becomes Unstable}

The condition \texttt{arr[j] >= key} causes instability because:
\begin{itemize}
    \item It shifts equal elements to the right.
    \item This breaks their original relative order.
\end{itemize}

\begin{tcolorbox}[colback=white, colframe=black, title=Stability Violation]
Using \texttt{arr[j] >= key} moves earlier equal elements after the current one, reversing their order.
\end{tcolorbox}

\vspace{1em}

\newpage
\subsection*{How to Make it Stable Even with \texttt{>=}}

We can track the original positions of the elements by pairing each value with its index.  
This allows us to break ties using original position, preserving stability.

\begin{lstlisting}[language=Python, caption=Stable Insertion Sort Using Index Tracking]
def stable_insertion_sort(arr):
    # Attach original indices
    arr_with_index = [(val, idx) for idx, val in enumerate(arr)]

    for i in range(1, len(arr_with_index)):
        key = arr_with_index[i]
        j = i - 1

        while j >= 0 and (
            arr_with_index[j][0] > key[0] or
            (arr_with_index[j][0] == key[0] and arr_with_index[j][1] > key[1])
        ):
            arr_with_index[j + 1] = arr_with_index[j]
            j -= 1

        arr_with_index[j + 1] = key

    # Extract values without indices
    return [val for val, idx in arr_with_index]
\end{lstlisting}

\vspace{0.5em}

\subsection*{Explanation}

\begin{itemize}
    \item Elements are stored as tuples: \texttt{(value, original\_index)}.
    \item If values are equal, we preserve their original order using the index.
    \item This approach works even when using \texttt{>=}.
\end{itemize}

\begin{tcolorbox}[colback=white, colframe=black, title=Result]
The sort remains stable even when \texttt{arr[j] >= key} is used.
\end{tcolorbox}

\section*{10. Step-by-Step Example with Index Tracking}

We apply \texttt{stable\_insertion\_sort} on the input:

\begin{tcolorbox}[colback=white, colframe=black, title=Input]
\texttt{[5a, 3, 5b, 2]}  
\textit{Here, 5a and 5b are equal values but come from different positions.}
\end{tcolorbox}

\subsection*{Initial Array with Index}
\[
[(5,0), (3,1), (5,2), (2,3)]
\]

---

\subsection*{Pass 1: i = 1 (key = (3,1))}

Compare with (5,0):

\[
(5 > 3) \Rightarrow \text{Shift (5,0) to right}
\]

\[
[(5,0), (5,0), (5,2), (2,3)] \Rightarrow \text{Insert (3,1) at index 0}
\]

\[
[(3,1), (5,0), (5,2), (2,3)]
\]

---

\subsection*{Pass 2: i = 2 (key = (5,2))}

Compare with (5,0):  
\[
(5 == 5 \text{ and } 0 < 2) \Rightarrow \text{Don't shift (5,0)}  
\Rightarrow \text{Insert (5,2) after (5,0)}
\]

\[
[(3,1), (5,0), (5,2), (2,3)]
\]

---

\subsection*{Pass 3: i = 3 (key = (2,3))}

Compare with (5,2):  
\[
(5 > 2) \Rightarrow \text{Shift}
\]

Compare with (5,0):  
\[
(5 > 2) \Rightarrow \text{Shift}
\]

Compare with (3,1):  
\[
(3 > 2) \Rightarrow \text{Shift}
\]

Insert (2,3) at index 0:

\[
[(2,3), (3,1), (5,0), (5,2)]
\]

---

\subsection*{Final Sorted Output (Without Index)}

\[
[2, 3, 5a, 5b]
\]

\begin{tcolorbox}[colback=white, colframe=black, title=Stability Verified]
Even with \texttt{arr[j] >= key}, the relative order of equal elements (5a before 5b) is preserved.
\end{tcolorbox}


\end{document}
