\documentclass[14pt]{extarticle}
\usepackage[margin=1in]{geometry}
\usepackage{amsmath}
\usepackage{enumitem}
\usepackage{titlesec}
\usepackage{algorithm}
\usepackage{algpseudocode}
\usepackage{hyperref} % optional, for clickable TOC/list
\usepackage{caption}
\usepackage{float}
\titleformat{\section}{\normalfont\Large\bfseries}{\thesection}{1em}{}

\usepackage{listings}
\usepackage{xcolor}

% Place this in your preamble
\lstdefinestyle{python}{
  language=Python,
  basicstyle=\ttfamily\small,
  keywordstyle=\color{blue},
  commentstyle=\color{gray},
  stringstyle=\color{orange},
  showstringspaces=false,
  numbers=left,
  numberstyle=\tiny\color{gray},
  stepnumber=1,
  breaklines=true,
  frame=single,
  tabsize=4,
  captionpos=b
}

\lstdefinestyle{cpp}{
  language=C++,
  basicstyle=\ttfamily\small,
  keywordstyle=\color{blue},
  commentstyle=\color{gray},
  stringstyle=\color{orange},
  showstringspaces=false,
  numbers=left,
  numberstyle=\tiny\color{gray},
  stepnumber=1,
  breaklines=true,
  frame=single,
  tabsize=4,
  captionpos=b
}


\title{The Knapsack Problem: Theory Overview}
\author{Varun Kumar}
\date{\today}

\begin{document}
\maketitle

\newpage
\tableofcontents
\newpage
\lstlistoflistings
\listofalgorithms
\newpage

%%%%%%%%  Knapsack Problem Overview %%%%%%%%%%
\section{Introduction}

The Knapsack Problem is a fundamental problem in combinatorial optimization. It models a situation where a set of items, each with a given weight and value, must be selected to include in a knapsack of limited capacity such that the total value is maximized without exceeding the weight constraint. 

This problem arises in many real-world scenarios such as resource allocation, budgeting, cargo loading, and decision-making under constraints. There are two common variants:

\begin{itemize}[label=\textbullet]
    \item \textbf{Fractional Knapsack:} Items can be divided into smaller parts.
    \item \textbf{0/1 Knapsack:} Each item is either fully taken or not taken at all.
\end{itemize}

While the fractional version can be solved efficiently using a greedy strategy, the 0/1 version requires dynamic programming to guarantee an optimal solution.
 

\section{Problem Statement}

Given $n$ items, each with a value $v_i$ and weight $w_i$, and a knapsack with capacity $W$, the 
goal is to choose a subset of the items to maximize the total value without exceeding the weight capacity.

\begin{itemize}[label=\textbullet]
    \item Maximize: $\sum v_i x_i$
    \item Subject to: $\sum w_i x_i \leq W$
    \item Where $x_i \in \{0, 1\}$ for the 0/1 Knapsack
\end{itemize}
%%%%%%%%%%%%%%%%%%%%%%%%%%%%%%%%%%%%%%%%%%%%%%%%%%%%%%%%%%%%%%%%%%%%%%%%%%%%%%%%%%%%%%%%
\section{Fractional Knapsack (Greedy Approach)}

\begin{itemize}[label=\textbullet]
    \item Items can be broken into fractions.
    \item Sort items based on the value-to-weight ratio $\frac{v_i}{w_i}$ in descending order.
    \item Pick the item with the highest ratio until the knapsack is full.
    \item Time Complexity: $\mathcal{O}(n \log n)$
\end{itemize}

\textbf{Note:} This method gives the optimal solution for the fractional variant only.
%%%%%%%%%%%%%%%%%%%%%%%%%%%%%%%%%%%%%%%%%%%%%%%%%%%%%%%%%%%%%%%%%%%%%%%%%%%%%%%%%%%
\subsection{Problem Statement}

You are given $n$ items. Each item has:

\begin{itemize}[label=\textbullet]
    \item Value $v_i$
    \item Weight $w_i$
\end{itemize}

You are also given a knapsack with capacity $W$. You can take \textbf{fractions} of items.

\textbf{Goal:} Maximize total value such that the total weight does not exceed $W$.
%%%%%%%%%%%%%%%%%%%%%%%%%%%%%%%%%%%%%%%%%%%%%%%%%%%%%%%%%%%%%%%%%%%%%%%%%%%%%%%%
\subsection{Greedy Strategy}

To solve the Fractional Knapsack problem, use the following greedy approach:

\begin{enumerate}
    \item Compute the value-to-weight ratio $\frac{v_i}{w_i}$ for each item.
    \item Sort items by this ratio in \textbf{descending} order.
    \item Initialize \texttt{currentWeight = 0} and \texttt{totalValue = 0}.
    \item For each item:
    \begin{itemize}
        \item If the item fits fully (\texttt{$w_i \leq$ remaining capacity}), take the whole item.
        \item Else, take the fraction of the item that fits.
    \end{itemize}
    \item Stop when the knapsack is full.
\end{enumerate}
%%%%%%%%%%%%%%%%%%%%%%%%%%%%%%%%%%%%%%%%%%%%%%%%%%%%%%%%%%%%%%%%%%%%%%%%%%%%%%%%%%%
\subsection{Worked Example}

\subsubsection*{Given:}

\begin{center}
\begin{tabular}{|c|c|c|}
\hline
\textbf{Item} & \textbf{Value ($v_i$)} & \textbf{Weight ($w_i$)} \\
\hline
1 & 60 & 10 \\
2 & 100 & 20 \\
3 & 120 & 30 \\
\hline
\end{tabular}
\end{center}

Knapsack capacity: $W = 50$
%%%%%%%%%%%%%%%%%%%%%%%%%%%%%%%%%%%%%%%%%%%%%%%%%%%%%%%%%%%%%%%%%%%%%%%%%%%%%%%%%%
\subsubsection*{Step 1: Compute Value/Weight Ratios}

\begin{center}
\begin{tabular}{|c|c|}
\hline
\textbf{Item} & $\frac{v_i}{w_i}$ \\
\hline
1 & 60 / 10 = 6 \\
2 & 100 / 20 = 5 \\
3 & 120 / 30 = 4 \\
\hline
\end{tabular}
\end{center}

Order of selection based on ratio: \textbf{Item 1 $\rightarrow$ Item 2 $\rightarrow$ Item 3}
%%%%%%%%%%%%%%%%%%%%%%%%%%%%%%%%%%%%%%%%%%%%%%%%%%%%%%%%%%%%%%%%%%%%%%%%%%%%%%%%%%%%%
\subsubsection*{Step 2: Pick Items Greedily}

\begin{itemize}
    \item Take all of Item 1 (10kg): Value += 60, remaining capacity = 40
    \item Take all of Item 2 (20kg): Value += 100, remaining capacity = 20
    \item Take $\frac{2}{3}$ of Item 3 (20kg of 30kg): \\
          Value += $120 \times \frac{20}{30} = 80$
\end{itemize}
%%%%%%%%%%%%%%%%%%%%%%%%%%%%%%%%%%%%%%%%%%%%%%%%%%%%%%%%%%%%%%%%%%%%%%%%%%%%%%%%%%%%%%%
\subsection{Final Answer}

\begin{itemize}
    \item Total value obtained = $60 + 100 + 80 = 240$
    \item Total weight used = $10 + 20 + 20 = 50$ (Knapsack is full)
\end{itemize}

%%%%%%%%%% Fractional Knapsack Algorithm Pseudo Code%%%%%%%%%%
\begin{algorithm}
\caption{Fractional Knapsack}
\begin{algorithmic}[1]
\Require A list of $n$ items with value $v_i$ and weight $w_i$, capacity $W$
\Ensure Maximum total value without exceeding capacity
\State Compute $\frac{v_i}{w_i}$ for each item
\State Sort items by decreasing $\frac{v_i}{w_i}$
\State $totalValue \gets 0$
\State $currentWeight \gets 0$
\For{each item $i$ in sorted order}
    \If{$currentWeight + w_i \leq W$}
        \State Take the whole item
        \State $currentWeight \gets currentWeight + w_i$
        \State $totalValue \gets totalValue + v_i$
    \Else
        \State Take fraction $(W - currentWeight)/w_i$ of item $i$
        \State $totalValue \gets totalValue + v_i \times \frac{W - currentWeight}{w_i}$
        \State \textbf{break}
    \EndIf
\EndFor
\State \Return $totalValue$
\end{algorithmic}
\end{algorithm}

%%%%%%%%%% Python Code for Fractional Knapsack %%%%%%%%%%
\newpage
\section{Fractional Knapsack Code Implementation}
\subsection{Python Implementation}
\begin{lstlisting}[style=python, caption={Fractional Knapsack in Python}]
def fractional_knapsack(values, weights, capacity):
    items = [(v, w, v/w) for v, w in zip(values, weights)]
    items.sort(key=lambda x: x[2], reverse=True)

    total_value = 0.0
    for value, weight, ratio in items:
        if capacity >= weight:
            total_value += value
            capacity -= weight
        else:
            total_value += ratio * capacity
            break
    return total_value
\end{lstlisting}


\subsection{Line-by-Line Breakdown}

\paragraph{Line 1}
\begin{verbatim}
def fractional_knapsack(values, weights, capacity):
\end{verbatim}
\begin{itemize}
    \item Defines a function with:
    \begin{itemize}
        \item \texttt{values} – list of item values
        \item \texttt{weights} – list of item weights
        \item \texttt{capacity} – total capacity of the knapsack
    \end{itemize}
\end{itemize}

\paragraph{Line 2}
\begin{verbatim}
items = [(v, w, v/w) for v, w in zip(values, weights)]
\end{verbatim}
\begin{itemize}
    \item Creates a list of tuples \texttt{(value, weight, ratio)}.
    \item \texttt{zip(values, weights)} combines both lists.
    \item \texttt{v/w} computes the value-to-weight ratio.
\end{itemize}

\textbf{Example:}
\begin{verbatim}
values  = [60, 100, 120]
weights = [10, 20, 30]
# → items = [(60, 10, 6.0), (100, 20, 5.0), (120, 30, 4.0)]
\end{verbatim}

\paragraph{Line 3}
\begin{verbatim}
items.sort(key=lambda x: x[2], reverse=True)
\end{verbatim}
\begin{itemize}
    \item Sorts items in descending order by value-to-weight ratio.
    \item Highest "value per kg" items come first.
\end{itemize}

\paragraph{Line 4}
\begin{verbatim}
total_value = 0.0
\end{verbatim}
\begin{itemize}
    \item Initializes the total value of items added to the knapsack.
\end{itemize}

\paragraph{Line 5–11 (Main loop)}
\begin{verbatim}
for value, weight, ratio in items:
\end{verbatim}
\begin{itemize}
    \item Iterates over each sorted item.
\end{itemize}

\begin{verbatim}
    if capacity >= weight:
        total_value += value
        capacity -= weight
\end{verbatim}
\begin{itemize}
    \item If the full item fits, take it completely.
    \item Add its value, and subtract its weight from capacity.
\end{itemize}

\begin{verbatim}
    else:
        total_value += ratio * capacity
        break
\end{verbatim}
\begin{itemize}
    \item If the item doesn't fully fit:
    \begin{itemize}
        \item Take a fraction that fits.
        \item Add proportional value.
        \item Break the loop as knapsack is full.
    \end{itemize}
\end{itemize}

\paragraph{Line 12}
\begin{verbatim}
return total_value
\end{verbatim}
\begin{itemize}
    \item Return the maximum value that can be carried in the knapsack.
\end{itemize}

\subsection*{Example Call}
\begin{verbatim}
fractional_knapsack([60, 100, 120], [10, 20, 30], 50)
\end{verbatim}

\textbf{Output:}
\begin{verbatim}
240.0
\end{verbatim}

\subsection*{Why?}
\begin{itemize}
    \item Take all of item 1: 60 (10kg)
    \item Take all of item 2: 100 (20kg)
    \item Take 2/3 of item 3: $120 \times \frac{2}{3} = 80$
\end{itemize}

\textbf{Total} = $60 + 100 + 80 = 240$

\subsection*{Example Input and Output}

\begin{verbatim}
# Input
values  = [60, 100, 120]
weights = [10, 20, 30]
capacity = 50

# Function call
result = fractional_knapsack(values, weights, capacity)

# Output
print(result)  # Output: 240.0
\end{verbatim}

\newpage
\subsection{C++ Implementation}
%%%%%%%%%%% C++ Code for Fractional Knapsack %%%%%%%%%%
\begin{lstlisting}[style=cpp, caption={Fractional Knapsack in C++}]
#include <iostream>
#include <vector>
#include <algorithm>
using namespace std;

struct Item {
    int value, weight;
};

bool compare(Item a, Item b) {
    double r1 = (double)a.value / a.weight;
    double r2 = (double)b.value / b.weight;
    return r1 > r2;
}

double fractionalKnapsack(int W, vector<Item> &items) {
    sort(items.begin(), items.end(), compare);
    double totalValue = 0.0;

    for (Item &item : items) {
        if (W >= item.weight) {
            W -= item.weight;
            totalValue += item.value;
        } else {
            totalValue += item.value * ((double)W / item.weight);
            break;
        }
    }

    return totalValue;
}
\end{lstlisting}


\newpage
%%%%%%%%% 0/1 Knapsack Problem %%%%%%%%%%
% The 0/1 Knapsack Problem is a more constrained version where items cannot be divided. 
% Each item is either included in the knapsack or excluded, leading to a combinatorial explosion of possibilities.
\section{0/1 Knapsack (Dynamic Programming)}

\begin{itemize}[label=\textbullet]
    \item Items cannot be broken; each item is either taken or not.
    \item Define a DP table: $dp[i][w]$ = maximum value for first $i$ items with weight limit $w$.
    \item Recurrence relation:
    \[
        dp[i][w] =
        \begin{cases}
            dp[i-1][w], & \text{if } w_i > w \\
            \max(dp[i-1][w], dp[i-1][w - w_i] + v_i), & \text{otherwise}
        \end{cases}
    \]
    \item Time Complexity: $\mathcal{O}(nW)$
\end{itemize}

\textbf{Note:} This method guarantees the optimal solution for the 0/1 Knapsack.


%%%%%%%%%%%%%%%%%% Worked Example %%%%%%%%%%%%%%%%%%
\subsection{0/1 Knapsack Problem (Dynamic Programming)}

\subsection*{Problem Statement}

Given $n$ items with:
\begin{itemize}
    \item Values: $v_i$
    \item Weights: $w_i$
    \item A knapsack of capacity $W$
\end{itemize}

Determine the maximum total value that can be obtained by selecting a subset of items such that:
\begin{itemize}
    \item Each item is either fully included or excluded (no fractions)
    \item Total weight $\leq W$
\end{itemize}

\subsection*{Example}

\textbf{Given:}

\begin{center}
\begin{tabular}{|c|c|c|}
\hline
\textbf{Item} & \textbf{Value ($v_i$)} & \textbf{Weight ($w_i$)} \\
\hline
1 & 60 & 1 \\
2 & 100 & 2 \\
3 & 120 & 3 \\
\hline
\end{tabular}
\end{center}

Knapsack Capacity: $W = 5$

\subsection*{Approach: Dynamic Programming}

Let $dp[i][w]$ be the maximum value that can be obtained by considering the first $i$ items with capacity $w$.

We use the recurrence:

\[
dp[i][w] =
\begin{cases}
    dp[i-1][w] & \text{if } w_i > w \\
    \max(dp[i-1][w], dp[i-1][w - w_i] + v_i) & \text{otherwise}
\end{cases}
\]

---

\subsection*{DP Table Construction}

Let’s build a table for $dp[0\,..\,3][0\,..\,5]$

\begin{center}
\begin{tabular}{|c|c|c|c|c|c|c|}
\hline
$i \backslash w$ & 0 & 1 & 2 & 3 & 4 & 5 \\
\hline
0 & 0 & 0 & 0 & 0 & 0 & 0 \\
1 & 0 & 60 & 60 & 60 & 60 & 60 \\
2 & 0 & 60 & 100 & 160 & 160 & 160 \\
3 & 0 & 60 & 100 & 160 & 180 & 220 \\
\hline
\end{tabular}
\end{center}

\textbf{Explanation:}
\begin{itemize}
    \item Row $i$ considers first $i$ items
    \item Column $w$ represents capacity
    \item Final answer is at $dp[3][5] = \boxed{220}$
\end{itemize}

\subsection*{Result}

The maximum value we can carry in the knapsack is:

\[
\boxed{220}
\]

\subsection*{Items Selected}

Using backtracking from the table:
\begin{itemize}
    \item Item 3 is included (weight 3, value 120)
    \item Remaining capacity = 2
    \item Item 2 is included (weight 2, value 100)
\end{itemize}

\textbf{Final Selection:} Item 2 and Item 3

\textbf{Total Weight:} $2 + 3 = 5$, \textbf{Total Value:} $100 + 120 = \boxed{220}$

%%%%%%%%%%%%%%%%%% Construction of DP Table %%%%%%%%%%%%%%%%%%
\subsection{Step-by-Step Construction of DP Table}

We define a DP table \texttt{dp[i][w]} where:
\begin{itemize}
    \item $i$ = number of items considered
    \item $w$ = current knapsack capacity (from $0$ to $W$)
    \item \texttt{dp[i][w]} stores the \textbf{maximum value} using first $i$ items and capacity $w$
\end{itemize}

\textbf{Initialization:}

\begin{itemize}
    \item For all $w$: \quad \texttt{dp[0][w] = 0} \quad (No items → 0 value)
    \item For all $i$: \quad \texttt{dp[i][0] = 0} \quad (0 capacity → 0 value)
\end{itemize}

\textbf{Input:}

\begin{tabular}{|c|c|c|}
\hline
Item $i$ & Value $v_i$ & Weight $w_i$ \\
\hline
1 & 60 & 1 \\
2 & 100 & 2 \\
3 & 120 & 3 \\
\hline
\end{tabular}
\quad \quad
Capacity $W = 5$

\bigskip

\textbf{Filling the table:}

We use this recurrence:
\[
dp[i][w] =
\begin{cases}
    dp[i-1][w] & \text{if } w_i > w \quad \text{(Item can't fit)} \\
    \max(dp[i-1][w], dp[i-1][w - w_i] + v_i) & \text{otherwise}
\end{cases}
\]

---

\paragraph{Row 1 (Item 1: value=60, weight=1):}

\begin{itemize}
    \item $w = 0$: can't include → $dp[1][0] = 0$
    \item $w = 1$: can include → $dp[1][1] = \max(0, 0 + 60) = 60$
    \item $w = 2$: $dp[1][2] = \max(0, 0 + 60) = 60$
    \item $w = 3$: $dp[1][3] = \max(0, 0 + 60) = 60$
    \item $w = 4$: $dp[1][4] = \max(0, 0 + 60) = 60$
    \item $w = 5$: $dp[1][5] = \max(0, 0 + 60) = 60$
\end{itemize}

\paragraph{Row 2 (Item 2: value=100, weight=2):}

\begin{itemize}
    \item $w = 0,1$: can't include → same as above
    \item $w = 2$: $dp[2][2] = \max(60, 0 + 100) = 100$
    \item $w = 3$: $dp[2][3] = \max(60, 60 + 100) = 160$
    \item $w = 4$: $dp[2][4] = \max(60, 60 + 100) = 160$
    \item $w = 5$: $dp[2][5] = \max(60, 60 + 100) = 160$
\end{itemize}

\paragraph{Row 3 (Item 3: value=120, weight=3):}

\begin{itemize}
    \item $w = 0,1,2$: can't include → same as above
    \item $w = 3$: $dp[3][3] = \max(160, 0 + 120) = 160$
    \item $w = 4$: $dp[3][4] = \max(160, 60 + 120) = 180$
    \item $w = 5$: $dp[3][5] = \max(160, 100 + 120) = \boxed{220}$
\end{itemize}

---

\subsection*{Final DP Table}

\begin{center}
\renewcommand{\arraystretch}{1.3}
\begin{tabular}{|c|c|c|c|c|c|c|}
\hline
$i \backslash w$ & 0 & 1 & 2 & 3 & 4 & 5 \\
\hline
0 & 0 & 0 & 0 & 0 & 0 & 0 \\
1 & 0 & 60 & 60 & 60 & 60 & 60 \\
2 & 0 & 60 & 100 & 160 & 160 & 160 \\
3 & 0 & 60 & 100 & 160 & 180 & \textbf{220} \\
\hline
\end{tabular}
\end{center}

\textbf{Final Answer: } \quad Maximum value = $\boxed{220}$


%%%%%%%%%%%% Code Implementation of 0/1 Knapsack %%%%%%%%%%%%%
\section{Theoretical Logic Behind 0/1 Knapsack Implementation}

The 0/1 Knapsack problem is a classic example of \textbf{Dynamic Programming}, where the optimal solution of a problem depends on the optimal solutions of its subproblems. The idea is to build a table that stores the maximum value for each subproblem defined by:

\begin{itemize}
    \item The number of items considered so far $(i)$
    \item The remaining capacity of the knapsack $(w)$
\end{itemize}

Let $dp[i][w]$ represent the maximum value achievable by considering the first $i$ items with a knapsack capacity $w$.

\subsection*{Recurrence Relation}

\[
dp[i][w] =
\begin{cases}
    dp[i-1][w] & \text{if } w_i > w \quad \text{(Item doesn't fit)} \\
    \max(dp[i-1][w], dp[i-1][w - w_i] + v_i) & \text{otherwise}
\end{cases}
\]

Where:
\begin{itemize}
    \item $v_i$ is the value of the $i^\text{th}$ item
    \item $w_i$ is the weight of the $i^\text{th}$ item
\end{itemize}

\subsection*{Base Case Initialization}

\begin{itemize}
    \item $dp[0][w] = 0$ for all $w$ (no items means no value)
    \item $dp[i][0] = 0$ for all $i$ (zero capacity means no value)
\end{itemize}

\subsection*{Bottom-Up Construction}

We iterate over all items and capacities to fill the DP table. For each item and capacity:

\begin{itemize}
    \item If the item weight is more than the current capacity, we can't include it, so we inherit the value from above.
    \item Otherwise, we decide whether to include the item or not by taking the maximum of:
    \begin{itemize}
        \item Excluding the item: $dp[i-1][w]$
        \item Including the item: $dp[i-1][w - w_i] + v_i$
    \end{itemize}
\end{itemize}

\subsection*{Final Answer}

The final result is stored in $dp[n][W]$, where:
\begin{itemize}
    \item $n$ is the total number of items
    \item $W$ is the maximum capacity of the knapsack
\end{itemize}

This value represents the \textbf{maximum total value} that can be achieved without exceeding the 
knapsack's capacity using the given items.

\newpage
\subsection{Python Implementation}
\begin{lstlisting}[style=python, caption={0/1 Knapsack in Python}]
def knapsack(values, weights, capacity):
    n = len(values)
    # Initialize DP table with 0
    dp = [[0 for _ in range(capacity + 1)] for _ in range(n + 1)]

    # Fill the table
    for i in range(1, n + 1):
        for w in range(0, capacity + 1):
            if weights[i - 1] > w:
                dp[i][w] = dp[i - 1][w]  # Can't include item
            else:
                # Max of including or excluding the item
                dp[i][w] = max(dp[i - 1][w],
                               dp[i - 1][w - weights[i - 1]] + values[i - 1])

    return dp[n][capacity]
\end{lstlisting}

\begin{verbatim}
    values = [60, 100, 120]
    weights = [1, 2, 3]
    capacity = 5

    result = knapsack(values, weights, capacity)
    print(result)  # Output: 220
\end{verbatim}

\newpage
\subsection{C++ Implementation}
\begin{lstlisting}[style=cpp, caption={0/1 Knapsack in C++}]
#include <iostream>
#include <vector>
using namespace std;

int knapsack(vector<int>& values, vector<int>& weights, int capacity) {
    int n = values.size();
    vector<vector<int>> dp(n + 1,vector<int>(capacity + 1, 0));
    for (int i = 1; i <= n; ++i) {
        for (int w = 0; w <= capacity; ++w) {
            if (weights[i - 1] > w) {
                dp[i][w] = dp[i - 1][w]; // Cannot include item
            } else {
                dp[i][w] = max(
                    dp[i - 1][w],
                    dp[i-1][w - weights[i-1]] + values[i-1]
                );
            }
        }
    }
    return dp[n][capacity];
}

int main() {
    int n, capacity;
    cout << "Enter number of items: ";
    cin >> n;

    vector<int> values(n), weights(n);

    cout << "Enter values:\n";
    for (int i = 0; i < n; ++i) cin >> values[i];
    cout << "Enter weights:\n";
    for (int i = 0; i < n; ++i) cin >> weights[i];
    cout << "Enter knapsack capacity: ";
    cin >> capacity;
    int result = knapsack(values, weights, capacity);
    cout << "Maximum value: " << result << endl;

    return 0;
}
\end{lstlisting}

\newpage
%%%%%%%%%% Key Points %%%%%%%%%%
\section{Key Points to Remember}

\begin{enumerate}
    \item Knapsack problems involve selecting items to maximize total value under a weight constraint.
    \item 0/1 Knapsack: each item is either taken fully or not at all.
    \item Fractional Knapsack: items can be broken and partially taken.
    \item 0/1 Knapsack is solved using Dynamic Programming.
    \item Fractional Knapsack is solved using Greedy strategy.
    \item 0/1 Knapsack does not follow greedy choice property.
    \item Fractional Knapsack follows both greedy choice and optimal substructure properties.
    \item Time complexity of 0/1 Knapsack: $\mathcal{O}(nW)$, where $n$ = items, $W$ = capacity.
    \item Time complexity of Fractional Knapsack: $\mathcal{O}(n \log n)$ (due to sorting).
    \item In 0/1 Knapsack, a DP table is built based on item count and capacity.
    \item Backtracking the DP table can give the list of selected items.
    \item Fractional Knapsack always yields the optimal solution.
    \item 0/1 Knapsack may require approximation if constraints are large.
    \item Space optimization is possible using 1D arrays in 0/1 Knapsack.
    \item Knapsack is a classic NP-complete problem (for 0/1 case).
\end{enumerate}

\newpage
%%%%%%%%%%% Real World Applications %%%%%%%%%%
\section{Real-World Applications}

\begin{itemize}
    \item \textbf{Cargo Loading:} Selecting goods to load onto a truck/ship with weight limits.
    \item \textbf{Budget Allocation:} Choosing the best set of projects under a limited budget.
    \item \textbf{Resource Scheduling:} Assigning limited compute or memory to jobs for maximum value.
    \item \textbf{Investment Planning:} Selecting the best combination of assets under risk/capital constraints.
    \item \textbf{Time Management:} Choosing the most rewarding activities within limited time.
    \item \textbf{Memory Management in OS:} Efficiently selecting data blocks to keep in memory.
    \item \textbf{Marketing Campaigns:} Allocating limited resources (ad budget, team) to most impactful actions.
    \item \textbf{Cloud Computing:} VM placement and workload optimization under physical constraints.
    \item \textbf{Logistics Optimization:} Selecting orders or shipments based on value-to-weight ratio.
    \item \textbf{Resource-constrained AI Agents:} Selecting actions under energy/time restrictions.
\end{itemize}


\newpage
%%%%%%%%%%%%%% Final %%%%%%%%%%%%%%%%
\section{Comparison}

\begin{tabular}{|l|p{4.2cm}|p{4.2cm}|}
\hline
\textbf{Feature} & \textbf{Fractional\newline Knapsack} & \textbf{0/1 Knapsack} \\
\hline
Problem Type & Optimization with fractions & Combinatorial subset selection \\
Item Division & Allowed (can take part of item) & Not Allowed (take whole or none) \\
Approach & Greedy & Dynamic Programming \\
Guarantees Optimality & Only for fractional case & Yes, for 0/1 selection \\
Time Complexity & $\mathcal{O}(n \log n)$ & $\mathcal{O}(nW)$ \\
\hline
\end{tabular}

%%%%%%%%%%%%%%%% conclusion %%%%%%%%%%
\section{Conclusion}

The Knapsack problem illustrates a fundamental trade-off between resource usage and value optimization. 
It exists in two major forms:

\begin{itemize}
    \item The \textbf{0/1 Knapsack}, which is discrete and requires Dynamic Programming due to its combinatorial nature.
    \item The \textbf{Fractional Knapsack}, which is continuous and optimally solvable using a Greedy approach.
\end{itemize}

These problems are widely applicable in fields like operations research, finance, computer science, and logistics. 
A deep understanding of Knapsack variants not only helps in mastering algorithmic techniques but also equips one to 
model and solve many real-world optimization problems effectively.


\end{document}
