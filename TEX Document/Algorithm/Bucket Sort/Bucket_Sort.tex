\documentclass[14pt]{extarticle}
\usepackage[margin=1in]{geometry}
\usepackage{amsmath}
\usepackage{listings}
\usepackage{xcolor}
\usepackage{graphicx}
\usepackage{tcolorbox}
\usepackage{multicol}
\usepackage{titlesec}
\usepackage{enumitem}
\usepackage{makecell}

\definecolor{codegray}{gray}{0.95}
\lstset{
    basicstyle=\ttfamily\small,
    frame=single,
    breaklines=true,
    columns=fullflexible,
}

\title{\textbf{Bucket Sort}}
\author{Varun Kumar}
\date{July 5, 2025}

\begin{document}

\maketitle

\section*{1. Logic}

Bucket Sort distributes the elements into a number of buckets. Each bucket is then sorted individually (often using Insertion Sort or another algorithm), and finally, all the buckets are concatenated to form the sorted array.

\begin{tcolorbox}[
  colback=white,
  colframe=black,
  title=Key Idea
]
Distribute elements into several buckets, sort each bucket, and concatenate them to get the final sorted array.
\end{tcolorbox}

\section*{2. Number of Comparisons and Shifts}

Let $n$ be the number of elements.

\subsection*{Worst Case (All in One Bucket)}

\begin{itemize}
    \item Comparisons: $O(n^2)$ (if sorted using Insertion Sort inside each bucket)
    \item Shifts: Depends on internal sort
\end{itemize}

\subsection*{Best Case (Uniform Distribution)}

\begin{itemize}
    \item Comparisons: $O(n)$
    \item Shifts: $O(n)$
\end{itemize}

\section*{3. Optimal Behavior}

Bucket sort works best when:
\begin{itemize}
    \item Input is uniformly distributed over a known range.
    \item Number of buckets is well chosen.
\end{itemize}

\section*{4. Pseudocode}

\begin{lstlisting}[language=Python]
function bucketSort(arr):
    n = length(arr)
    create n empty buckets

    for each element in arr:
        insert it into the appropriate bucket

    for each bucket:
        sort the bucket

    concatenate all buckets into arr
\end{lstlisting}

\begin{tcolorbox}[
  colback=white,
  colframe=black,
  title=Stability Note
]
Bucket Sort is not inherently stable. Stability depends on the sorting algorithm used inside each bucket.
\end{tcolorbox}

\section*{5. Example Walkthrough}

Given: \texttt{[0.42, 0.32, 0.23, 0.52, 0.25, 0.47]}

\subsection*{Step 1: Bucket Distribution}

\begin{itemize}
    \item Bucket 0: [0.23, 0.25]
    \item Bucket 1: [0.32]
    \item Bucket 2: []
    \item Bucket 3: [0.42, 0.47]
    \item Bucket 4: [0.52]
\end{itemize}

\subsection*{Step 2: Sort Each Bucket}

\begin{itemize}
    \item Bucket 0: [0.23, 0.25]
    \item Bucket 3: [0.42, 0.47]
\end{itemize}

\subsection*{Step 3: Concatenate}

\texttt{[0.23, 0.25, 0.32, 0.42, 0.47, 0.52]}

\section*{6. Python Code with Explanation}

\begin{lstlisting}[language=Python]
def bucket_sort(arr):
    n = len(arr)
    if n == 0:
        return arr

    # Create n empty buckets
    buckets = [[] for _ in range(n)]

    # Insert elements into buckets
    for num in arr:
        index = int(n * num)
        buckets[index].append(num)

    # Sort each bucket
    for bucket in buckets:
        bucket.sort()

    # Concatenate all buckets
    sorted_arr = []
    for bucket in buckets:
        sorted_arr.extend(bucket)

    return sorted_arr
\end{lstlisting}

\section*{7. Is Bucket Sort Stable?}

\begin{tcolorbox}[
  colback=white,
  colframe=black,
  title=Stability of Bucket Sort
]
Bucket Sort is \textbf{not inherently stable}. Its stability depends on the sorting algorithm used inside each bucket.

\medskip

\textbf{Explanation:} If a \textit{stable} sorting algorithm (like Insertion Sort) is used within the buckets, then Bucket Sort will be stable. However, if an \textit{unstable} sort is used, then the overall algorithm will also be unstable.
\end{tcolorbox}

\section*{8. How Insertion Sort Ensures Stability}

Insertion Sort compares elements using the \texttt{>} operator. When two elements are equal, it does \textbf{not swap} or move them unnecessarily. This means their original relative order is preserved.

\medskip

For example, consider the list:
\[
[ (4, \text{A}), (4, \text{B}), (3, \text{C}) ]
\]

After sorting:
\[
[ (3, \text{C}), (4, \text{A}), (4, \text{B}) ]
\]

Elements with the same key \texttt{4} (\texttt{A} and \texttt{B}) remain in the same order as they appeared in the input.

\medskip

\textbf{Therefore, if Insertion Sort is used inside each bucket, the overall Bucket Sort becomes stable.}

\newpage
\section*{9. Is Bucket Sort In-place?}

\textbf{Answer:} \textbf{No, Bucket Sort is not in-place.}

\medskip

\textbf{Explanation:} Bucket Sort creates multiple auxiliary lists (buckets) to store elements during the sorting process. These additional data structures require extra memory proportional to the number of input elements and buckets.

\begin{itemize}
    \item It uses $O(n + k)$ extra space, where $k$ is the number of buckets.
    \item Since it does not sort within the original array without extra space, it is not considered in-place.
\end{itemize}


\section*{10. Can Bucket Sort Be Made In-place?}

\textbf{Answer:} \textbf{Theoretically possible, but not practical.}

\medskip

\textbf{Explanation:} Bucket Sort requires auxiliary space to create multiple buckets. To make it in-place, we would need to:
\begin{itemize}
    \item Divide the input array into segments (buckets) without extra space.
    \item Sort each segment in-place.
    \item Carefully rearrange elements to simulate bucket behavior.
\end{itemize}

However, this approach:
\begin{itemize}
    \item Complicates implementation.
    \item Risks losing the linear time benefit.
    \item Often requires pointer juggling or index remapping.
\end{itemize}

\textbf{Conclusion:} Although an in-place variant is possible with complex logic and trade-offs, the standard Bucket Sort is not in-place and is preferred with auxiliary space for simplicity and speed.



\newpage
\section*{11. Python Code: Bucket Sort using Insertion Sort}

\begin{lstlisting}[language=Python]
# Insertion Sort function for individual buckets
def insertion_sort(bucket):
    for i in range(1, len(bucket)):
        key = bucket[i]
        j = i - 1
        while j >= 0 and bucket[j] > key:
            bucket[j + 1] = bucket[j]
            j -= 1
        bucket[j + 1] = key

# Bucket Sort using Insertion Sort in each bucket
def bucket_sort(arr):
    n = len(arr)
    if n == 0:
        return arr

    # Create n empty buckets
    buckets = [[] for _ in range(n)]

    # Distribute elements into appropriate buckets
    for num in arr:
        index = int(n * num)  # Assumes input in range [0, 1)
        buckets[index].append(num)

    # Sort each bucket using insertion sort
    for bucket in buckets:
        insertion_sort(bucket)

    # Concatenate all sorted buckets
    sorted_arr = []
    for bucket in buckets:
        sorted_arr.extend(bucket)

    return sorted_arr
\end{lstlisting}

\newpage
\section*{12. Time \& Space Complexity and Its Properties}

\begin{center}
\begin{tabular}{|l|l||l|l|}
\hline
\textbf{Case}     & \textbf{Complexity}       & \textbf{Property}     & \textbf{Value} \\
\hline
Best Case         & $O(n + k)$                & Stable                & Yes (if insertion sort) \\
Average Case      & $O(n + k)$                & In-place              & No                     \\
Worst Case        & $O(n^2)$                  & Adaptive              & No                     \\
\hline
\noalign{\vskip 2pt}
Space Complexity  & $O(n + k)$ (extra space)  & Recursive             & No                     \\
\hline
\end{tabular}
\end{center}

\section*{13. Final Summary (10 Key Points)}

\begin{enumerate}
    \item Bucket Sort is a \textbf{distribution-based} sorting algorithm.
    \item It divides the input into multiple \textbf{buckets} and sorts each bucket individually.
    \item Ideal for inputs that are \textbf{uniformly distributed} over a known range.
    \item Each bucket can be sorted using any sorting algorithm (e.g., Insertion Sort).
    \item The overall time complexity is $O(n + k)$ in the average and best cases.
    \item Worst-case time is $O(n^2)$ when all elements fall into the same bucket.
    \item \textbf{Stability} depends on the sorting method used inside each bucket.
    \item It is \textbf{not in-place} due to additional memory for buckets.
    \item It is \textbf{not adaptive} — does not take advantage of existing order.
    \item Bucket Sort is efficient for large inputs with uniform distribution, but not general-purpose.
\end{enumerate}


\end{document}
