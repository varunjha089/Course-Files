\documentclass[14pt]{extarticle}
\usepackage[margin=1in]{geometry}
\usepackage{amsmath}
\usepackage{listings}
\usepackage{xcolor}
\usepackage{graphicx}
\usepackage{tcolorbox}
\usepackage{multicol}
\usepackage{titlesec}
\usepackage{enumitem}
\usepackage{makecell}

\definecolor{codegray}{gray}{0.95}

\lstset{
    basicstyle=\ttfamily\small,
    frame=single,
    breaklines=true,
    breakatwhitespace=false,
    postbreak=\mbox{\textcolor{red}{$\hookrightarrow$}\space},
    numbers=left,
    captionpos=b,
    showstringspaces=false,
    tabsize=4,
    language=C++,
    escapeinside={(*@}{@*)},
    breakautoindent=true
}

\title{\textbf{Radix Sort}}
\author{Varun Kumar}
\date{July 5, 2025}

\begin{document}

\maketitle

\section*{01. Logic}

Radix Sort is a non-comparative sorting algorithm that sorts integers by processing individual digits. It processes digits from least significant digit (LSD) to most significant digit (MSD), or vice versa.

\begin{tcolorbox}[
  colback=white,
  colframe=black,
  title=Key Idea
]
Sort the numbers digit by digit using a stable sorting algorithm (like Counting Sort), starting from the least significant digit.
\end{tcolorbox}

\section*{02. Number of Passes and Behavior}

Let $n$ be the number of elements, and $d$ be the number of digits in the maximum number.

\begin{itemize}
    \item The number of passes = $d$
    \item Each pass uses a stable sort (usually Counting Sort)
\end{itemize}

\section*{03. Optimal Behavior}

Radix Sort is efficient when the range of digits ($d$) is small and the number of elements ($n$) is large. It avoids comparisons and is ideal for sorting integers and fixed-length strings.

\section*{04. Pseudocode}

\begin{lstlisting}[language=Python]
function radixSort(arr):
    max_digit = number of digits in the largest number
    for digit_pos in range(0, max_digit):
        stable_sort(arr, digit_pos)
\end{lstlisting}

\begin{tcolorbox}[
  colback=white,
  colframe=black,
  title=Stability Note
]
Radix Sort is stable if the sorting algorithm used in each digit pass is stable (e.g., Counting Sort).
\end{tcolorbox}

\section*{05. Example Walkthrough}

Given: \texttt{[170, 45, 75, 90, 802, 24, 2, 66]}

\subsection*{Pass 1 (LSD - unit digit)}
\texttt{[170, 90, 802, 2, 24, 45, 75, 66]}

\subsection*{Pass 2 (tens digit)}
\texttt{[802, 2, 24, 45, 66, 170, 75, 90]}

\subsection*{Pass 3 (hundreds digit)}
\texttt{[2, 24, 45, 66, 75, 90, 170, 802]}

\newpage
\section*{06. Python Code with Explanation}

\begin{lstlisting}[language=Python]
def counting_sort(arr, exp):
    n = len(arr)
    output = [0] * n
    count = [0] * 10

    # Count digits
    for i in range(n):
        index = (arr[i] // exp) % 10
        count[index] += 1

    # Accumulate counts
    for i in range(1, 10):
        count[i] += count[i - 1]

    # Build output
    for i in range(n - 1, -1, -1):
        index = (arr[i] // exp) % 10
        output[count[index] - 1] = arr[i]
        count[index] -= 1

    # Copy to arr
    for i in range(n):
        arr[i] = output[i]

def radix_sort(arr):
    max_val = max(arr)
    exp = 1
    while max_val // exp > 0:
        counting_sort(arr, exp)
        exp *= 10
\end{lstlisting}

\section*{07. Why is Radix Sort Stable?}
Radix Sort processes digits one place at a time starting from the least significant digit (LSD), and it uses a \textbf{stable sorting algorithm} (like Counting Sort) in each pass. Stability in these intermediate passes ensures that:

\begin{itemize}[leftmargin=1.5em]
    \item Elements with the same digit maintain their original relative order.
    \item When a more significant digit is sorted later, it doesn't disturb the previous ordering.
\end{itemize}

Hence, by using a stable sort at each digit level, the overall Radix Sort becomes stable.

\section*{08. Is Radix Sort In-Place?}
Radix Sort is \textbf{not in-place} because it requires additional memory for:

\begin{itemize}[leftmargin=1.5em]
    \item Temporary output array (same size as input) to store the sorted elements after each digit pass.
    \item Counting array (usually of size 10 for base-10 digits) used in Counting Sort.
\end{itemize}

Thus, its space complexity is $O(n + k)$, where:
\begin{itemize}
    \item $n$ = number of elements
    \item $k$ = range of digits (usually 10 for decimal numbers)
\end{itemize}

\section*{09. Can Radix Sort Be Made In-Place?}
In general, \textbf{Radix Sort is not in-place} due to its reliance on an auxiliary output array to maintain stability across digit-wise passes.

\textbf{Why not in-place?}
\begin{itemize}[leftmargin=1.5em]
    \item Requires $O(n)$ extra space for output in each pass.
    \item Avoiding extra space would break the \textbf{stability} or \textbf{increase complexity}.
\end{itemize}

Some theoretical attempts exist to make Radix Sort in-place, but they are not practical due to complexity, instability, or loss of linear-time performance.

\section*{10. Why can't Radix Sort be made in-place without breaking stability or increasing complexity?}

\textbf{Answer:} Radix Sort works by sorting digits one at a time (typically starting from the least significant digit), and in each digit pass, it uses a \textbf{stable sort} like Counting Sort to preserve the relative order of elements with equal digits.

To maintain \textbf{stability}, we must:
\begin{itemize}[leftmargin=1.5em]
    \item Use an auxiliary output array to store the sorted result.
    \item Copy back this output into the original array after each pass.
\end{itemize}

Removing the auxiliary array (to make it in-place) would:
\begin{enumerate}[leftmargin=1.5em]
    \item \textbf{Break Stability:} Direct in-place swapping can rearrange equal elements, violating stability.
    \item \textbf{Increase Complexity:} Advanced in-place stable sorting techniques exist, but they require extra logic (like complex cycle-leader transformations or linked-index simulation), which leads to:
    \begin{itemize}
        \item More comparisons and swaps
        \item Harder implementation
        \item Loss of linear-time advantage
    \end{itemize}
\end{enumerate}

\textbf{Conclusion:} While in theory, it is possible to design an in-place version of Radix Sort, it is \textit{not practical} because:
\begin{itemize}
    \item It would no longer run in linear time.
    \item It may no longer remain stable.
    \item It would require complex and error-prone logic.
\end{itemize}

Hence, \textbf{standard Radix Sort is not in-place}, and efforts to make it in-place come at the cost of either \textbf{breaking stability} or \textbf{losing linear time efficiency}.






\section*{11. Best Case and Worst Case Using a Single Example}

Let us consider the array:

\[
\texttt{[170, 45, 75, 90, 802, 24, 2, 66]}
\]

We will use this same example to illustrate both the best-case and worst-case scenarios for Radix Sort.

\subsection*{Best Case Scenario}

\begin{itemize}[leftmargin=1.5em]
    \item All elements are uniformly distributed over digit positions.
    \item Digit range is small (e.g., base 10), and all numbers have nearly the same number of digits.
    \item \textbf{No significant digit collisions} at each place value, so Counting Sort per digit is efficient.
    \item Array size $n = 8$, maximum digits $d = 3$ (e.g., 802 has 3 digits).
\end{itemize}

\begin{tcolorbox}[
  colback=white,
  colframe=black,
  title=Best Case Time Complexity
]
\[
T(n) = O(d \cdot (n + k)) = O(3 \cdot (8 + 10)) = O(1) \cdot O(n) = O(n)
\]
Where:
\begin{itemize}
  \item $d =$ number of digits (3)
  \item $k =$ base/range of digits (10)
\end{itemize}
\end{tcolorbox}

\subsection*{Worst Case Scenario}

\begin{itemize}[leftmargin=1.5em]
    \item Numbers have large variation in digit lengths (e.g., mixing 3-digit and 10-digit numbers).
    \item Very large base $k$ or inefficient implementation of Counting Sort.
    \item Overhead in each digit pass increases due to long digit ranges.
    \item Still better than $O(n^2)$ algorithms, but constants can degrade performance.
\end{itemize}

\begin{tcolorbox}[
  colback=white,
  colframe=black,
  title=Worst Case Time Complexity
]
\[
T(n) = O(d \cdot (n + k)) \quad \text{where $d$ is large, e.g., 10 or more digits.}
\]
Even with same 8 elements, if max number = $9876543210$, then $d = 10$:

\[
T(n) = O(10 \cdot (8 + 10)) = O(180)
\]
\end{tcolorbox}

\subsection*{Summary}

\begin{center}
\begin{tabular}{|c|l|l|}
\hline
\textbf{Scenario} & \textbf{Condition} & \textbf{Time Complexity} \\
\hline
Best Case   & Uniform digit length, small $d$ & $O(n)$ \\
Worst Case  & Large $d$, large base $k$       & $O(d \cdot (n + k))$ \\
\hline
\end{tabular}
\end{center}



\section*{12. Time \& Space Complexity and Its Properties}

\begin{center}
\begin{tabular}{|l|l||l|l|}
\hline
\textbf{Case}        & \textbf{Complexity}       & \textbf{Property}   & \textbf{Value} \\
\hline
Best Case            & $O(n \cdot k)$            & Stable              & Yes            \\
Average Case         & $O(n \cdot k)$            & In-place            & No             \\
Worst Case           & $O(n \cdot k)$            & Adaptive            & No             \\
\hline
\noalign{\vskip 2pt}
Space Complexity     & $O(n + k)$                & Recursive           & No             \\
\hline
\end{tabular}
\end{center}

\textbf{Note:}
\begin{itemize}[leftmargin=1.5em]
    \item Actually $O(k \cdot (n + b)) \approx O(n \cdot k)$
	\begin{itemize}[leftmargin=1.5em]
		\item Where $n$ = number of elements, $k$ = number of digits and $b$ = base 
	\end{itemize}
    \item Radix Sort is faster than comparison-based $O(n \log n)$ sorts when $k$ is small and digits are bounded.
    \item Not adaptive: it does the same number of passes regardless of initial order.
    \item Not in-place: requires auxiliary arrays (e.g., output array in Counting Sort).
\end{itemize}

\newpage
\section*{14. C++ Implementation of Radix Sort (LSD)}

\subsection*{Code with Explanation}

\begin{lstlisting}[language=C++, caption=Radix Sort in C++, numbers=left]
#include <iostream>
#include <vector>
#include <algorithm>
using namespace std;

// Get the maximum element in the array
int getMax(const vector<int>& arr) {
    return *max_element(arr.begin(), arr.end());
}

// Counting Sort used for a specific digit (exp = 1, 10, 100, ...)
void countingSort(vector<int>& arr, int exp) {
    int n = arr.size();
    vector<int> output(n);       // Output array to store sorted values
    int count[10] = {0};         // Count array for digits 0-9

    // Count occurrences of digits at 'exp' place
    for (int i = 0; i < n; i++)
        count[(arr[i] / exp) % 10]++;

    // Convert count[] to actual positions
    for (int i = 1; i < 10; i++)
        count[i] += count[i - 1];

    // Build output[] using reverse traversal for stability
    for (int i = n - 1; i >= 0; i--) {
        int digit = (arr[i] / exp) % 10;
        output[count[digit] - 1] = arr[i];
        count[digit]--;
    }

    // Copy back to original array
    for (int i = 0; i < n; i++)
        arr[i] = output[i];
}

// Main Radix Sort function
void radixSort(vector<int>& arr) {
    int maxVal = getMax(arr);

    // Sort each digit (unit, tens, hundreds, etc.)
    for (int exp = 1; maxVal / exp > 0; exp *= 10)
        countingSort(arr, exp);
}

// Utility function to print array
void printArray(const vector<int>& arr) {
    for (int num : arr)
        cout << num << " ";
    cout << endl;
}

// Driver code
int main() {
    vector<int> arr = {170, 45, 75, 90, 802, 24, 2, 66};

    cout << "Original array:\n";
    printArray(arr);

    radixSort(arr);

    cout << "Sorted array:\n";
    printArray(arr);

    return 0;
}
\end{lstlisting}

\subsection*{Explanation of the Code}

\begin{itemize}[leftmargin=1.5em]
    \item \texttt{getMax()} finds the largest number to determine the number of digit passes.
    \item \texttt{countingSort()} is called for each digit position (units, tens, hundreds, etc.) using \texttt{exp} as the digit place.
    \item Counting Sort is stable because we fill the \texttt{output[]} array from the end of the original array.
    \item \texttt{radixSort()} loops over digit places until all digits of the largest number are processed.
    \item \texttt{printArray()} is a helper to display the array before and after sorting.
\end{itemize}

\subsection*{Properties of This Implementation}

\begin{itemize}[leftmargin=1.5em]
    \item \textbf{Stable:} Yes, due to reverse traversal in Counting Sort.
    \item \textbf{Not In-Place:} Uses extra memory for the output array.
    \item \textbf{Time Complexity:} $O(n \cdot d)$ where $d$ is the number of digits in the max element.
    \item \textbf{Space Complexity:} $O(n + k)$ where $k = 10$ (digits from 0 to 9).
    \item \textbf{Limitation:} Only supports non-negative integers in this version.
\end{itemize}


\section*{15. Handling Negative Numbers in Radix Sort}

\begin{tcolorbox}[
  colback=white,
  colframe=black,
  title=Problem
]
Standard Radix Sort (especially LSD variant) works only on non-negative integers because it processes digits. Negative numbers contain a sign, and digit-wise processing is undefined for negative values.
\end{tcolorbox}

\subsection*{Solution Strategy}

To extend Radix Sort to handle both positive and negative integers:

\begin{enumerate}[leftmargin=1.5em]
    \item \textbf{Separate the array into:}
    \begin{itemize}
        \item Negative numbers
        \item Non-negative numbers
    \end{itemize}
    
    \item \textbf{Take absolute values} of negative numbers before sorting.

    \item \textbf{Apply Radix Sort:}
    \begin{itemize}
        \item On positive numbers as-is
        \item On absolute values of negative numbers
    \end{itemize}
    
    \item \textbf{Reverse the sorted negative list} (because more negative values come first).

    \item \textbf{Convert back to negative} by multiplying with \(-1\).

    \item \textbf{Concatenate:}
    \begin{itemize}
        \item Final array = Sorted negatives (in reverse) + sorted positives
    \end{itemize}
\end{enumerate}

\subsection*{Example}

\[
\text{Input: } [-170, 45, -75, 90, -802, 24, 2, -66]
\]

\begin{itemize}
    \item Negatives: \texttt{[-170, -75, -802, -66]}
    \item Positives: \texttt{[45, 90, 24, 2]}
    \item Absolute Negatives: \texttt{[170, 75, 802, 66]}
\end{itemize}

Apply Radix Sort:

\begin{itemize}
    \item Sorted Positives: \texttt{[2, 24, 45, 90]}
    \item Sorted Absolute Negatives: \texttt{[66, 75, 170, 802]}
\end{itemize}

Reverse and restore negatives:

\[
\text{Sorted Negatives: } [-802, -170, -75, -66]
\]

\[
\textbf{Final Output: } [-802, -170, -75, -66, 2, 24, 45, 90]
\]

\subsection*{Conclusion}

This method:
\begin{itemize}
    \item Preserves Radix Sort’s efficiency: $O(n \cdot d)$
    \item Requires temporary arrays for splitting and merging
    \item Ensures correct sorting of negative values without modifying the core algorithm
\end{itemize}


\end{document}
