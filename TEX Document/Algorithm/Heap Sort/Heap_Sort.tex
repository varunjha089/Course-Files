\documentclass[12pt]{article}
\usepackage[margin=1in]{geometry}
\usepackage{fancyhdr}
\usepackage{listings}
\usepackage{xcolor}
\usepackage{titlesec}
\usepackage{tcolorbox}

% Header and footer setup
\pagestyle{fancy}
\fancyhf{}
\rhead{Date: \today}
\lhead{Subject: Algorithms}
\cfoot{\thepage}

% Section formatting
\titleformat{\section}{\large\bfseries\color{blue}}{\thesection}{1em}{}
\titleformat{\subsection}{\normalsize\bfseries\color{cyan}}{\thesubsection}{1em}{}

% Python code style
\lstdefinestyle{mypython}{
  language=Python,
  basicstyle=\ttfamily\small,
  keywordstyle=\color{blue}\bfseries,
  stringstyle=\color{red},
  commentstyle=\color{gray}\itshape,
  showstringspaces=false,
  frame=single,
  breaklines=true,
  tabsize=2,
  captionpos=b
}

% Minimal Cornell box (compatible version)
\newtcolorbox{cornellbox}[1][]{%
  colback=white,
  colframe=black,
  fonttitle=\bfseries,
  title=#1,
  boxrule=0.5pt,
  sharp corners
}

\title{\textbf{Cornell Notes: Heap Sort Algorithm with Python}}
\date{}
\author{}

\begin{document}
\maketitle

\begin{cornellbox}[Cues / Questions]
\begin{itemize}
  \item What is Heap Sort?
  \item How does heapify work?
  \item What is time complexity?
  \item Is it in-place?
  \item Python code?
\end{itemize}
\end{cornellbox}

\begin{cornellbox}[Notes]
\textbf{Heap Sort} is a comparison-based, in-place sorting algorithm that uses a \textbf{binary heap} (usually a max heap).

\textbf{Steps:}
\begin{enumerate}
  \item Convert array into a max heap.
  \item Repeatedly remove the root (largest) and move to the end.
  \item Heapify the reduced heap.
\end{enumerate}

\textbf{Time Complexity:}
\begin{itemize}
  \item Build Heap: \( O(n) \)
  \item Reheapify: \( O(\log n) \) for each of \( n \) elements
  \item Total: \( O(n \log n) \)
\end{itemize}

\textbf{Space Complexity:} \( O(1) \) (in-place)

\textbf{Stability:} Not stable
\end{cornellbox}

\begin{cornellbox}[Python Code]
\begin{lstlisting}[style=mypython, caption={Heap Sort in Python}]
def heapify(arr, n, i):
    largest = i
    left = 2*i + 1
    right = 2*i + 2

    if left < n and arr[left] > arr[largest]:
        largest = left
    if right < n and arr[right] > arr[largest]:
        largest = right

    if largest != i:
        arr[i], arr[largest] = arr[largest], arr[i]
        heapify(arr, n, largest)

def heap_sort(arr):
    n = len(arr)

    # Build max heap
    for i in range(n//2 - 1, -1, -1):
        heapify(arr, n, i)

    # Extract elements from heap
    for i in range(n-1, 0, -1):
        arr[i], arr[0] = arr[0], arr[i]
        heapify(arr, i, 0)

arr = [4, 10, 3, 5, 1]
heap_sort(arr)
print("Sorted:", arr)
\end{lstlisting}
\end{cornellbox}

\begin{cornellbox}[Summary]
Heap Sort is an efficient, in-place sorting algorithm with a guaranteed time complexity of \( O(n \log n) \). It is useful when memory is constrained and stable sorting is not required.
\end{cornellbox}

\end{document}
