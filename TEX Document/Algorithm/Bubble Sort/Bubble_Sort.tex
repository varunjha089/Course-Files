\documentclass[14pt]{extarticle}
\usepackage[margin=1in]{geometry}
\usepackage{amsmath}
\usepackage{listings}
\usepackage{xcolor}
\usepackage{graphicx}
\usepackage{tcolorbox}
\usepackage{multicol}
\usepackage{titlesec}
\usepackage{enumitem}
\usepackage{makecell}


\definecolor{codegray}{gray}{0.95}
\lstset{
    basicstyle=\ttfamily\small,
    frame=single,
    breaklines=true,
    columns=fullflexible,
}

\title{\textbf{Bubble Sort}}
\author{Varun Kumar}
\date{July 4, 2025}

\begin{document}

\maketitle

\section*{1. Logic}

Bubble Sort is a comparison-based sorting algorithm that works by repeatedly swapping adjacent elements if they are in the wrong order. The largest unsorted element "bubbles" to the correct position in each pass.


\begin{tcolorbox}[
  colback=white,        % white background
  colframe=black,       % black border
  title=key Idea
]
After the $i^{th}$ pass, the $i^{th}$ largest elements is at it's sorted position.
\end{tcolorbox}

\section*{2. Number of Comparisons and Swaps}

Let $n$ be the number of elements.

\subsection*{Worst Case (Reverse Sorted)}

\begin{itemize}
    \item Comparisons: $\frac{n(n-1)}{2}$
    \item Swaps: $\frac{n(n-1)}{2}$
\end{itemize}

\subsection*{Best Case (Already Sorted)}

\begin{itemize}
    \item Comparisons: $n - 1$ (with optimization)
    \item Swaps: $0$
\end{itemize}

\section*{3. Optimal Bubble Sort (Early Termination)}

Introduce a flag to detect if any swaps happened during the pass. If no swaps occur, the array is already sorted.

\section*{4. Pseudocode}

\begin{lstlisting}[language=Python]
function bubbleSort(arr):
    n = length(arr)
    for i = 0 to n-1:
        swapped = false
        for j = 0 to n-i-2:
            if arr[j] > arr[j+1]:
                swap arr[j] and arr[j+1]
                swapped = true
        if not swapped:
            break
\end{lstlisting}

\begin{tcolorbox}[
  colback=white,        % white background
  colframe=black,       % black border
  title=Stability Note
]
Bubble Sort is stable by default because it swaps elements only when \texttt{arr[j] > arr[j+1]}.

If we change the condition to \texttt{arr[j] $\geq$ arr[j+1]}, it may swap equal elements, making the sort unstable.
\end{tcolorbox}

\section*{5. Example Walkthrough}

Given: \texttt{[5, 1, 4, 2, 8]}

\subsection*{Pass 1}
\texttt{[5, 1, 4, 2, 8]} $\Rightarrow$ \texttt{[1, 5, 4, 2, 8]}  
$\Rightarrow$ \texttt{[1, 4, 5, 2, 8]}  
$\Rightarrow$ \texttt{[1, 4, 2, 5, 8]}

\subsection*{Pass 2}
\texttt{[1, 4, 2, 5, 8]} $\Rightarrow$ \texttt{[1, 2, 4, 5, 8]}

\subsection*{Pass 3}
\texttt{[1, 2, 4, 5, 8]} → No swaps → Done


\section*{6. Python Code with Explanation}

\begin{lstlisting}[language=Python]
def bubble_sort(arr):
    # Get the length of the list
    n = len(arr)

    # Traverse the list n times
    for i in range(n):
        # Initialize swapped flag as False at the beginning of each pass
        swapped = False

        # Perform comparisons up to the unsorted portion (n - i - 1)
        for j in range(0, n - i - 1):
            # Compare adjacent elements
            if arr[j] > arr[j + 1]:
                # Swap if they are in the wrong order
                arr[j], arr[j + 1] = arr[j + 1], arr[j]
                # Set flag to True to indicate a swap happened
                swapped = True

        # If no elements were swapped during the inner loop, the list is sorted
        if not swapped:
            break

    # Return the sorted list
    return arr
\end{lstlisting}


\section*{7. Time \& Space Complexity and It's Properties}

\begin{center}
\begin{tabular}{|l|l||l|l|}
\hline
\textbf{Case}     & \textbf{Complexity}   & \textbf{Property}     & \textbf{Value} \\
\hline
Best Case         & $O(n)$                & Stable                & Yes            \\
Average Case      & $O(n^2)$              & In-place              & Yes            \\
Worst Case        & $O(n^2)$              & Adaptive              & Yes (optimized)\\
\hline
\noalign{\vskip 2pt} % Adds vertical space
Space Complexity  & $O(1)$ (in-place)     & Recursive             & No             \\
\hline
\end{tabular}
\end{center}


\end{document}
