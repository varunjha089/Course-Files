\documentclass[14pt]{extarticle}
\usepackage[margin=1in]{geometry}
\usepackage{amsmath}
\usepackage{enumitem}
\usepackage{titlesec}
\usepackage{algorithm}
\usepackage{algpseudocode}
\usepackage{hyperref}
\usepackage{caption}
\usepackage{float}
\usepackage{listings}
\usepackage{xcolor}

% Algorithm listings
\lstdefinestyle{python}{
  language=Python,
  basicstyle=\ttfamily\small,
  keywordstyle=\color{blue},
  commentstyle=\color{gray},
  stringstyle=\color{orange},
  showstringspaces=false,
  numbers=left,
  numberstyle=\tiny\color{gray},
  breaklines=true,
  frame=single,
  tabsize=4,
  captionpos=b
}

\lstdefinestyle{cpp}{
  language=C++,
  basicstyle=\ttfamily\small,
  keywordstyle=\color{blue},
  commentstyle=\color{gray},
  stringstyle=\color{orange},
  showstringspaces=false,
  numbers=left,
  numberstyle=\tiny\color{gray},
  breaklines=true,
  frame=single,
  tabsize=4,
  captionpos=b
}

\title{N-Queens Problem: A Backtracking Approach}
\author{Varun Kumar}
\date{\today}

\begin{document}
\maketitle
\newpage
\tableofcontents
\newpage
\lstlistoflistings
\listofalgorithms
\newpage

\section{Introduction}
The N-Queens problem is a classic backtracking puzzle where the goal is to place $N$ queens on an $N \times N$ chessboard such that no two queens threaten each other.

\textbf{Constraints:}
\begin{itemize}
    \item No two queens can be in the same row.
    \item No two queens can be in the same column.
    \item No two queens can be on the same diagonal.
\end{itemize}

\section{Problem Statement}
Place $N$ queens on an $N \times N$ board so that:
\begin{itemize}
    \item No queen can attack another.
    \item Return all distinct solutions (or any one).
\end{itemize}

\section{Approach: Backtracking}
\begin{itemize}
    \item Start with the first row.
    \item Try placing a queen in each column.
    \item If placing is safe, move to the next row.
    \item If not, backtrack and try the next column.
\end{itemize}

\subsection*{Key Conditions}
To check if placing a queen at $(row, col)$ is safe:
\begin{itemize}
    \item No other queen is in the same column.
    \item No other queen on upper-left diagonal.
    \item No other queen on upper-right diagonal.
\end{itemize}

\section{Pseudocode}
\begin{algorithm}
\caption{Solve N-Queens}
\begin{algorithmic}[1]
\Procedure{solveNQueens}{$n$}
    \State $board \gets$ empty $n \times n$ grid
    \State $result \gets$ empty list
    \State \Call{Backtrack}{0}
    \State \Return $result$
\EndProcedure

\Procedure{Backtrack}{$row$}
    \If{$row == n$}
        \State Add current board to $result$
        \State \Return
    \EndIf
    \For{$col = 0$ to $n-1$}
        \If{Safe to place at $(row, col)$}
            \State Place queen at $(row, col)$
            \State \Call{Backtrack}{$row + 1$}
            \State Remove queen from $(row, col)$
        \EndIf
    \EndFor
\EndProcedure
\end{algorithmic}
\end{algorithm}

\section{Python Implementation}
\begin{lstlisting}[style=python, caption={N-Queens in Python}]
def solve_n_queens(n):
    board = [['.' for _ in range(n)] for _ in range(n)]
    result = []

    def is_safe(row, col):
        for i in range(row):
            if board[i][col] == 'Q': return False
            if col - (row - i) >= 0 and board[i][col - (row - i)] == 'Q': return False
            if col + (row - i) < n and board[i][col + (row - i)] == 'Q': return False
        return True

    def backtrack(row):
        if row == n:
            result.append([''.join(r) for r in board])
            return
        for col in range(n):
            if is_safe(row, col):
                board[row][col] = 'Q'
                backtrack(row + 1)
                board[row][col] = '.'

    backtrack(0)
    return result
\end{lstlisting}

\newpage
\section{C++ Implementation}
\begin{lstlisting}[style=cpp, caption={N-Queens in C++}]
#include <iostream>
#include <vector>
using namespace std;

bool isSafe(int row, int col, vector<string> &board, int n) {
    for (int i = 0; i < row; ++i) {
        if (board[i][col] == 'Q') return false;
        if (col - (row - i) >= 0 && board[i][col - (row - i)] == 'Q') return false;
        if (col + (row - i) < n && board[i][col + (row - i)] == 'Q') return false;
    }
    return true;
}

void solve(int row, vector<string> &board, vector<vector<string>> &result, int n) {
    if (row == n) {
        result.push_back(board);
        return;
    }
    for (int col = 0; col < n; ++col) {
        if (isSafe(row, col, board, n)) {
            board[row][col] = 'Q';
            solve(row + 1, board, result, n);
            board[row][col] = '.';
        }
    }
}

vector<vector<string>> solveNQueens(int n) {
    vector<vector<string>> result;
    vector<string> board(n, string(n, '.'));
    solve(0, board, result, n);
    return result;
}
\end{lstlisting}

\newpage
\section{Key Points to Remember}
\begin{enumerate}
    \item N-Queens is a classic backtracking problem.
    \item A solution is valid if no two queens threaten each other.
    \item We only place one queen per row.
    \item Use diagonals and column checks to prune invalid states.
    \item Base case: all $n$ queens are placed → store result.
    \item Recursive case: try placing a queen in all columns.
    \item If placing fails → backtrack and try other positions.
    \item Total solutions vary: 92 for N=8.
    \item Very useful in solving constraint satisfaction problems.
    \item Can be extended to optimization and parallel problems.
\end{enumerate}

\section{Real-World Applications}
\begin{itemize}
    \item \textbf{Constraint Solvers:} Used in scheduling and resource allocation.
    \item \textbf{Circuit Design:} Placing non-overlapping components.
    \item \textbf{Parallel Processing:} Task assignments without interference.
    \item \textbf{Sudoku Solvers:} Similar combinatorial strategy.
    \item \textbf{AI Problem Solving:} CSP-based agents use similar logic.
\end{itemize}

\section{Conclusion}
The N-Queens problem demonstrates a powerful application of backtracking for constraint satisfaction. It helps build strong recursive reasoning, efficient state pruning, and is foundational for problems in AI, search, and optimization domains.

\end{document}
